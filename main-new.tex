\documentclass[12pt]{article}
\usepackage{amsmath}
\usepackage{fullpage}
\usepackage{amsfonts}
\usepackage{amssymb}
\usepackage{graphicx}
\usepackage{verbatim}
\immediate\write18{texcount -tex -sum  \jobname.tex > \jobname.wordcount.tex}
\usepackage[ruled,vlined]{algorithm2e}
\usepackage[titles]{tocloft}%----------------------------------------
\usepackage{mathrsfs}
\usepackage{bm}
\usepackage{enumitem}
\usepackage{amsthm}
\usepackage{chngcntr}
\usepackage{titlesec}
\usepackage{stfloats}
\usepackage{hyperref}
\hypersetup{linktocpage}
\hypersetup{
    colorlinks,
    citecolor=red,
    filecolor=black,
    linkcolor=blue ,
    urlcolor=black
}
%%%%%%%%%%%%%%%%%%%%%%%%%%%%%%%%%%%%%%%%%%%%%%%%%%%%%%%%%%%%%%%%%%%%%%%%%%%%%%
%%%%%%%%%%%%%%%%%%%%%%%%%%%%%%%%%%%%%%%%%%%%%%%%%%%%%%%%%%%%%%%%%%%%%%%%%%%%%%
% new commands and shortcuts
%%%%%%%%%%%%%%%%%%%%%%%%%%%%%%%%%%%%%%%%%%%%%%%%%%%%%%%%%%%%%%%%%%%%%%%%%%%%%%


\newcommand\myeq{\mathrel{\stackrel{\makebox[0pt]{\mbox{\normalfont\tiny ?}}}{=}}}
% script variables 
\def\sO{\mathscr{O}}
\def\sL{\mathscr{L}}
\def\sLA{\mathscr{L}^{\mA}}
\def\sLfA{\mathscr{L}^{\A}}
\def\sA{\mathscr{A}}
\def\scrQ{\ensuremath{\mathscr{Q}}}
\def\rs{\mathscr{R}}
\def\sW{\tilde{W}}
\def\sI{\mathscr{I}}
% frak variables 
%%%%%%%%%%%%%%%%%%%%%%%%%%%%%%%%%%%%%%%%%%%%%%%%%%%%%%%%%%%%%%%%%%%%%%%%%%%%%%
\def\fA{{\mathfrak A}}
\def\fZ{{\mathfrak Z}}
\def\fZp{{\mathfrak Z^{'}}}
\def\fB{{\mathfrak B}}
\def\fN{{\mathfrak N}}
\def\fD{{\mathfrak D}}
\def\fp{\mathfrak{P}}
\def\Z{\mathfrak{Z}}
\def\A{\mathfrak{A}}
%\def\L{\mathfrak{L}}
%%%%%%%%%%%%%%%%%%%%%%%%%%%%%%%%%%%%%%%%%%%%%%%%%%%%%%%%%%%%%%%%%%%%%%%%%%%%%%
% bold variables 
\def\bs{\textit{\textbf{s}}}
\def\bz{{{\bm 0}}}
\def\Lb{\textit{\textbf{L}}}
\def\Xb{\textit{\textbf{X}}}
\def\Lambdab{\bm{\Lambda}}
\def\Thetab{\bm{\Theta}}
\def\thetab{\bm{\vartheta}}
\def\mA{{\bm A}}
\def\ub{{\bm u}}
\def\lb{{\bm l}}
\def\lxb{{\bm l_{\xb}}}
\def\ax{\bm \alpha_{\xb}} 
\def\xb{{\bm x}}
\def\yb{{\bm y}}
\def\fb{\textit{\textbf{f}}}
\def\gb{\textit{\textbf{g}}}
\def\ab{\textit{\textbf{a}}}
\def\pb{\textit{\textbf{p}}}
\def\ajb{\overline{a_j}}
\def\bjb{\overline{b_{j,l}}}
\def\pjb{\overline{P_{j}}}
\def\bI{\textbf{I}}
%%%%%%%%%%%%%%%%%%%%%%%%%%%%%%%%%%%%%%%%%%%%%%%%%%%%%%%%%%%%%%%%%%%%%%%%%%%%%%
% tilde variables 
\def\jt{\widetilde{J}}
\def\At{\widetilde{A}}
\def\Yt{\widetilde{Y}}
\def\dtt{\widetilde{d}}
\def\kt{\widetilde{k}}
\def\Dt{\widetilde{D}}
%% \def\dt{\widetilde{d}}
%%%%%%%%%%%%%%%%%%%%%%%%%%%%%%%%%%%%%%%%%%%%%%%%%%%%%%%%%%%%%%%%%%%%%%%%%%%%%%
% greek letters  
\def\vt{\vartheta}
\def\d{\delta}
\def\D{\Delta}
%%%%%%%%%%%%%%%%%%%%%%%%%%%%%%%%%%%%%%%%%%%%%%%%%%%%%%%%%%%%%%%%%%%%%%%%%%%%%%
% declare math operator  
\DeclareMathOperator{\GL}{GL}
\DeclareMathOperator{\LCM}{LCM}
\DeclareMathOperator{\J}{J}
\DeclareMathOperator{\htt}{ht}
\DeclareMathOperator{\sing}{sing}
\DeclareMathOperator{\codim}{codim}
\DeclareMathOperator{\jac}{jac}
%\DeclareMathOperator{\exp}{exp}
\DeclareMathOperator{\grad}{grad}
\DeclareMathOperator{\rank}{rank}
\DeclareMathOperator{\reg}{reg}
\DeclareMathOperator{\rk}{rank}
\def\minors{\textrm{Minors}(F,p)}
\def\minorsA{\textrm{Minors}(F^{\mA},p)}
\def\minorsfA{\textrm{Minors}(F^{\A},p)}

%%%%%%%%%%%%%%%%%%%%%%%%%%%%%%%%%%%%%%%%%%%%%%%%%%%%%%%%%%%%%%%%%%%%%%%%%%%%%%
% other math notation shortcuts  
\def\pa{\partial}
\newcommand{\softO}{{O^{\sim}}}
\def\dt{s}
%%%%%%%%%%%%%%%%%%%%%%%%%%%%%%%%%%%%%%%%%%%%%%%%%%%%%%%%%%%%%%%%%%%%%%%%%%%%%%
% sets of numbers
\newcommand{\ZZ}{{\mathbb{Z}}}
\def\C{\mathbb{C}}
\def\Q{\mathbb{Q}}
\def\R{\mathbb{R}}
\def\K{\mathbb{K}}
\def\pr{\mathbb{P}}
\def\P{\mathscr{P}}
%%%%%%%%%%%%%%%%%%%%%%%%%%%%%%%%%%%%%%%%%%%%%%%%%%%%%%%%%%%%%%%%%%%%%%%%%%%%%%
% polar varieties
% minors 
\def\Wi{W(\pi_i,F)}
\def\WiA{W(\pi_i,V(F^{\mA}))}
\def\WifA{W(\pi_i,V(F^{\A}))}
\def\I{\mathfrak{I}}
\def\IA{\mathfrak{I}^{\mA}}
\def\IfA{\mathfrak{I}^{\A}}
\def\Ii{\mathfrak{I}(i,F)}
\def\IiA{\mathfrak{I}(i,F^A)}
\def\Ir{\sqrt{\mathfrak{I}}}
\def\Iir{\sqrt{\mathfrak{I}(i,F)}}
\def\IiAr{\sqrt{\mathfrak{I}(i,F^{\mA})}}
\def\IAr{\sqrt{\mathfrak{I}^{\mA}}}
\def\IfAr{\sqrt{\mathfrak{I}^{\A}}}
\def\IifAr{\sqrt{\mathfrak{I}(\pi_i,F^{\A})}}
\def\IifA{\mathfrak{I}(\pi_i,F^{\A})}
\def\ji{\jac_{\xb}(F,i)}
\def\jiA{\jac_{\xb}(F^{\mA},i)}
\def\jifA{\jac_{\xb}(F^{\A},i)}
\DeclareMathOperator{\lag}{Lagrange}
\def\lagF{{\bm{Lagrange}}(F,i,(L_1,\hdots,L_c))}
\def\lagFA{{\bm{Lagrange}}(F^{\mA},i,(L_1,\hdots,L_p))}
\def\lagFAl{{\bm{Lagrange}}(F^{\mA},i,(l_1,\hdots,l_p))}
\def\lagFfA{{\bm{Lagrange}}(F^{\A},i,(L_1,\hdots,L_p))}
%\def\W{\mathfrak{I}}
% lagrange 
\def\Il{\mathscr{I}_{\ub}}
\def\Iil{\mathscr{I}_{\ub}(i,F)}
\def\IilA{\mathscr{I}_{\ub}(i,F^{\mA})}
\def\IilAnu{\mathscr{I}(i,F^{\mA})}
\def\IilfA{\mathscr{I}_{\ub}(i,F^{\A})}
\def\IilfAV{\mathscr{I}_{\bm V}(i,F^{\A})}

\def\IlfA{\mathscr{I}_{\ub}^{\A}}
\def\IlA{\mathscr{I}_{\ub}^{\mA}}
\def\jiA{\jac_{\xb}(F^{\mA},i)}
\def\Wil{\mathscr{W}_{\ub}(\pi_i,V)}
\def\WilA{\mathscr{W}_{\ub}(\pi_i,V^{\mA})}
\def\WilAnu{\mathscr{W}(\pi_i,V^{\mA})}
\def\WilfA{\mathscr{W}_{\ub}(\pi_i,V^{\fA})}
\def\Wl{\mathscr{W}_{\ub}}
\def\WlA{\mathscr{W}_{\ub}^{\mA}}
\def\udl{\sum_{i=1}^pu_iL_i}
\def\Vdl{\sum_{i=1}^pV_iL_i}
%%%%%%%%%%%%%%%%%%%%%%%%%%%%%%%%%%%%%%%%%%%%%%%%%%%%%%%%%%%%%%%%%%%%%%%%%%%%%%
% brackets
\def\la{\langle}
\def\ra{\rangle}
%%%%%%%%%%%%%%%%%%%%%%%%%%%%%%%%%%%%%%%%%%%%%%%%%%%%%%%%%%%%%%%%%%%%%%%%%%%%%%
% matrices 
\def\bbm{\begin{bmatrix}}
\def\ebm{\end{bmatrix}}
\def\bmat{\begin{matrix}}
\def\emat{\end{matrix}}
%%%%%%%%%%%%%%%%%%%%%%%%%%%%%%%%%%%%%%%%%%%%%%%%%%%%%%%%%%%%%%%%%%%%%%%%%%%%%%
% other shortcuts 
\newcommand\todo[1]{(\textcolor{red}{{\bf todo}}: #1)}
\def\td{(\textcolor{red}{{\bf todo}})}
\def\gip{\Gamma_i^{'}}
\def\gi{\Gamma_i}
%%%%%%%%%%%%%%%%%%%%%%%%%%%%%%%%%%%%%%%%%%%%%%%%%%%%%%%%%%%%%%%%%%%%%%%%%%%%%%
%%%%%%%%%%%%%%%%%%%%%%%%%%%%%%%%%%%%%%%%%%%%%%%%%%%%%%%%%%%%%%%%%%%%%%%%%%%%%%
% Theorems 
\newtheorem{theorem}{Theorem}[section]
\newtheorem{corollary}[theorem]{Corollary}
\newtheorem{lemma}[theorem]{Lemma}
\newtheorem{ex}[theorem]{Example}
\newtheorem{observation}[theorem]{Observation}
\newtheorem{prop}[theorem]{Proposition}
\newtheorem{definition}[theorem]{Definition}
\newtheorem{claim}[theorem]{Claim}
\newtheorem{fact}[theorem]{Fact}
\newtheorem{assumption}[theorem]{Assumption}
\newtheorem{remark}[theorem]{Remark}
\newtheorem{question}{Question}
%\newtheorem*{ex}{Example}
\newtheorem{cor}[theorem]{Corollary}
%
%
%
% Keywords command
\providecommand{\keywords}[1]
{
  \small	
  \textbf{\textit{Keywords---}} #1
}
%
%
%
\title{Bit complexity for computing one point on each connected component of a smooth real algebraic set}
\author{Jesse Elliott$\dagger$, Mark Giesbrecht$\dagger$, Eric Schost$\dagger$  \\
        \small $\dagger$David R. Cheriton School of Computer Science, University of Waterloo, On, Canada \\
}
\date{} % Comment this line to show today's date
%
%
%
\begin{document}
%
\maketitle
%
%% \tableofcontents
%% \newpage
%
%
%
\begin{abstract}
We analyze the bit complexity of an efficient
  algorithm for the computation of at least one point in each
  connected component of a smooth real algebraic set. This work is a continuation of an analysis by Elliott, Giesbrecht, and Schost ({\em On the bit complexity of finding points on connecting components of a smooth real hypersurface}, ISSAC'20), where the hypersurface case was analyzed. In this paper, we extend the analysis to the more general algebraic set case. We determine the bit complexity for computing at least one point in each connected component of a smooth real algebraic set.  
    
    The analysis is of an algorithm by Safey El Din and Schost ({\em
    Polar varieties and computation of one point in each connected
    component of a smooth real algebraic set}, ISSAC'03). This
  algorithm uses random changes of variables that are proven to
  generically ensure certain desirable geometric properties. The
  cost of the algorithm was given in an algebraic complexity
  model. We analyze the bit complexity and the error probability. And we provide a
  quantitative analysis of the genericity statements. 
  
  Let $F=(f_1,\hdots, f_s) \in \mathbb{Q}[X_1, \hdots , X_n]^s$ be a sequence of polynomials with $V = V(F) \subset \C^n$ a smooth and equidimensional variety and $\langle F \rangle \subset  \C[X_1, \hdots , X_n]$ a radical ideal. Then, computing suitable sections of the
polar varieties associated to generic projections of $V$
gives at least one point in each connected component of
$V \cap \mathbb{R}^n$. Our main algorithm solves a square system and we model polar varieties as projections of Lagrangian systems so that we can maintain square systems. This modeling introduces additional genericity assumptions which we incorporate into our analysis.   

\keywords{Real algebraic geometry; weak transversality; Noether position; complexity}
\end{abstract}
%
%
%



%%%%%%%%%%%%%%%%%%%%%%%%%%%%%%%%%%%%%%%%%%%%%%%%%%%%%%%%%%%%
%%%%%%%%%%%%%%%%%%%%%%%%%%%%%%%%%%%%%%%%%%%%%%%%%%%%%%%%%%%%
%%%%%%%%%%%%%%%%%%%%%%%%%%%%%%%%%%%%%%%%%%%%%%%%%%%%%%%%%%%%

\section{Introduction}

\paragraph*{Background and problem statement.}
Computing one point in each connected component of a real algebraic
set $S$ is a basic subroutine in real algebraic and semi-algebraic
geometry; it is also useful in its own right, since it allows one to
decide if $S$ is empty or not. 

We consider the case where $S$ is given as $S=V \cap \R^n$, where
$V=V(F) \subset \C^n$ is a complex algebraic set defined by a sequence
of polynomials $F = (f_1,\hdots,f_p)$ in
$\ZZ[X_1,\dots,X_n]$. Algorithms for this task have been known for
decades, and their complexity is to some extent well
understood. Suppose that all $f_i$'s have degree at most $d$, and
coefficients of bit-size at most $h$. Without making any assumption on
these polynomials, the algorithm given
in~\cite[Section~13.1]{BaPoRo03} solves our problem using $d^{O(n)}$
operations in $\Q$; in addition, the output of the algorithm is
represented by polynomials of degree $d^{O(n)}$, with coefficients of
bit-size $hd^{O(n)}$. The key idea behind this algorithm goes back
to~\cite{GrVo88}: sample points are found through the computation of
critical points of well-chosen functions on $V(f_1,\dots,f_p)$.

The number of connected components of $V(f_1,\dots,f_p)$ admits the
lower bound $d^{\Omega(n)}$, so up to polynomial factors this result
is optimal. However, due to the generality of the algorithm, the
constant hidden in the exponent $O(n)$ in its runtime turns out to be
rather large: the algorithm relies on infinitesimal deformations, that
affect runtime non-trivially.

In this paper, we will work under the additional assumption that
$V=V(f_1,\dots,f_p)$ is a {\em smooth} complex algebraic set, and that
$f_1,\dots,f_p$ form a regular reduced sequence (we explain these
terms in the next section)\td.  We place ourselves in the continuation of
the line of work initiated by~\cite{BaGiHeMb97}: that reference deals
with cases where $V$ is smooth and $V \cap \R^n$ is compact, pointing
out how {\em polar varieties} (that were introduced in the 1930's in
order to define characteristic classes~\cite{Piene78,Teissier88}) can
play a role in effective real geometry. This paper was extended in
several directions: to $V$ being a smooth complete intersection, still
with $V\cap \R^n$ compact~\cite{BaGiHeMb01}, then without the
compactness assumption~\cite{EMP,BaGiHePa05}; the smoothness
assumption was then partly dropped in~\cite{BaGiHe14,BaGiHeLePa12}.

Our starting point is the algorithm in~\cite{EMP}, whose assumptions
are slightly more general than ours (the  reduced regular sequence
assumption is not needed). In the cases we consider in this paper, its
runtime is ${ n \choose p}{}^{2+o(1)}d^{(4+o(1))n}$ operations in
$\Q$.  As with many results in this vein, the algorithm is randomized,
as we need to assume that we are in generic coordinates; this is done
by applying a random change of coordinates prior to all
computations. In addition, the algorithm relies on procedures for
solving systems of polynomial equations that are themselves
randomized.  Altogether, we choose $n^{O(1)}$ random vectors, each of
them in an affine space of dimension $n^{O(1)}$; every time a choice
is made, there exists a hypersurface of the parameter space that one
has to avoid in order to guarantee success. In this paper, we revisit
this algorithm and give a complete analysis of its probability of
success and its bit complexity.

This work is a continuation of the analysis of the hypersurface case
that we gave in~\cite{ElGiSc20} (that is, the case $p=1$). In this
paper, we extend this analysis to reduced regular sequences. A very
useful property in the hypersurface case is that polar varieties can
be described by straightforward equations (the partial derivatives of
the input polynomial) that form a regular sequence, at least in
generic coordinates. In higher codimension, this is not the case
anymore: the natural description of polar varieties now involves
minors of the Jacobian matrix of the input equations. The resulting
equations are in general not a complete intersection anymore, which
makes it impossible to extend direcly several arguments we used
in~\cite{ElGiSc20}. 

Our solution is to use a description of polar varieties by means of
bso-called Lagrange equations. These equations are complete
intersections (in generic coordinates), but they involve more
variables. As such, they describe algebraic sets that cover polar
varieties; we will discuss in detail the relationship between these
two presentations.


\paragraph*{Data structures.} 
The output of the algorithm is a finite set in $\overline{\Q}{}^n$. To
represent it, we rely on a widely used data structure based on
univariate
polynomials~\cite{Kronecker82,Macaulay16,GiMo89,GiHeMoPa95,ABRW,GiHaHeMoMoPa97,GiHeMoMoPa98,Rouillier99}.
For a zero-dimensional algebraic set $S \subset \C^n$ defined over
$\Q$, a {\em zero-dimensional parameterization}
$\scrQ=((q,v_1,\dots,v_n),\lambda)$ of $S$ consists in polynomials
$(q,v_1,\dots,v_n)$, such that $q\in \Q[T]$ is monic and squarefree,
all $v_i$'s are in $\Q[T]$ and satisfy $\deg(v_i) < \deg(q)$, and in a
$\Q$-linear form $\lambda$ in variables $X_1,\dots,X_n$, such that
\begin{itemize}
\item $\lambda(v_1,\dots,v_n)=T q' \bmod q$;
\item we have the equality
  $S=\left \{\left(
      \frac{v_1(\tau)}{q'(\tau)},\dots,\frac{v_n(\tau)}{q'(\tau)}\right
    ) \ \mid \ q(\tau)=0 \right \}.$
\end{itemize}
The constraint on $\lambda$ says that the roots of $q$ are the values
taken by $\lambda$ on $S$. The parameterization of the coordinates by
rational functions having $q'$ as a denominator goes back
to~\cite{Kronecker82,Macaulay16}: as pointed out in~\cite{ABRW}, it
allows one to control precisely the size of the coefficients of
$v_1,\dots,v_n$.

\paragraph*{Main result.} To state our main result, we need to define 
the {\em height} of a rational number, and of a polynomial with
rational coefficients.

The {\em height} of a non-zero $a=u/v \in \Q$ is the maximum of
$\ln(|u|)$ and $\ln(v),$ where $u \in \mathbb{Z}$ and $v \in
\mathbb{N}$ are coprime. For a polynomial $f$ with rational
coefficients, if $v \in \mathbb N$ is the minimal common denominator
of all non-zero coefficients of $f$, then the \textit{height}
$\htt(f)$ of $f$ is defined as the maximum of the logarithms of $v$
and of the absolute values of the coefficients of $vf$.




\begin{theorem}\label{theo:main}
  Let $F= (f_1,\hdots,f_p)\in\ZZ[X_1,\hdots,X_n]^p$ be a sequence of
  polynomials with $\deg(f_i) \leq d$ and $\htt(f_i) \leq b$. Suppose
  that the ideal $\langle F \rangle \subset \C[X_1,\hdots,X_n]$ is
  radical and that $V=V(F) \subset \C^n$ is smooth with $\dim V =
  n-p$. Also suppose that $0 < \epsilon < 1.$ \td

  There exists a randomized algorithm that takes $F$ and $\epsilon$ as
  input and produces $n$ zero-dimensional parameterizations, the union
  of whose zeros includes at least one point in each connected
  component of $V(F) \cap \R^n$, with probability at least
  $1-\epsilon$. Otherwise, the algorithm either returns a proper
  subset of the points, or FAIL.  In any case, the algorithm uses
  \[
  \td
  \]
  bit operations. The polynomials in the output have degree at most
  $d^{n+p},$ and height
  \[
  \td
  \]
\end{theorem}

Here we assume that $F$ is given as a sequence of polynomials in dense
representation.  Following references such
as~\cite{GiHeMoPa95,GiHaHeMoMoPa97,GiHeMoMoPa98,BaGiHeMb97,EMP}, it
would be possible to refine the runtime estimate by assuming that $F$
is given by a {\em straight-line program} (that is, a sequence of
operations $+,-,\times$ that takes as input $X_1,\dots,X_n$ and
evaluates $F$). Any polynomial of degree $d$ in $n$ variables can be
computed by a straight-line program that does $O(d^n)$ operations:
evaluate all monomials of degree up to $d$ in $n$ variables, multiply
them by their respective coefficients and sum the results. However,
some inputs may be given by shorter straight-line program, and the
algorithm would actually be able to benefit from this.

The algorithm itself is rather simple. To describe it, we need to
define {\em polar varieties}, which will play a crucial role in this
paper. Let $V=V(F)$, for $F=(f_1,\dots,f_p)$ as in the theorem. For $i
\in \{1,\hdots,n-1\},$ denote by $\pi_i:\C^n \rightarrow \C^i$ the
projection $(x_1,\hdots,x_n) \mapsto (x_1,\hdots,x_i)$.  The $i$-th
\textit{polar variety} \[W(\pi_i,V) := \{\xb \in V~|~\dim \pi_i(T_\xb
V) < i\}\] is the set of critical points of $\pi_i$ on $V$.  We will
recall below that it is defined by the vanishing of all $p$-minors
$M_{i,1},\dots,M_{i,S_i}$ of the last $n-i$ columns of the Jacobian
matrix of $F$, together with the equations $F$ themselves (here, $S_i$
is simply the binomial number $\binom{n-i}{p}$).

In general, we cannot say much about the geometry of $W(\pi_i,V)$, but
if we apply a generic change of coordinates $\mA$ to $f$, then
$W(\pi_i,V)$ is known to be equidimensional of dimension $(i-1)$ or
empty~\cite{BaGiHeMb97}, and to be in so-called {\em Noether
  position}~\cite{EMP} (background notions in algebraic geometry are
in~\cite{Mumford76,Shafarevich77,ECA}; we will recall key
definitions). If this is the case, it suffices to choose arbitrary
$\sigma_1,\dots,\sigma_{n-1}$ in $\Q$, and solve the systems defined
by
\begin{equation}\label{eq:syst1}
  X_1-\sigma_1 = \dots = X_{i-1}-\sigma_{i-1} = f_1 = \cdots = f_p = M_{i,1} = \cdots = M_{i,S_i} = 0
\end{equation}
for $i=1,\dots,n$.  They all admit finitely many solutions, and
Theorem~2 in~\cite{EMP} proves that the union of their solution sets
contains one point on each connected component of $V \cap \R^n$.

Our main contribution is to analyze precisely what conditions on our
change of coordinates $\mA$ guarantee success. This is done by
revisiting the key ingredients in the proofs given
in~\cite{BaGiHeMb97} and~\cite{EMP}, and giving quantitative versions
of these results, bounding the degree of the hypersurfaces we have to
avoid.  To solve the equations~\eqref{eq:syst}, we use the algorithm
in~\cite{SH}, for which a complete bit complexity analysis is
available.

This work should be seen as a first step toward the analysis of
further randomized algorithms in real algebraic geometry. An immediate
follow-up question would be to handle the case of algebraic sets
defined by {\em regular sequences}: the algorithm in~\cite{EMP} still
applies, but the modifications needed are beyond the scope of this
publication. Further still, randomized algorithms for deciding {\em
  connectivity queries} on smooth, compact algebraic sets have been
developed in a series of papers
\cite{SchostMohabBabySteps2011,SchostMohabBabySteps2014}, and could be
revisited using the techniques introduced here.


\paragraph*{Further work.}
This work should be seen as a generalization of \cite{ElGiSh20}, where
the analysis was done for the hypersurface case. In addition, this
work should also be seen as a step toward the analysis of further
randomized algorithms in real algebraic geometry.  In particular,
randomized algorithms for deciding {\em connectivity queries} on
smooth, compact algebraic sets have been developed in a series of
papers \cite{SchostMohabBabySteps2011,SchostMohabBabySteps2014}, and
could be revisited using the techniques introduced here. The
techniques would apply to algorithms in real algebraic geometry where
transversality or Noether position are required geometric properties
established by a random change of coordinates.

\paragraph*{Outline.} \td

%%%%%%%%%%%%%%%%%%%%%%%%%%%%%%%%%%%%%%%%%%%%%%%%%%%%%%%%%%%%
%%%%%%%%%%%%%%%%%%%%%%%%%%%%%%%%%%%%%%%%%%%%%%%%%%%%%%%%%%%%
%%%%%%%%%%%%%%%%%%%%%%%%%%%%%%%%%%%%%%%%%%%%%%%%%%%%%%%%%%%%

\section{Preliminaries}

%% Let $\Xb = (X_1, \hdots , X_n)$ be a sequence of variables, and for $\ell
%% \in \{1,\hdots,n\}$ let $\Xb_{\leq \ell}$ be the subsequence of variables
%% $(X_1, \hdots , X_\ell)$.


In this section, \td

\paragraph*{Algebraic sets.} An algebraic set $V \subset \C^n$ is the set of common 
zeros of an ideal $I$ in $\C[X_1,\hdots,X_n].$ Conversely, the ideal
of a subset $V$ of $\C^n$, the set of polynomials in
$\C[X_1,\dots,X_n]$ that vanish at all points of $V$ is called the
\textit{ideal} of $V$; this is a radical ideal, which we write $I(V)$.

\paragraph*{Irreducible decomposition.}
An algebraic set $V \subset \C^n$ is \textit{irreducible} when $V =
V_1 \cup V_2$ implies $V=V_1$ or $V=V_2$, for any $V_1,V_2 \subset V$;
this is the case if and only if $I(V)$ is prime.  An algebraic set $V
\subset \C^n$ can be decomposed into a finite union of irreducible
algebraic sets
\[
V = V_1 \cup V_2 \cup \hdots V_r,
\]
with $V_i \not \subset V_j$ for all $i \ne j$. The sets where $V_j,
1\leq j \leq r$, are called the {\em irreducible components} of $V$;
they are uniquely defined, up to order. In terms of ideals, $I(V)$
being radical, it admits a decomposition an intersection of prime ideals
$I_1,\dots,I_r$; the irreducible algebraic sets $V(I_1),\dots,V(I_r)$ are the 
irreducible components of $V$.

\paragraph*{Dimension.}
The \textit{dimension} of an algebraic set $V \subset \C^n$, denoted
$\dim(V)$, can be defined as the unique integer $d$ such that $V \cap
H_1 \cap \cdots \cap H_d$ is finite, but not empty, for a generic
choice of hyperplanes $H_1,\dots,H_d$. The \textit{codimension} of $V$
is $n - \dim (V)$. 

An algebraic set $V$ is \textit{equidimensional} if each of its
irreducible components has the same dimension; if each component has
dimension $d$ then we say that $V$ is $d$-equidimensional.

\paragraph*{Degree.}
We use the definition of degree from~\cite{H}: the \textit{degree}
$\deg(V)$ of an irreducible algebraic set $V$ is the number of
intersection points between itself and $\dim (V)$ generic hyperplanes,
and the degree of an arbitrary algebraic set is defined as the sum of
the degrees of its irreducible components.

The degree of a hypersurface defined by a squarefree polynomial $f$ is
$\deg(f)$. We particularly care about algebraic sets of dimension zero; by
definition, these sets are finite and their degree is equal to their
cardinality.

We will often apply the B\'ezout’s bound from \cite[Theorem 1]{H},
which says that $\deg(V \cap V') \le \deg(V) \deg(V')$ holds for all
algebraic sets $V,V'$.

\paragraph*{Noether position.}
Suppose that the ambient dimension $n$ is fixed.  For $i$ in
$\{1,\dots,n\},$ let $\pi_i$ denote the projection
\begin{align*}
\C^n  &\rightarrow \C^i \\
(x_1,\hdots,x_n) &\mapsto  (x_1,\hdots,x_i).    
\end{align*} 
A $d$-equidimensional algebraic set $V \subset \C^n$ is in
\textit{Noether position} for the projection $\pi_d$ when the
extension \[\C[X_1,\hdots,X_{d}] \rightarrow \C[X_1,\hdots,X_n]/I(V)\]
is integral; here, $I(V) \subset \C[X_1,\hdots,X_n]$ is the ideal of
$V$. It is then a consequence that for any $\xb$ in $\C^d,$ the fiber
$V \cap \pi_d^{-1}(\xb)$ has dimension zero and is thus finite and not
empty.

\paragraph*{Tangent spaces, regular and singular points.}
Assume that $V \subset \C^n$ is a $d$-equidimen\-sional algebraic set;
let also $\grad_{\xb}(f)$ be the evaluation of the gradient vector of
$f \in \C[X_1,\hdots,X_n]$ at $\xb \in \C^n$. The
\textit{Zariski-tangent space} to $V$ at $\xb \in V$ is the vector
space $T_{\xb}V \subset \C^n$ defined by the equations
\[
\grad_{\xb} (g) \cdot \bm v = 0, \quad g \in I(V).
\] 
Then, the point $\xb \in V$ is a \textit{regular point} (or
non-singular) if $\dim (T_{\xb}V) = d$; otherwise, $\xb$ is a
\textit{singular point}. We let $\reg(V)$ and $\sing(V)$ respectively
denote the regular and singular points of $V$; when the latter is
empty, we say that $V$ is \textit{smooth}. If $I(V)$ is generated by
polynomials $G=(g_1,\hdots,g_s) \in \C[X_1,\hdots,X_n]^s$, then at any
point $\xb$ of $\reg(V)$, the Jacobian matrix $\jac_{\xb}(G)$ has full
rank $n - d$ and the kernel of $\jac_{\xb}(G)$ is $T_{\xb}V.$

\paragraph*{Changes of variables.}
For a matrix $\mA$ in $\C^{n\times n}$ and a polynomial $g$ in
$\C[X_1,\hdots,X_n],$ we write \[g^\mA:=g(\mA \Xb) \in
\C[X_1,\dots,X_n],\] where $\Xb$ is the column vector with entries
$X_1,\dots,X_n$. Similarly, for a sequence of polynomials
$G=(g_1,\hdots,g_s)$ in $\C[X_1,\hdots,X_n]^s$, we write $G^{\mA} =
\left(g_1^{\mA},\hdots,g_p^{\mA}\right).$ For an algebraic set $V
\subset \C^n$ and a matrix $\mA \in \GL(n),$ we define $V^{\mA}$ as
the image of $V$ by the map $\phi_{\mA} : \xb \mapsto \mA^{-1}\xb.$
 Notice in particular that
$V(G^{\mA}) = \phi_{\mA}(V(G)) = V(G)^{\mA}. $

\paragraph*{Locally open sets.}
We will also need to work with {\em locally open} sets: we say that $Y
\subset \C^n$ is locally open if we can write it as $Y=V-V'$, for some
algebraic sets $V,V'$. 

The notions of dimension and equidimensionality carry over to this
context (they are defined through the Zariski closure of $Y$), as does
that of tangent space: for $\xb$ in $Y$, we set $T_\xb Y = T_\xb V$
(this is independent of the choice of $V,V'$ in the definition above).
If $Y$ is equidimensional, as we did for algebraic sets, we can then
define the {\em regular points} (or non-singular points) of $Y$ as
those points at which the tangent space has dimension $d$, and we say
that $Y$ is smooth if all its points are regular.

Open sets are locally closed. As another example, for any
$d$-equidimensional algebraic set $V$, $\reg(V)$ is a smooth
$d$-equidimensional locally open set.

%% We will also have to consider matrices with generic entries. For this,
%% we introduce $n^2$ new indeterminates $(\frak A_{j,k})_{1\le j,k \le
%%   n}$. Then, $\A$ will denote the matrix with entries $(\frak
%% A_{j,k})_{1\le j,k \le n}$, $\C(\A)$ will denote the rational function
%% field $\C((\frak A_{j,k})_{1\le j,k \le n})$ and $\C[\A]$ the
%% polynomial ring $\C[(\frak A_{j,k})_{1\le j,k \le n}]$.  For $g$ as
%% above, we will then define the polynomial $g^\A:=g(\A \Xb)$, 
%% which we may consider in either
%% $\C(\A)[X_1,\dots,X_n]$ or $\C[\A,X_1,\dots,X_n]$.

%%%%%%%%%%%%%%%%%%%%%%%%%%%%%%%%%%%%%%%%%%%%%%%%%%%%%%%%%%%%
%%%%%%%%%%%%%%%%%%%%%%%%%%%%%%%%%%%%%%%%%%%%%%%%%%%%%%%%%%%%
%%%%%%%%%%%%%%%%%%%%%%%%%%%%%%%%%%%%%%%%%%%%%%%%%%%%%%%%%%%%

\section{Describing determinantal varieties}\label{ssec:detvar}

In this section, we work with polynomials in $\C[Y_1,\dots,Y_N]$.
Given a matrix $\bm A$ in $\C[Y_1,\dots,Y_N]^{q \times r}$, with $q
\le r$, together with some equations $b_1,\dots,b_s$ in
$\C[Y_1,\dots,Y_N]$, we consider the locus
\[X = \{\yb \in \C^N \ \mid \ b_1(\yb) = \cdots = b_s(\yb) =0 
     \text{~and~} {\rm rank}(\bm A(\yb)) < q\}.\]
\noindent
Our first goal here is to give a degree bound for $X$; this will be
used twice, in the next section for our discussion of the weak
transversality lemma (in a slightly general context where we work in
an open subset of $\C^N$), then also to control the degrees of polar
varieties.

A direct approach involves using the B\'ezout bound on the system of
polynomials mentioned above, $b_1,\dots,b_s$ and the minors of $\bm
A$, but even the refined form given in \cite[Proposition
  2.3]{Heintz1980} involves an exponential dependency in either the
ambient dimension $N$ or the number of minors $r \choose q$. This
might be acceptable in some contexts (such as when estimating the
degrees of polar varieties), but is way beyond our target bound in the
context of weak transversality, for instance.

Instead, we use Lagrange systems. We let $L_1,\hdots,L_q$ be new
variables, thought of as Lagrange multipliers, and consider the
``Lagrange polynomials'' given by the entries of $ [ L_1 ~\cdots~
  L_q]\cdot \bm A$. We denote by $Z \subset \C^{N+q}$ the algebraic set
defined by the vanishing of
\[\mathscr{I} = ( b_1,\dots,b_s, \  [ L_1 ~\cdots~ L_q]\cdot \bm A )\]
and by $Z'$ the algebraic set
\[
Z' := \overline{Z - \{(\yb,0,\dots,0) \in \C^{N+q}~|~(\yb,0,\dots,0) \in Z\}},
\]
where the bar denotes Zariski closure (we have to remove such points,
since $L_1=\cdots=L_q=0$ is always a trivial solution to the Lagrange
equations). Finally, consider the projection
\begin{align*} 
  \mu :~ \C^{N+q} &\rightarrow \C^{N}\\
  (\yb,\bm \ell)~ &\mapsto \yb.
\end{align*}
\begin{lemma}\label{lemma:Z_W}
  $X$ is the Zariski closure of ${\mu(Z')}$.
\end{lemma}
\begin{proof}
  Put $W := \overline{\mu(Z')}$, so that we want to prove the equality $X=W$.

  Let $C$ be an irreducible component of $Z'.$ There exists an open
  and dense subset $C' \subset C$ such that for all $(\yb,\bm \ell)$
  in $C'$, $\yb$ is in $\C^N$ and $\bm\ell$ is not identically
  zero. For such points, $\bm A(\yb)$ has rank less than $m$; since
  $b_1(\yb)=\cdots=b_s(\yb)=0$, $\yb$ is in $X$. In other words,
  $\mu(C')$ is contained in $X$.  Taking the union over all
  irreducible components $C$ of $Z'$, then the Zariski closure, yields
  $ \overline{\mu(\cup_{C} C')} \subset {X}$.
  Since $\overline{\cup_{C} C'} = Z'$,
  we get 
  \[
  W = \overline{\mu(Z')} = \overline{\mu(\overline{\cup_{C} C'})}
  = \overline{\mu(\cup_{C} C')} \subset {X},
  \]
  where the last inclusion follows from the previous paragraph. On the
  other hand, by construction, $X$ is contained in $\mu(Z')$, since
  for any $\yb$ in $X$, there exists $\bm \ell$ non-zero in $\C^q$
  such that $ \bm \ell \cdot \bm A(\yb) = \bm 0.$ Hence, we have
  \[
  X\subset\mu(Z')\subset \overline{\mu(Z')} = W \subset{X}.\qedhere
  \]
\end{proof}

\begin{corollary} \label{coro:degree}
  If all polynomials $b_1,\dots,b_s$ have degree at most $d$, and all
  entries of $\bm A$ have degree at most $d'$, then the degree of
  $\overline{X}$ is at most $d^s (d'+1)^r.$
\end{corollary}
\begin{proof}
  The algebraic set $Z$ is defined by $s$ equations of degree at most
  $d$ and $r$ equations of degre at most $d'+1$. It follows from
  B\'ezout's Theorem~\cite{H} that $\deg(Z) \leq d^s (d'+1)^r$, and
  the same upper bound holds for $\deg(Z')$, since it consists of
  certain irreducible components of $Z$. Degree does not increase
  through projection, so the conclusion follows from the previous
  lemma.
\end{proof}

While introducing $Z'$ is convenient, computing defining equations for
it is non-trivial, as it involves saturation; besides, in several
contexts, it will be advantageous to work with equations in complete
intersection; the following construction will guarantee it in certain
cases. For $\bm u = (u_1,\dots,u_q) \in \C^q$, consider the equations
\[\mathscr{I}_{\bm u} = ( b_1,\dots,b_s, \  [ L_1 ~\cdots~ L_q]\cdot \bm A,\ u_1 L_1 + \cdots + u_q L_q -1 ),\]
and let $Z_{\bm u} \subset \C^{N+q}$ be its zero-set.

\begin{prop}\label{prop:projection}
  For any $\bm u$ in $\C^q$, we have the inclusion $\mu(Z_{\bm u})
  \subset X$. There exists a non-empty open set $\mathscr{O} \subset
  \C^q$ such that for $\bm u$ in $\mathscr{O}$, we have the inclusion
  $X \subset \overline{\mu(Z_{\bm u})}$, and thus the equality $X =
  \overline{\mu(Z_{\bm u})}$.
\end{prop}
\begin{proof}
  If $(\bm y, \bm \ell)$ cancels all polynomials in $\mathscr{I}_{\bm
    u}$, then $\bm \ell$ is non-zero, so that $\bm A(\bm y)$ is
  rank-defective. As a consequence, $\bm y$ is in $X$.

  For the converse, let $X_1,\dots,X_K$ be the irreducible components
  of $X$.  For any given $k$ in $\{1,\dots,K\}$, since all $q$-minors
  of $\mA$ vanish on $X_k$, they vanish in the function field
  $\C(X_k)$, so $\mA$ has rank less than $q$ as a matrix over
  $\C(X_k)$. Thus, there exists a non-zero vector of rational
  functions
  \[\bm \ell_k = (\ell_{k,1},\hdots,\ell_{k,q})=\left(\frac{N_{k,1}}{D_k},\hdots,\frac{N_{k,q}}{D_k}\right)\in \C(X_k)^q,\]
  such that $\bm \ell_k  \cdot \bm A = 0$ in $\C(X_k)^r$. 
  For definiteness, assume that $N_{k,\iota_k} \ne 0.$ Then, in particular,
  $X'_k = X_k - V( D_k N_{k,\iota_k})$ is dense in $X_k$; for 
  $\bm y$ in $X'_k$, $\ell_k(\bm y)$ is well-defined and still
  satisfies $\bm \ell_k(\bm y) \cdot \bm A(\bm y) = 0$.
  
  Then, pick a point $\bm y_k$ in $X'_k$, so that
  $\bm \ell_k(\bm y_k)$ is a well-defined, non-zero vector in
  $\C^q$. This allows us to define a non-empty Zariski open set
  $\mathscr{O}_k \subset \C^q$ by the condition
  \[
  \mathscr{O}_k := 
  \left\{\bm u \in \C^q~|~ \bm u \cdot \bm \ell_k(\bm y_k) \ne 0\right \}.
  \]
  Finally, we let $\mathscr{O} := \cap_{1 \le k \le K}
  \mathscr{O}_{k}$, which is open and non-empty. We claim that for
  $\bm u$ in $\mathscr{O}$, the inclusion $X \subset
  \overline{\mu(Z_{\bm u})}$ holds. 

  For this, we take $k$ as above, 
  and we prove that $X_k$ is contained in $\overline{\mu(Z_{\bm u})}$.
  Consider the rational mapping 
  \begin{align*}
    X'_k  &\rightarrow \C\\    
    \bm y &\mapsto \ub \cdot \bm \ell_k(\bm y) = \frac{ u_1 N_{k,1}(\bm y) + \cdots + u_q N_{k,q}(\bm y)}{D_k(\bm y)}.    
  \end{align*}
  Put $X''_k = X'_k - V(u_1 N_{k,1} + \cdots + u_q N_{k,q})$; this is
  again an open subset of $X_k$, and the fact that $\bm u \cdot \bm
  \ell_k(\bm y_k)$ is non-zero, with $\bm y_k$ in $X'_k$, shows that
  $X''_k$ is not empty. In particular, it is dense in $X_k$.   

  Take $\bm y$ in $X_k$. Then, $\alpha:=\bm u \cdot \bm \ell(\bm y)$
  is non-zero, set we can define $\bm \ell' := 1/\alpha\ \ell(\bm y)$.
  Then, $\bm \ell'$ is still in the left nullspace of $\bm A(\bm y)$,
  and by construction $\bm y \cdot \bm \ell' =1$, so that $(\bm y, \bm
  \ell')$ is in $Z_{\bm u}$. In other words, $X''_k$ is contained in
  $\mu(Z_{\bm u})$. Taking the Zariski closure, we obtain that $X_k$ 
  is contained in $\overline{\mu(Z_{\bm u})}$, as claimed.
\end{proof}





\subsection{Extension to locally closed determinantal sets}

This is a locally closed set, since we can define it as the
intersection of $\Omega$ with the zero-set of $b_1,\dots,b_s$ and all
$q$-minors of $\bm A$.




%%%%%%%%%%%%%%%%%%%%%%%%%%%%%%%%%%%%%%%%%%%%%%%%%%%%%%%%%%%%
%%%%%%%%%%%%%%%%%%%%%%%%%%%%%%%%%%%%%%%%%%%%%%%%%%%%%%%%%%%%
%%%%%%%%%%%%%%%%%%%%%%%%%%%%%%%%%%%%%%%%%%%%%%%%%%%%%%%%%%%%

\section{Weak transversality}

Several of the generic properties of polar varieties are consequences
of {\em weak transversality}, which is an important extension of
Sard's lemma due to Thom (this observation goes back to work of
Giusti, Heintz and collaborators~\cite{BaGiHeMb97,BaGiHeLePa12}).  In
this section, we develop a quantitative extension of Thom's weak
transversality theorem, specialized to the particular case of
transversality to a point. In the sequel, we will apply this result to
bound the degree of particular hypersurfaces our algorithm needs to
avoid to guarantee success.

%%%%%%%%%%%%%%%%%%%%%%%%%%%%%%%%%%%%%%%%%%%%%%%%%%%%%%%%%%%%

\subsection{Definitions and statement of the result}

In its differential version, Sard's lemma states that the set of
critical values of a smooth function $\R^n \to \R^m$ has measure zero;
extensions exist to smooth mapping between differential manifolds.  In
our algebraic context, we will use the following definitions.

Consider a polynomial mapping $\Psi : Y \rightarrow \C^m$ from a
smooth $n$-equidimensional locally open set $Y$ to $\C^m$, with $m\le
n$. A {\em critical point} of $\Psi$ is a point $\bm y \in Y$ for
which the image of the tangent space $T_{\bm y} Y$ by the Jacobian
matrix $\jac_{\bm y}(\Psi)$ has dimension less than $m$. For instance,
the case that will interest us in this section is when $Y$ is Zariski
open in $\C^n$, in which case we have $T_{\bm y} Y=\C^n$ for all $\bm
y$ in $Y$, and the condition is equivalent to the Jacobian of $\Psi$
having rank less than $m$ at $\bm y$. {\em Critical values} are the
images by $\Psi$ of critical points; the complement of this set are
the {\em regular values}. Notice then, a regular value is not
necessarily in the image of~$\Psi$.

One can then give ``algebraic'' versions of Sard's lemma: for
instance,~\cite[(3.7)]{Mumford76} shows that for $Y$ an irreducible
algebraic set and $\Psi$ dominant, the critical values of $\Psi$ are
contained in a strict algebraic subset of $\C^m$; below, we will rely
on a straightforward generalization given in~\cite{TWT}. See
also~\cite[Chapter~9]{bochnak1998real} for the semi-algebraic case.

Thom's weak transversality lemma, as given for instance
in~\cite{demazure2000bifurcations}, generalizes Sard's lemma. In this
section, we consider a particular case of this result (transversality
to a point), and establish a quantitative version of it.

Let $n,\dt,$ and $m$ be positive integers, with $m \le n$ as before,
let $\mathscr{O}$ be a Zariski open subset of $\C^n$, and denote by
$\Phi: \mathscr{O} \times \C^{\dt} ~ \rightarrow \C^{m}$ a mapping
given by polynomials in $n+\dt$ indeterminates
$X_1,\dots,X_n,\Theta_1,\dots,\Theta_\dt$ (the latter should be
thought of as parameters). For $\thetab$ in $\C^{\dt}$, we let
$\Phi_{\thetab} : \mathscr{O} \rightarrow \C^{m}$ be the induced
mapping $\xb\mapsto\Phi(\xb,\thetab)$.  Thom's weak transversality
lemma tells us that if $0$ is a regular value of the mapping $\Phi$,
then $0$ remains a regular value of the induced mapping $\Phi_{\bm
  \vt}$ for a generic $\bm \vt$. (Here, we are dealing with the
particular case of transversality to a point, which can be rephrased
entirely in terms of regular and critical values.)  Our quantitative
version this result is the following.

\begin{prop} [Weak transversality]\label{prop:weak_t}
  Let all notation be as before, and suppose that $\Phi$ is defined by
  $m$ polynomials of degree at most $d$. If $0$ is a regular value of
  $\Phi$, there exists a non-zero polynomial $\Gamma \in
  \C[\Theta_1,\dots,\Theta_s]$ of degree at most $d^{m+n}$ such that
  for $\thetab$ in $\C^\dt$, if $\Gamma(\thetab)\ne 0$, then $0$ is a
  regular value of~$\Phi_{\thetab}$.
\end{prop}
\begin{ex}
  Consider a squarefree polynomial $f$ in $\C[X_1,X_2]$, with degree
  at most $d$, defining a smooth curve $V(f)$ in $\C^2$, and let the
  mapping $\Phi:\C^2\times \C \to \C^2$ be defined by
  $\Phi(X_1,X_2,\Theta) = (f(X_1,X_2), X_1-\Theta)$ (so $m=n=2$ and
  $s=1$). One checks that the Jacobian of $\Phi$ with respect to
  $(X_1,X_2,\Theta)$ has full rank two at any point in $\Phi^{-1}(0)$,
  so that $\bz$ is a regular value of $ \Phi$ and therefore the
  assumptions of the proposition apply.

  We then deduce that a non-zero polynomial $\Gamma \in \C[\Theta]$
  exists, with degree at most $d^{4}$ with the property that, if
  $\vartheta$ in $\C$ does not cancel $\Gamma$ then $0$ is a regular
  value of the induced mapping $ \Phi_{\bm \vt}$. That is, for all
  $\vartheta$ in $\C$ except at most $d^4$ values, the ideal
  $(f(X_1,X_2), X_1-\vartheta)$ is radical in $\C[X_1,X_2]$;
  equivalently, $f(\vartheta, X_2)$ is squarefree.
\end{ex}
We could of course obtain the same result (with a sharper degree
bound) by considering the discriminant of $f$ with respect to $X_2$,
but the construction above will be useful later on, in a generalized
form. Note that the degree bound in the proposition could be
strenghtened by considering \td

The rest of the subsection is devoted to the proof of the proposition.
The proof of \cite[Theorem B.3]{TWT} already shows the existence of
$\Gamma$; it is essentially the classical proof for smooth
mappings~\cite[Section~3.7]{demazure2000bifurcations}, written in an
algebraic context. In what follows, we revisit this proof,
establishing a bound on the degree of $\Gamma$.

%%%%%%%%%%%%%%%%%%%%%%%%%%%%%%%%%%%%%%%%%%%%%%%%%%%%%%%

\subsection{Proof of the proposition}

In the context of Thom's weak transversality, the bad parameters show
up as the critical values of a certain projection; the proof begins by
characterizing the critical points of this function. In what follows,
we use the notation of Proposition~\ref{prop:weak_t}, so that we
consider $m$ polynomials $\Phi$ that depend on variables
$X_1,\dots,X_n$ and $\Theta_1,\dots,\Theta_s$, with $m \le n$, and an
open set $\sO \subset \C^n$.

Put $Y = \Phi^{-1}(0) \cap (\sO \times \C^s)$, and let $V$ be the
Zariski closure of $Y.$ If $V$ is empty, there is nothing to do, since
all values $\thetab$ in $\C^\dt$ satisfy the conclusion of the
proposition. We therefore assume that $V$ is not empty. Take $(\xb,
\thetab)$ in $Y$; then by assumption, $\jac_{(\xb,\thetab)}({\Phi})$
has full rank $m$. Since in a neighborhood of $(\xb,\thetab)$, $V$
coincides with ${\Phi}^{-1}(0) \cap (\sO \times \C^s)$, the Jacobian
criterion~\cite[Corollary 16.20]{ECA} implies that there is a unique
irreducible component $V_{(\xb,\thetab)}$ of $V$ that contains
$(\xb,\thetab),$ that $(\xb,\thetab)$ is regular on this component and
that $\dim V_{(\xb,\thetab)}=n+s-m$. Since every irreducible component
of $V$ contains such a point $(\xb,\thetab)$, this implies that $Y$ is
a smooth, $(n+s-m)$-equidimensional locally open set.

%% Furthermore, it
%% also follows that for $(\xb,\thetab)$ in $Y$,
%% $T_{(\xb,\thetab)}Y=T_{(\xb,\thetab)}V$.
    
Now, consider the projection
%
\begin{align*}
  \pi: \C^{n+\dt} &\rightarrow\C^{\dt} \\
  (\xb, \thetab)&\mapsto\thetab, 
\end{align*}
%
and let $Z$ be the set of critical points of $\pi_{|Y}$; that is, \[Z
:= \{(\xb,\bm \thetab) \in Y~|~\dim (\pi(T_{\xb,\bm \thetab}Y))<s\}.\]
The projection $\pi(Z) \subset \C^s$ is thus the set of critical values
of $\pi_{|Y}$.

\begin{lemma}
  The Zariski closure $\overline{\pi(Z^{})}$ is a strict subset of
  $\C^s$.
\end{lemma}
\begin{proof}
  Let $Z^{'}$ be the critical points of $\pi_{|\reg(V)}$; by the
  algebraic form of Sard's lemma in \cite[Theorem~3.7]{Mumford76} (for
  irreducible $V$) and~\cite[Proposition~B.2]{TWT} (for general $V$),
  the Zariski closure $\overline{\pi(Z^{'})}$ is a strict closed
  subset of $\C^s$. Now, at any point $(\xb,\thetab)$ of $Y$, the
  tangent spaces $T_{(\xb,\thetab)} Y$ and $T_{(\xb,\thetab)} V$
  coincide. As a result, $Z$ is contained in $Z^{'}$, and the claim
  follows.
\end{proof}

\noindent 
We can now explain how $\thetab$ being a regular value of $\pi_{|Y}$
relates to $0$ being a regular value of $\Phi_{\thetab}$. In what
follows, we write our indeterminates as blocks of variables, with
$\Xb=X_1,\dots,X_n$ and $\Thetab = \Theta_1,\dots,\Theta_s$, and we
let $\bm J$ denote the $m \times n$ Jacobian matrix of $\Phi$ with
respect to $\Xb$ (so that for any $(\xb,\thetab)$, $\bm
J(\xb,\thetab)$ is then the Jacobian matrix of $\Phi_{\thetab}$ taken
at $\xb$).


\begin{lemma}\label{prop:rankJ}
  For $(\xb,\thetab)$ in $Y$, $(\xb,\thetab)$ is in $Z$ if and only if
  $\bm J(\xb,\thetab)$ has rank less than $m$.
\end{lemma}
\begin{proof}
  Let
  $\bm M$ denote the $(s+m) \times (s+n)$ Jacobian matrix of $\pi$
  and $\Phi$ with respect to $\Xb$ and $\Thetab$, 
  that is,
  \begin{align*}
    \bm M &= 
    \bbm 
    \jac_{\Xb,\Thetab}(\pi)\\
    \jac_{\Xb,\Thetab}(\Phi) 
    \ebm 
    =
    \bbm 
    \textbf{0}_{\dt \times n}\hspace{5mm}\textbf{I}_{\dt} \\
    \jac_{\Xb,\Thetab}(\Phi)
    \ebm.
  \end{align*}
  Take $(\xb,\thetab)$ on $Y$, and let $\bm K(\xb,\thetab)$ be the
  Jacobian matrix $\jac_{\Xb,\Thetab}(\Phi)$ taken at $(\xb,\thetab)$.
  Then, the rank of $\bm M(\xb,\thetab)$ can be written as
  $\textup{rank}(\bm K(\xb,\thetab)) + \textup{rank}([\textbf{0}_{\dt
      \times n}~\textbf{I}_{\dt}] \mid \ker \bm K(\xb,\thetab))$, where
  the latter is the rank of the restriction of $[\textbf{0}_{\dt
      \times n}~\textbf{I}_{\dt}]$ to the nullspace of
  $\bm K(\xb,\thetab)$.

  Since $(\xb,\thetab)$ is in $Y$ and since $0$ is a regular value of
  $\Phi$, $\bm K(\xb,\thetab)$ has full rank $m$. On the other hand,
  the nullspace of $\bm K(\xb,\thetab)$ is the tangent space
  $T_{\xb,\thetab} Y$, and $\textup{rank}([\textbf{0}_{\dt \times
      n}~\textbf{I}_{\dt}] \mid \ker \bm K(\xb,\thetab))$ is the
  dimension of $\pi(T_{\xb,\thetab} Y)$.  In other words, the rank of
  $\bm M(\xb,\thetab)$ is equal to $m+\dim(\pi(T_{\xb,\thetab} Y))$.

  This proves that for $(\xb,\thetab)$ in $Y$, $(\xb,\thetab)$ is in
  $Z$ if and only if the matrix $\bm M$ has rank less than $\dt+m$ at
  $(\xb,\thetab)$. Now, notice that
  %
  \begin{align*}
   \bm M(\xb,\thetab)&= 
    \bbm 
    \textbf{0}_{\dt \times n} &\textbf{I}_{\dt} \\
    \bm J(\xb,\thetab)     &\bm J'(\xb,\thetab)
    \ebm,
  \end{align*}
  where $\bm J'$ consists of the remaining columns of the Jacobian matrix of
  $\Phi$.  Then, the rank of the former matrix is equal to the rank
  of
  \begin{align*}
   \bm M(\xb,\thetab)&= 
    \bbm 
    \textbf{0}_{\dt \times n} &\textbf{I}_{\dt} \\
    \bm J(\xb,\thetab)     & \bm 0_{m \times s}
    \ebm.
  \end{align*}
  The lemma follows.
\end{proof}

As a result, suppose we take $\thetab$ in $\C^\dt - {\pi(Z)}$.  Then
for all $\xb$ in $\Phi_{\thetab}^{-1}(\bm 0) \cap \sO$,
$(\xb,\thetab)$ is in $Y$, so it is not in $Z$; the previous lemma
then implies that the Jacobian matrix of $\Phi_{\thetab}$ has full
rank $m$ at $\xb$. In other words, $0$ is a regular value of
$\Phi_{\thetab}$. To prove Proposition~\ref{prop:weak_t}, it is thus
enough to establish the existence of a non-zero polynomial of degree
at most $d^{m+n}$ that vanishes on $\overline{\pi(Z)}$. We already
established that $\overline{\pi(Z)}$ is a strict subset of $\C^\dt$,
so the only missing ingredient to prove that is has degree
at most $d^{m+n}$.

We start by bounding above the degree of $\overline{Z}$.
The previous lemma shows the equality
\[Z = \{ (\xb,\thetab) \in \Omega \times \C^\dt \ \mid \Phi(\xb,\thetab) =0
\text{~and~} {\rm rank}(\bm J(\xb,\thetab)) < m\}.\] Since all
polynomials in $\Phi$ have degree at most $d$, and all entries of $\bm
J$ at most $d-1$, we can apply Corollary~\ref{coro:degree}, so as to
deduce that $\deg(\overline{Z}) \le d^{m+n}$. This implies 
that $\overline{\pi(\overline Z)}$ has degree at most $d^{m+n}$,
and the equality 
$\overline{\pi(\overline Z)}
=\overline{\pi(Z)}$ allows us to conclude the proof.

%%%%%%%%%%%%%%%%%%%%%%%%%%%%%%%%%%%%%%%%%%%%%%%%%%%%%%%%%%%%
%%%%%%%%%%%%%%%%%%%%%%%%%%%%%%%%%%%%%%%%%%%%%%%%%%%%%%%%%%%%
%%%%%%%%%%%%%%%%%%%%%%%%%%%%%%%%%%%%%%%%%%%%%%%%%%%%%%%%%%%%

\section{Polar varieties}

Let $V=V(F)\subset \C^n$ be an equidimensional algebraic set, with $F
= (f_1,\hdots,f_p)$ a sequence of polynomials in
$\C[X_1,\hdots,X_n]$. Suppose that the ideal $\langle F \rangle
\subset \C[X_1,\hdots,X_n]$ is radical and that $V$ is smooth of
dimension $\delta= n-p$.

%%%%%%%%%%%%%%%%%%%%%%%%%%%%%%%%%%%%%%%%%%%%%%%%%%%%%%%%%%%%

\subsection{Definition and first properties}

Recall that, for $i \in \{1,\hdots,n\},$ we denote by $\pi_i$ the
projection
%
\begin{align*}
  \C^n &\rightarrow \C^i \\
  (x_1,\hdots,x_n) &\mapsto  (x_1,\hdots,x_i).    
\end{align*} 
For $i \le \delta$, the $i$-th \textit{polar variety} $\Wi$ of $V$ is
the set of critical points of the restriction of $\pi_i$ to $V$, that
is,
\[\Wi := \left\{\xb \in V~|~\dim \pi_i(T_\xb V) < i\right\}.\]
We naturally extend this definition to $i=\delta+1$, by setting
$W(\pi_{\delta+1},F)=V$.

For $1\le i \le \delta+1$, let $\bm J$, resp.\ $\bm J_i$, denote the
Jacobian matrix of $F=(f_1,\hdots,f_p)$ with respect to
$(X_{1},\hdots,X_n)$, resp.\ to $(X_{i+1},\hdots,X_n):$
\[
\bm J=
\left[ 
\begin{array}{ccc}
\frac{\pa f_1}{\pa X_{1}}&\hdots& \frac{\pa f_1}{\pa X_{n}} \\
\vdots& &\vdots\\
\frac{\pa f_p}{\pa X_{1}}&\hdots& \frac{\pa f_p}{\pa X_{n}} 
\end{array}
\right ],
\quad
\bm J_i=
\left[ 
\begin{array}{ccc}
\frac{\pa f_1}{\pa X_{i+1}}&\hdots& \frac{\pa f_1}{\pa X_{n}} \\
\vdots& &\vdots\\
\frac{\pa f_p}{\pa X_{i+1}}&\hdots& \frac{\pa f_p}{\pa X_{n}} 
\end{array}
\right]. 
\]
Since $F$ generates a radical ideal, for any $\xb$ in $V$, the tangent
space $T_\xb(V)$ is the kernel of $\bm J(\xb)$; the assumption that
$V$ be $\delta$-equidimensional and smooth implies that this kernel
has dimension $\delta=n-p$ at all such $\xb$.


Let then $S_i =\binom{n-i}{p}$ be the number of $p$-minors in $\bm
J_i$, and let $M_{i,1},\hdots,M_{i,S_i}$ be these minors (for
$i=\delta+1$, $S_{\delta+1}=0$ since $\bm J_{d+1}$ has size $p \times
(p-1)$). Then, our definitions imply that $\Wi$ is the zero-set of the
polynomials $\big(f_1,\hdots,f_p,M_{i,1},\hdots,M_{i,S_{i}}\big)$.

\begin{lemma}
  For $i=1,\dots,\delta+1$, the degree of $W(\pi_i,F)$ is at most
  $d^{n+p}$.
\end{lemma}
\begin{proof}
  For $i \le \delta$, this is a consequence of
  Proposition~\ref{coro:degree}, since all polynomials $f_1,\dots,f_p$
  have degree at most $d$, and all entries in $\bm J_i$ have degree at
  most $d-1$ (that proposition works in an open set, and consider a
  locally closed determinantal locus; here, the open set in question
  is simply $\C^n$, and our determinantal locus $W(\pi_i,F)$ is thus
  Zariski closed); the resulting degree bound is $d^{n+p-i} \le
  d^{n+p}$.  For $i=\delta+1$, $W(\pi_{\delta+1},F)$ has degree at
  most $d^p \le d^{n+p}$ as a consequence of B\'ezout's theorem.
\end{proof}

 

\begin{ex} (\cite[Example 3.1]{TWT})
  Let $f = X_1^2 + X_2^2 + X_3^2-1\in \C[X_1,X_2,X_3]$ and consider
  the hypersurface $V=V(f) \subset \mathbb{C}^{3}$. The critical
  points $W(\pi_2,F)$ of the projection $\pi_2: (x_1,x_2,x_3) \mapsto
  (x_1,x_2)$ on $V$ are defined by
  \[
  X_1^2 + X_2^2 + X_3^2-1=X_3=0,
  \]
  whereas the critical points $W(\pi_1,F)$ of the projection $\pi_1: (x_1,x_2,x_3)
  \mapsto x_1$ on $V$ are defined by
  \[
  X_1^2 + X_2^2 + X_3^2-1=X_2=X_3=0.
  \]
  In both cases, the dimension of $W(\pi_i,F)$ is $i-1$.
  \begin{figure}[h]
    %% \includegraphics[width=0.75\linewidth]{pi1Andpi2.png}
    \centering
    \caption{The polar varieties $W(1, F)$ and $W(2, F)$.}
\end{figure}
\end{ex}

%% The following simple inequality on the degree of polar varieties will 
%% be useful in the sequel.
%% \begin{prop}\label{prop:degreePolarV}
%%   Let $F=(f_1,\hdots,f_p) \in \C[X_1,\hdots,X_n]^p$ be a sequence of
%%   polynomials generating a radical ideal and defining a smooth
%%   $\delta$-equidimensional algebraic set $V$, with $\delta=n-p$. If
%%   $f_1,\dots,f_p$ have degree at most $d$, then for
%%   $i=1,\dots,\delta+1$, the degree of $\Wi$ is at most $(nd)^n$.
%% \end{prop}
%% \begin{proof}
%%   All polynomials defining $\Wi$ have degree at most $nd$; the
%%   conclusion then follows from \cite[Proposition 2.3]{Heintz1980}.
%% \end{proof}


The downside to defining polar varieties using minors of the truncated
Jacobian matrix is that these equations are in general not complete
intersection, due to the relations between minors of a matrix (the
hypersurface case is an exception, since in this case only partial
derivatives are used to define polar varieties). For both the
polynomial system algorithm we will use below, or an application we
will make of an effective Nullstellensatz, it will be necessary to
have equations in complete intersection. To make this possible, we use
the alternative modeling of polar varieties that uses Lagrange
variables from Section~\ref{ssec:detvar} (this already underlies the
proof of Proposition~\ref{coro:degree}, used in the lemma above).

We may thus consider the zero-set the polynomials
\begin{equation}\label{eqdef:sIif}
  \sI(i,F)
  = \big(F, [L_1~\cdots~L_p]\cdot \bm J_i\big ) \in \C[X_1,\dots,X_n,L_1,\dots,L_p]^{p+n-i},
\end{equation}
As in Section~\ref{ssec:detvar}, we will want to discard from the
zero-set of these equations in $\C^{n+p}$ those components where all
$L_i$'s vanish identically. However, we also want our equations to
form a complete intersection, which the saturation needed to remove
such components is unlikely to guarantee.

Instead, we will introduce a single additional equation, of the form
$u_1 L_1 + \cdots + u_p L_p -1$, for certain $(u_1,\dots,u_p)$ in
$\C$. Introducing this equation discards all solutions with $L_1 =
\cdots = L_p =0$, but other components of interests may be removed as
well; in the algorithm, we will use random $u_i$'s and quantify bad
choices. Thus, for $\ub = (u_1,\hdots,u_p) \in \C^p,$ we will define the 
following polynomials:
\begin{equation}\label{eqdef:Iil}
\Iil= 
\big (F,\ [L_1~\cdots~L_p]\cdot \bm J_i,\ u_1 L_1 + \cdots + u_p L_p -1 \big )
\in \C[X_1,\dots,X_n,L_1,\dots,L_p]^{p+n-i+1}.
\end{equation}
\begin{ex}
Consider again $f=X_1^2 + X_2^2 + X_3^2-1 \in \C[X_1,X_2,X_3]$ and
\[V(X_1^2 + X_2^2 + X_3^2-1)\subset \C^3.\]
Since $\jac(X_1^2 + X_2^2 + X_3^2-1,2)=2X_3,$ the Lagrangian modeling
gives us
\[
 V(X_1^2 + X_2^2 + X_3^2-1, LX_3, L-1) = V(X_1^2 + X_2^2-1,X_3,L-1),
\]
where the equations on the right hand side are a lexicographically
ordered Gr\"obner basis of the ideal $\la X_1^2 + X_2^2 + X_3^2-1,
LX_3, L-1 \ra.$ Notice that we have only 1 Lagrange multiplier, $L$,
and we have $\ub=(u)=(1)$, but any $u \not = 0$ would suffice here. We
therefore have that \[ \pi_{\Xb}(V(X_1^2 + X_2^2-1,X_3,L-1))\]
describes the set where
\[ X_1^2 + X_2^2 + X_3^2-1 = X_3 = 0\] and therefore describes the polar variety $W(\pi_2,F).$
\end{ex}

%%%%%%%%%%%%%%%%%%%%%%%%%%%%%%%%%%%%%%%%%%%%%%%%%%%%%%%%%%%%

\subsection{The main algorithm}

Let $F$ be as before. In this subsection, we briefly recall an
algorithm described in~\cite{EMP} that computes at least one point on
each connected component of $V(F) \cap \R^n$. 

To ensure correctness, we will need certain genericity assumptions,
which will be discussed in detail below. For the moment, we simply
give a high-level description of the algorithm. After applying a
randomly chosen change of variables $\mA$, we choose random
$\sigma_1,\dots,\sigma_{\delta}$ in $\C$. Then, for
$i=1,\dots,\delta+1$, we compute (in the new coordinates) the
points $\xb=(x_1,\dots,x_n)$ satisfying
\begin{equation}\label{eq:syst}
x_1 = \sigma_1,\hdots,x_{i-1} = \sigma_{i-1},\ F(\xb)=0,\ {\rm rank} (\bm J_i(\xb)) < p.
\end{equation}
In geometric terms, this means that we compute the intersection of
$W(\pi_i,F)$ with the fiber $\pi_i^{-1}(\sigma_1,\dots,\sigma_{i-1})$.
Then, we return the union of all these sets.

Departing from~\cite{EMP}, and following the discussion in the
previous subsection, we will avoid solving the system generated by
$F=f_1,\dots,f_p$ and the $p$-minors of $\bm J_i$: to control costs,
it will be beneficial to use Lagrange systems of~\eqref{eqdef:sIif}
and~\eqref{eqdef:Iil} instead. Hence, our genericity assumption will
concern these equations. Then, for $i=1,\dots,\delta+1$, we define the 
following properties:
\begin{description}
\item [$\bm H_i(1):$] $W(\pi_i,F)$ is either empty or $(i-1)$-equidimensional;
\item [$\bm H_i(2):$] $W(\pi_i,F)$ is either empty or in Noether position for
  $\pi_{i-1}$;
\item [$\bm H_i(3):$] $0$ is a regular value of the $n+p-i$ polynomials
  $F,\ [L_1~\cdots~L_p]\cdot \bm J_i$ in the open set defined by
  $(L_1,\dots,L_p) \ne (0,\dots,0)$.
\end{description}
As we will see, these properties hold after applying a generic change
of variables. The first two ensure that Eq.~\eqref{eq:syst} defines a
finite set (as a consequence of the definition of Noether position),
and guarantee that the output of the algorithm contains at least one
point on each connected component of $V(F) \cap \R^n$ (this is proved
in~\cite[Theorem~2]{EMP}). The last one will be used to establish that
assumption $\bm H^{'}_i$ defined below holds generically.

Indeed, assuming (possibly after applying a change of variables) that
$F$ satisfies $\bm H_i$, we define our second genericity property:
\begin{description}
\item [$\bm H^{'}_i:$] $\bm \sigma$ is such that $0$ is a regular value of the $n+p-1$
  polynomials
  \[ X_1 - \sigma_1, \dots, X_{i-1} - \sigma_{i-1},\ F,\ [L_1~\cdots~L_p]\cdot \bm J_i, \]
  in the open set defined by $(L_1,\dots,L_p) \ne (0,\dots,0)$.
%% \item For any $(\xb,\bm \ell) \in \C^{n+p}$ that cancels all $n+p-1$
%%   polynomials
%%   \[ X_1 - \sigma_1, \dots, X_{i-1} - \sigma_{i-1},\ F,\ [L_1~\cdots~L_p]\cdot \bm J_i, \]
%%   and with $\bm\ell$ non-zero, the Jacobian matrix of these
%%   polynomials has full rank $n+p-1$ at $(\xb,\bm \ell)$.
\end{description}
Again, we will see that this property holds for a generic choice of
$\bm\sigma$ and that as a consequence, $0$ is a regular value of the
$n+p$ polynomials
\begin{equation}\label{eq:syst2}
  X_1 - \sigma_1, \dots, X_{i-1} - \sigma_{i-1},\ F,\ [L_1~\cdots~L_p]\cdot \bm J_i,\ u_1 L_1 + \cdots + u_p L_p -1.
\end{equation}
In particular, these equations admit finitely many solutions.

Suppose that for some $i$ in $\{1,\dots,\delta+1\}$, if $F$ satisfies
$\bm H_i$ and $\bm \sigma$ satisfies $\bm H_i^{'}$; then, we know that
both systems~\eqref{eq:syst} and~\eqref{eq:syst2} have finitely many
solutions. In order to find the solutions of~\eqref{eq:syst}, we will
compute those of~\eqref{eq:syst2} and project them on the
$X_1,\dots,X_n$-space; we choose to solve equations~\eqref{eq:syst2},
since we can use the algorithm in~\cite{SH}, for which a complete bit
complexity analysis is available. To guarantee success, we will rely
on the last genericity property:
\begin{description}
\item [$\bm H^{''}_i:$] $\bm u$ is such that the projections of the
  solutions of~\eqref{eq:syst2} on the $X_1,\dots,X_n$-space are the
  solutions of~\eqref{eq:syst1}.
\end{description}
Applying Proposition~\ref{prop:projection} to the polynomials
in~\eqref{eq:syst2} shows that this property holds for a generic
choice of $\bm u$ (notice that since~\eqref{eq:syst2} has finitely
many solutions, taking the Zariski closure, as done in the
proposition, is not necessary in this case).




%% \noindent 
%% When $F$ satisfies $\bm H_i(3)$ then $\eqref{sys:minors}$ is finite. And when $F$ and $\sigma$ satisfy $\bm H_i^{'}$, then each solution $\xb$ of $\eqref{sys:minors}$ corresponds to an irreducible compoent $Z$ with $\xb \in Z$ and $\lb_Z(\xb)$ well defined and not zero. We in addition say that $\bm u$ satisfies $\bm H_i^{''}$ when, 
%% \[
%% \ub \cdot l_Z(\xb) \not = 0
%% \]
%% for each solution $\xb$ of \eqref{sys:minors}.
%% \noindent 
%% We prove the following.  
%% %
%% \begin{theorem}\label{theo:NoetherPositionG}
%%   For $i=1,\dots,n-p+1$, there exists a non-zero polynomial $\D_i\in\C[\A]$ of degree at most $6n^2(2d)^{5n}$ such that if $\mA \in
%%   \C^{n\times n}$ does not cancel $\D_i$, then
%%   $F^\mA$ satisfies $\textbf{H}_i$.
%% \end{theorem}
%% %
%% \begin{theorem}\label{theo:sigmaG}
%%   For $i=1,\dots,n-p+1$, suppose that $F$ satisfies $\textbf{H}_i$, then there exists a non-zero
%%   polynomial $\Xi_{i} \in \C[S_1,\dots,S_{i-1}]$ of degree at most
%%   $3n(nd)^{3n}$ such that if $\bm \sigma \in \C^{i-1}$ does not
%%   cancel $\Xi_{i}$, then $F$ and $\bm \sigma$ satisfy $\textbf{H}_i'$.
%% \end{theorem}
%% %
%% \begin{theorem}\label{the:u}
%%   For $i=1,\dots,n-p+1$, suppose that $F$ satisfies $\textbf{H}_i$ and suppose that $F$ and $\bm \sigma$ satisfy $\textbf{H}_i^{'},$ there exists a non-zero
%%   polynomial $\Upsilon_{i} \in \C[T_1,\dots,T_{p}]$ of degree at most $(nd)^{n}$ such that if $\bm u \in \C^{p}$ does not
%%   cancel $\Upsilon_{i}$, then $\bm u$ satisfies $\textbf{H}_i{''}$.
%% \end{theorem}
%% %

%%%%%%%%%%%%%%%%%%%%%%%%%%%%%%%%%%%%%%%%%%%%%%%%%%%%%%%%%%%%
%%%%%%%%%%%%%%%%%%%%%%%%%%%%%%%%%%%%%%%%%%%%%%%%%%%%%%%%%%%%
%%%%%%%%%%%%%%%%%%%%%%%%%%%%%%%%%%%%%%%%%%%%%%%%%%%%%%%%%%%%

\section{Genericity of ${\bf H}_i(1)$ and ${\bf H}_i(3)$}
\label{sec:applications}

Let $F=(f_1,\hdots,f_p) \in \ZZ[X_1,\hdots,X_n]^p$ be a sequence of
polynomials defining a radical ideal, and where the degree of each
polynomial is at most $d$. Also, assume that the variety $V(F) \subset
\C^n$ is smooth. Recall that $\A$ denotes the matrix of indeterminates
with entries $(\frak A_{j,k})_{1\le j,k \le n}$.


For $=1,\dots,\delta+1$, with $\delta = n-p$, let $\Xb=X_1,\dots,X_n$
and let $\bm K_i(\Xb,\A)$ denote the $(p+i)\times n$ matrix
\[
\bm K_i(\Xb,\A)=
\bbm 
\jac(F)\\
\A_{1,1}~~ \hdots ~~\A_{1,n}\\
\vdots\hspace{10mm}\vdots\\
\A_{i,1}~~ \hdots ~~\A_{i,n}
\ebm.
\]
Consider elements $\bm a \in \C^{in}$ as vectors of length $i$ of the
form $\bm a = (\bm a_1,\hdots,\bm a_i)$ with $\bm a_i \in \C^n$;
we say that $\bm a$ has rank $i$
when $\bm a$ is a sequence of linearly independent vectors.  Then
for such an $\bm a$, $\bm K_i(\Xb,\bm a)$ is naturally defined with the
indeterminates evaluated at $\bm a$. 

Writing $\bm \lambda =(\lambda_1,\dots,\lambda_p)$ and 
$\bm \vartheta =(\vartheta_1,\dots,\vartheta_i)$, 
 let $\Phi$ define the polynomial mapping
\begin{align*}
  \C^{n+p+i}\times \C^{i n} &\rightarrow \C^{n+p}\\
  (\xb,\bm \lambda, \thetab, \bm a)&
  \mapsto
  \left(F(\xb),[\lambda_1 ~\cdots~ \lambda_p~ \vartheta_1 ~\cdots~ \vartheta_i ] \cdot \bm K_i(\xb,\bm a)
  \right)
\end{align*}
and for $\bm a$ in $\C^{ni}$, let $\Phi_{\bm a}$ the induced mapping   
\begin{align*}
  \C^{n+p+i} &\rightarrow \C^{n+p}\\
  (\xb,\bm \lambda,\bm \thetab)&\mapsto \Phi(\xb,\bm \lambda,\bm \thetab,\bm a).
\end{align*}
Let $\sA \subset \C^{n+p+i}$ be the open set defined by the condition
$\bm \lambda=(\lambda_1,\hdots,\lambda_p) \not = 0$. In \cite[Section
  3.2]{BaGiHeSaSh10}, it is shown that, for any $(\xb, \bm \lambda,\bm
\thetab, \bm a)$ in $\sA \times \C^{in}$, the Jacobian matrix
$\jac(\Phi)$ has full rank $p + n$ at $(\xb, \bm \lambda, \bm \thetab,
\bm a)$. In particular, this is true for $(\xb, \bm \lambda, \bm
\thetab, \bm a)$ in $\Phi^{-1}(0)$, so that $0$ is a regular value of
$\Phi$ on $\sA \times \C^{in}$. 

It therefore follows by Proposition \ref{prop:weak_t} that there
exists a non-zero polynomial $\Gamma_i \in
\C[\A_{1,1},\hdots,\A_{i,n}]$ of degree at most
\[
d^{(n+p+i)+(n+p)} \le d^{5n},
\]
such that if $\ab \in \C^{i \times n}$ does not cancel $\Gamma_i$,
then $0$ is a regular value of $\Phi_{\bm a}$ on $\sA$. That is, for
$(\xb, \bm \lambda,\bm \thetab) \in \sA \cap \Phi^{-1}_{\bm a}(0)$,
the Jacobian matrix $\jac(\Phi_{\bm a})$ has full rank $n+p$
at ${(\xb, \bm \lambda,\bm \thetab)}$.
    

Let $\fB=\A^{-1}$ in $\C(\A)^{n \times n}$ and let
$\fB_1=[\fB_{1,1},\hdots,\fB_{1,n}],\hdots,\fB_n=[\fB_{n,1},\hdots,\fB_{n,n}]$
denote the rows of $\fB.$ Set
\[
\Delta_{i,1} := \Gamma_i(\fB_1,\hdots,\fB_i)\cdot (\det( \A))^{\deg (\gi)+1}. 
\]
By multiplying through by $(\det( \A))^{\deg( \gi)+1},$ we cancel all
denominators and make $\D_{i,1}$ a polynomial divible by $\det(\A)$.
\begin{lemma}
  The degree of $\Delta_{i,1}$ is at most $n(d^{5n}+1).$
\end{lemma}
\begin{proof}
  Assume that 
  \[
  \fB_{i,j}=\fN_{i,j}/{\rm det}(\A),~ \fN_{i,j},\det(\A)\in \C[\A] \text{\todo{notation $\A$ used for two things}}.
  \]
  Then, by Cramer's formulas, we have $\deg( \fN_{i,j}),\deg(\det(\A))
  \leq n,$ and since we have cleared all denominators by multiplying
  through with $(\det( \A))^{\deg (\gi)+1},$ and guaranteed the
  presence of an extra factor $\det(\A)$, we therefore obtain
  \[
  \deg( \Delta_{i,1}) \leq n\deg( \gi) + n \leq n(d^{5n}+1). \qedhere
  \]
\end{proof}
The main ingredients in the proof of the following proposition are
taken from~\cite{TWT}, with no modification; they itself follow
previous work such as~\cite{BaGiHeSaSh10}.
\begin{prop}
  For $\mA \in \C^{n\times n}$, if $\mA$ does not cancel $\D_{i,1}$,
  then $\mA$ is invertible and the polar variety $W(\pi_i,F^\mA)$ is
  either empty or $(i-1)$-equidimensional.
\end{prop}
\begin{proof}
  Consider $\bm A \in \C^{n \times n}$ that does not cancels
  $\D_{i,1}$.  Since $\det(\A)$ divides $\D_{i,1}$, $\bm A$ is
  invertible, and by construction the first $i$ rows $\bm b$ of $\bm
  A^{-1}$ do not cancel $\gi$. We put
  \[
  Y := \left\{\xb \in V(F)~|~\rk (\bm K_i(\Xb,\bm b)) < p+i\right\}. \quad \text{\todo{notation for $\bm K$}}
  \]
  Lemma B.5 from~\cite{TWT} shows that all irreducible components of
  $Y$ have dimension at least $i-1$; our assumption on $\bm b$ allows
  us to apply Lemma B.11 from that reference, which shows that all
  irreducible components of $\dim Y$ have dimension at most $i-1$.
  Therefore, $Y$ is either empty or ($i-1)$-equidimensional. To
  conclude the proof, we use the equality
  \[
  Y^{\mA} = W\left(\pi_i,V\left(F^{\mA}\right)\right),
  \]
  established in the same reference immediately before Lemma~B.10.
  %% Indeed, consider the identity $\jac(F^{\mA}) = \jac(F)^{\mA}\mA$ and notice that
  %% \[
  %% \bm K_i(\mA\xb,\bm b)
  %% =
  %% \bbm 
  %% \jac_{\xb}(F)^\mA\\
  %% \bm b
  %% \ebm
  %% =
  %% \bbm 
  %%   \jac_{\xb}(F^\mA)\mA^{-1}\\
  %%   \bbm
  %%   \bm 1_i ~~~~~ \bm 0
  %%   \ebm\mA^{-1} 
  %%    \ebm
  %%    =
  %%    \bbm 
  %%   \jac_{\xb}(F^\mA)\\
  %%   \jac_{\xb}(\pi_i)
  %%    \ebm
  %%   \mA^{-1}
  %%   \]
  %%   and therefore 
  %%   \[
  %%   \rk (\bm K_i(\mA\xb,\bm b))
  %%   =
  %%   \rk
  %%    \bbm 
  %%   \jac_{\xb}(F^\mA)\\
  %%   \jac_{\xb}(\pi_i)
  %%    \ebm.
  %%   \]
  %%   \todo{not clear}
  %%   Furthermore, since $V(F)$ is smooth, it follows from \cite[Corollary
  %%     16.20]{ECA} that for all $\xb$ in $V(F), \jac_{\xb}(F)$ has full
  %%   rank $n - \dim V = n -(n-p) = p$. Therefore
  %%   \begin{align*}
  %%    Y^{\mA} &= \left\{\xb \in V(F^{\mA})~
  %%   \big|~\rk \jac_{\xb}(F^{\mA})=p
  %%   \textrm{ and }
  %%   \rk     
  %%   \bbm 
  %%   \jac_{\xb}(^{\mA})\\
  %%   \jac_{\xb}(\pi_i)
  %%    \ebm
  %%    < p+i\right\} \\
  %%    &= W\left(\pi_i,V\left(F^\mA\right)\right). 
  %%   \end{align*}
\end{proof}


\begin{prop}
  For $\mA \in \C^{n\times n}$, if $\mA$ does not cancel $\D_{i,1}$,
  then $0$ is a regular value of the $n+p-i$ polynomials
  $F,\ [L_1~\cdots~L_p]\cdot \bm J_i$ in the open set defined by
  $(L_1,\dots,L_p) \ne (0,\dots,0)$.  \td{notation $L$ or $\lambda$}
\end{prop}
\begin{proof}
  Take $\mA$ in $\C^{n \times n}$ so that $\Delta_{i,1}(\mA) \not = 0$,
  and again let $\bm b$ denote the first $i$ rows of $\mA^{-1}.$ Let
  $T_1,\hdots,T_i$ be new indeterminates. By assumption, for $(\xb, \bm
  \lambda, \bm \thetab) \in \sA \cap \Phi^{-1}_{\bm b}(\bz)$, the
  Jacobian matrix of the polynomials
    \[
    \left(F,\ [L_1 ~\cdots~ L_p~ T_1 ~\cdots~ T_i ] \cdot 
   \bm K_i(\Xb,\bm b)
    \right)
    \]
    has full rank $p + n$. The Jacobian matrix of the polynomials
    \[
    \left(F(\mA\bm X),\ [L_1 ~\cdots~ L_p~ T_1 ~\cdots~ T_i ] \cdot \bm K_i(\mA \Xb,[\bm I_i\ \bm 0])
    \right)
    \]    
    taken with respect to the variables
    $X_1,\dots,X_n,L_1,\hdots,L_p,T_1,\hdots,T_i$ is equal to 
    \begin{align*}
    \left[ 
    \begin{array}{cc}
    \jac_{\bm X}(F^{\mA})~~~~~ \bz_{p \times p} & \bz_{p\times i}\\
    \jac_{\bm X,\bm L}\left([\bm L, \bm T ] \cdot 
    \bbm 
    \jac_{\bm X}(F^{\mA})\\
    \jac_{\bm X}(\pi_i)\\
    \ebm\right) & \bmat \bm I_{i}\\ \bz_{p\times i} \emat\\
    \end{array}
    \right]
    &=
    \left[ 
    \begin{array}{c|c}
    \jac_{(\xb,\lb)}(F^{\mA}) & \bz_{p\times i} \\
    \ast \ast \ast & I_{i}\\
    \jac_{(\xb,\lb)}\left(\bm l \cdot \jac_{\xb}(F^{\mA},i) \right)& \bz_{p\times i}
    \end{array}
    \right].
    \end{align*}
    %



Now, from the chain rule $\jac(F^\mA)=\jac(F)^\mA \mA$,
we get
    \[
    \bm K_i(\mA\xb,\bm b)
    =
    \bbm 
    \jac(F)^\mA\\
    \bm b
     \ebm
     =
    \bbm 
    \jac(F^\mA)\mA^{-1}\\
    \bbm
    \bm I_i ~~~~~ \bm 0
    \ebm\mA^{-1} 
     \ebm
     =
     \bbm 
    \jac(F^\mA)\\
    \jac(\pi_i)
     \ebm
    \mA^{-1}
    \]
    and therefore, for any $\xb$,
    \[
    \rk( \bm K_i(\mA\xb,\bm b))
    =
    \rk\left (
     \bbm 
    \jac(F^\mA)\\
    \jac(\pi_i)
     \ebm (\xb) \right ).
    \] 






    Therefore, after rearranging blocks and after removing $i$ columns, we can see that 
    %
    \begin{align}
    \jac_{(\xb, \lb)}\left( F^{\mA},\bm l \cdot \jac_{\xb}(F^{\mA},i)\right) =
    \left[ 
    \begin{array}{c}
    \jac_{(\xb,\lb)}(F^{\mA}) \\
    \jac_{(\xb,\lb)}\left(\bm l \cdot \jac_{\xb}(F^{\mA},i) \right) 
    \end{array}
    \right]
    \end{align}
    has full rank $p + n-i$. Now, recall that
    \[
    \bm l \cdot \jac_{\xb}(F^{\mA},i) = \lagFA,
    \]
    and therefore it becomes clear that, at any 
    \[(\xb,\lb) \in V(\IilAnu)=\WilAnu,\]
    the Jacobian of the polynomials 
    \[
    \left(F^{\mA},\lagFA\right)
    \]
    has full rank $n+p-i$. 
    \end{proof}
    


    \begin{prop}\label{prop:RadLagPolarV}
     Let $\ub = (u_1,\hdots,u_p) \in \C^p$ be any complex point. Then,
     for any \[(\xb,\lb)\in
     \mathscr{W}_{\ub}\left(\pi_i,V(F^{\mA})\right) \subset
     \C^{n+p},\] the Jacobian matrix of the polynomials
     \[
    \left( F^{\mA},\lagFA,\udl-1\right) 
    \]
 has full rank $n+p-i+1$ at $(\xb,\lb).$ 
    \end{prop}
    \begin{proof}
    Note that $(\xb,\lb)\in\mathscr{W}_{\ub}(\pi_i,V(F^{\mA}))$
    implies that $(\xb,\lb)\in\mathscr{W}(\pi_i,V(F^{\mA}))$; thus, by
    Proposition \ref{prop:LagIdeal}, the Jacobian of the
    polynomials \[\sI(i,F^{\mA}) = (F^{\mA},\lagFA)\] has full rank
    $n+p-i$ at $(\xb,\lb)$. The conclusion therefore holds if
    $[\bz~|~\ub]$ is not in the row space of the Jacobian of the
    polynomials
    \[
    \left(F^{\mA},\lagFA,\udl -1\right),
    \]
    for any $(\xb,\bm l)$ that cancels equations. This matrix is equal to
    \[
    \left[ 
    \begin{array}{c|c|c}
    \frac{\pa f_1^{\mA}}{\pa X_1}(\xb) \hdots \frac{\pa f_1^{\mA}}{\pa X_n}(\xb) & \bz_{1\times p} & \bz_{1\times p}\\
    \ddots & \ddots & \ddots\\
    \frac{\pa f_p^{\mA}}{\pa X_1}(\xb) \hdots \frac{\pa f_p^{\mA}}{\pa X_n}(\xb) & \bz_{1\times p} & \bz_{1\times p}\\
    \ddots &\jac_{\xb}(F^{\mA},i)^T & \bz_{n-i+1\times p}\\
    \bz_{1 \times p} & u_1 \hdots u_p & l_1 \hdots l_p 
    \end{array}
    \right];
    \]
    consider the upper left block
    \[
    \left[ 
    \begin{array}{c|c}
    A & \bz_{p\times p} \\
    B & C 
    \end{array}
    \right]
    :=
    \left[ 
    \begin{array}{c|c}
    \frac{\pa f_1^{\mA}}{\pa X_1}(\xb) \hdots \frac{\pa f_1^{\mA}}{\pa X_n}(\xb) & \bz_{1\times p} \\
    \ddots & \ddots \\
    \frac{\pa f_p^{\mA}}{\pa X_1}(\xb) \hdots \frac{\pa f_p^{\mA}}{\pa X_n}(\xb) & \bz_{1\times p} \\
    \ddots &\jac_{\xb}(F^{\mA},i)^T
    \end{array}
    \right],
    \]
    and suppose for contradiction that $[\bz~|~\ub]$ is in the row-space. Then 
    \[
    [\bz~|~\ub]
    =
    \lambda [A~|\bz] + \mu[B~|~C]
    \]
    and 
    \[
    \ub = \mu \cdot C = \mu \cdot \jac_{\xb} (F^{\mA},i)^T
    \]
    so that 
    \[
    \ub^T = \jac_{\xb} (F^{\mA},i) \cdot \mu^T.
    \]
    Now we have a contradiction because, $(\xb,\lb)$ is such that 
    \begin{align*}
        &\lb \cdot \jac_{\xb} (F^{\mA},i)=0\\
        \Rightarrow
        ~&\lb \cdot \jac_{\xb} (F^{\mA},i) \mu^T= 0\\
        \Rightarrow
        ~&~\lb \cdot \ub^T = 0,
    \end{align*}
    when by assumption $\lb \cdot \ub^T = 1.$ 
    \end{proof}
    %
    %
    \begin{corollary}\label{cor:LagIdealRadical}
Let $\ub = (u_1,\hdots,u_p) \in \C^p$ be any complex point. Then, for any \[(\xb,\lb)\in \mathscr{W}_{\ub}(\pi_i,V(F^{\mA})) \subset \C^{n+p}\] the ideal defined by $\sI_{\ub}(i,F^{\mA}) :$\[ \left(F^{\mA},\lagFA,\sum_{i=1}^p u_iL_i-1\right)\] is radical.
    \end{corollary}
    \begin{proof}
    Given Proposition \ref{prop:RadLagPolarV}, the conclusion now follows from the Jacobian Criterion~\cite[Corollary 16.20]{ECA}.
    \end{proof}
    %
    %
    \begin{prop}\label{prop:polarVs2}
Let $\ub = (u_1,\hdots,u_p) \in \C^p$ be any complex point. Then we have the inclusion
    \[
    \sqrt{\frak{I}(i,F^{\mA})} \subset \IilA.
    \]
    \end{prop}
    \begin{proof}
    First note that $\IiA \subset \sqrt{\IilA}.$ Indeed, let $f \in \IiA$ and \[\bm \alpha = (\xb,\lb) \in V(\IilA)=\WilA.\] Then $\sum_{i=1}^p u_i l_i = 1$, so that $\lb \not = (0,\hdots,0)$ is in the left null space and therefore the rank of $\jac_{\xb} (F,i)$ is less than $p.$ Therefore all minors are zero at $\xb$ and $f(\bm \alpha)= f(\xb,\lb)=f(\xb)=0,$ so that $f \in \sqrt{\IilA}$ and $\IiA \subset \sqrt{\IilA}.$
    \par 
    Now, since by Corollary \ref{cor:LagIdealRadical}, $\sqrt{\IilA} = \IilA$, we have the inclusion
    \[
    \sqrt{\frak{I}(i,F^{\mA})} \subset \sqrt{\IilA} = \IilA.
    \]
    \end{proof}
    %
    %

















%%%%%%%%%%%%%%%%%%%%%%%%%%%%%%%%%%%%%%%%%%%%%%%%%%%%%%%%%%%%
%%%%%%%%%%%%%%%%%%%%%%%%%%%%%%%%%%%%%%%%%%%%%%%%%%%%%%%%%%%%
%%%%%%%%%%%%%%%%%%%%%%%%%%%%%%%%%%%%%%%%%%%%%%%%%%%%%%%%%%%%
\section{Proof of the Noether position property $\textbf{H}_i(3)$}\label{ssec:Hi2}


Consider a sequence of polynomials $F= (f_1,\hdots,f_p) \in \ZZ[X_1,\hdots,X_n]^p,$ with degrees at most $d$, defining a radical ideal $\langle F \rangle \subset \C[X_1,\hdots,X_n]$ and a smooth variety $V=V(F) \subset \C^n$ with $\dim V = n-p$. We prove that there exists a non-zero polynomial $\D_{i}$ in $\C[\A]$ of degree at most $6n^2(2d)^{5n}$ such that if $\mA$ does not cancel $\D_{i}$, then $F^{\mA}$ satisfies  $\bm H_i$.

Recall that we let $\Xb = (X_1, \hdots , X_n)$ be a sequence of variables, and for $l \in \{1,\hdots,n\}$ we let $\Xb_{\leq l}$ be the subsequence of variables $(X_1, \hdots , X_l)$. Consider again the $n\times n$ matrix of indeterminates
\[\A=(\A_{j,k})_{1 \le j,k \le n}\] and the field $\C(\A)$,
and define $F^{\A}=(f_1^{\A},\hdots,f_p^{\A})$  as \[(f_1(\A\Xb),\hdots,f_p(\A\Xb)) \in \C(\A)[\Xb]^p.\]  
%Since $i$ is fixed, to
%simplify notation, let $\I^\A$ denote the polynomials $\I(i,F^\A)= \big (F^\A, \minorsfA \big )$ in
%$\C(\A)[\Xb]$, and let $\Il^\A$ denote the polynomials $\IilfA= \big (F^\A, \bm L \cdot \jac_{\xb}(F^\A,i),\ub^T\cdot\Lb-1 \big )$ in
%$\C(\A)[\Xb,\Lb]$. Let $W^\A$ denote  $W(\pi_i,V(F^\A))$ and let $\Wl^\A$ denote  $\WilfA$. 



Some related results appear in the literature. For instance, Lemma 5
in~\cite{JeSa10} or Proposition 4.5
in~\cite{SharpEstimatesForTheEffectiveN} are quantitative Noether
position statements. However, Theorems~\ref{theo:H} and
\ref{theo:NoetherPositionG} do not follow from these previous
results. Indeed, those references would allow us to quantify when
$W(\pi_i,V(F))^\mA$ is in Noether position, whereas we need to
understand when $W(\pi_i,V(F^\mA))$ is. And these two sets are in
general different; for instance, their dimensions may vary.

%%%%%%%%%%%%%%%%%%%%%%%%%%%%%%%%%%%%%%%%%%%%%%%%%%%%%%%%%%%%
\subsubsection{Degree bounds for the integral dependence relationship} 
%
In Section~\ref{sec:applications}, we saw that
$F^\A$ satisfies $\bm H_i(1)$, so that $\WifA$ is equidimensional of dimension $i-1$. We
now point out that $F^\A$ also satisfies $\bm H_i(3)$.
\begin{lemma}\label{lem:6.1}
 The extension $$\C(\A)[\Xb_{\leq i-1}]\rightarrow
 \C(\A)[\Xb]/\IifAr$$ is integral.
\end{lemma}
%
\begin{proof}
  Let $(\fp_\ell)_{1 \le \ell \leq L}$ be the prime components of $\IifAr$. By \cite[Proposition 1]{EMP}, for all
  $\ell$,
  \[
    \C(\A)[\textbf{X}_{\leq i-1}]\rightarrow\C(\A)[\Xb]/\fp_\ell
  \] 
  is integral. Therefore polynomials
  $q_{\ell,j}\in\C(\A)[\textbf{X}_{\leq i-1},X_j]$ exist, all monic in
  $X_j$, with $q_{\ell,j}(X_j)\in \fp_\ell$ for each $j$ in
  $\{i,\hdots,n-p+1\}.$ Thence, \[Q_{j} := \prod_{1 \le \ell\le L}
  q_{\ell,j}\] is monic in $X_j$ and satisfies $ Q_{j} \in \sqrt{\IifA}$, for
  each $j \in \{i,\hdots,n-p+1\}.$ This proves our claim.
\end{proof}
%
\noindent 
Now let $\ub \in \C^p$ be any complex number. 
\begin{corollary}\label{lem:6.2}
 The extension $$\C(\A)[\Xb_{\leq i-1}]\rightarrow
 \C(\A)[\Xb]/(\IilfA \cap \C(\A)[\Xb])$$ is integral.
\end{corollary}
%
\begin{proof}
By Lemma \ref{lem:6.1}, polynomials
  $P_{j}\in\C(\A)[\Xb_{\leq i-1},X_j]$ exist, all monic in
  $X_j$, with $P_{j}(X_j)\in \IifAr$ for each $j$ in
  $\{i,\hdots,n-p+1\}.$ By Proposition \ref{prop:polarVs2}, 
  \[
  \sqrt{\IifA} \subset \IilfA,
  \]
  and therefore $P_{j}(X_j)\in \IilfA$ for each $j$ in 
  $\{i,\hdots,n-p+1\}$ and 
  \[
  \C(\A)[\Xb_{\leq i-1}]\rightarrow
 \C(\A)[\Xb]/(\IilfA \cap \C(\A)[\Xb])
  \]
 is integral.
\end{proof}
%
\noindent
If $P$ is any polynomial in $\C(\frak A)[\Xb]$, we will let
$D \in \C[\frak A]$ be the minimal common denominator of all its
coefficients, and we will write $\overline P := D P$, so that
$\overline P$ is in $\C[\A,\Xb]$.


    \begin{lemma} 
    For each $j \in \{i,\dots,n-p+1\}$, there exists $P_j$ in $
    \C(\A)[\Xb_{\leq i-1},X_j]$, monic in $X_j$, with $\pjb$ in
    $\IilfA$, and such that $\deg(\pjb)\leq (2d)^{2n}.$
    \end{lemma} 

\begin{proof}
     We let $\frak L^\A$ denote the extension of $\IilfA \cap \C(\A)[\Xb]$ given by 
  \[
  \frak
  L^\A =(\IilfA \cap \C(\A)[\Xb])\cdot \C(\frak A, \Xb_{\leq i-1})[X_i,\dots,X_n].
  \]
  By Corollary \ref{lem:6.2}, $Q_j\in\C(\A)[\textbf{X}_{\leq i-1},X_j],$ exists, monic in $X_j$, with 
    \[
    Q_j(X_j) \in \IilfA \cap \C(\A)[\Xb].
    \]
  Thence,
  \begin{equation}\label{eq:3}
    \C(\frak A, \Xb_{\leq i-1}) \to \C(\frak A,
    \Xb_{\leq i-1})[X_i,\dots,X_n]/\frak L^\A
  \end{equation}
    is an algebraic extension. Let $P_j \in \C(\A)(\textbf{X}_{\leq i-1})[X_j]$ 
be the minimal polynomial of $X_j$ in \eqref{eq:3}, and note that $P_j$ is monic in $X_j.$ 
    Hence, $Q_j$ is also in the extension $\frak{L}^\A$, and thus $P_j$ divides $Q_j$ in $\C(\A)(\textbf{X}_{\leq i-1})[X_j].$ We can therefore write 
    \begin{align*}
    &Q_j = P_jR_j,~~~~ P_j,R_j \in \C(\A)(\textbf{X}_{\leq i-1})[X_j]-\C(\A)(\textbf{X}_{\leq i-1}).
    \end{align*}
    It then follows by Gauss's lemma that 
    \begin{align*}
    Q_j = p_jr_j, ~~~~p_j,r_j \in \C(\A)[\textbf{X}_{\leq i-1}][X_j]-\C(\A),
    \end{align*}
    and such that $\mu_j \in \C(\A)(\Xb_{\leq i-1})$ exists with 
    \[
    P_j = \mu_j p_j,~~~~ R_j = \mu_j^{-1}r_j.
    \]
    Since $Q_j$ is monic in $X_j$, $p_j$ and $r_j$ must also be monic in $X_j$, and $\mu_j$ must be the coefficient of the highest degree term of $P_j$ in $X_j.$ Since $P_j$ is monic in $X_j$, $\mu_j =1$ and hence \[P_j=1\cdot p_j=p_j \in \C(\A)[\textbf{X}_{\leq i-1}][X_j].\]

  \noindent 
  Now, consider the polynomials $\IilfA:$
\[
 \left(F^{\A}, \lagFfA,\udl-1 \right)
\]
in $\C[\A,\Xb,\Lb]$, let $\frak W$ be their zero-set, and let $\deg(\frak W)$ be its
  degree, in the sense of~\cite{H}. Since $\pjb$ is a polynomial in $\C[\A,\Xb]$, Proposition~1 in~\cite{CGR} 
  implies that $\pjb$ has degree at most $\deg(\frak W)$. Since all
  polynomials defining $\frak W$, seen in $\C[\A,\Xb,\Lb]$, have
  degree at most $2d$, the B\'ezout inequality of~\cite{H} gives
  \[\deg(\pjb) \le (2d)^{n+p-i+1} \le (2d)^{2n}.\]
\end{proof}





%%%%%%%%%%%%%%%%%%%%%%%%%%%%%%%%%%%%%%%%%%%%%%%%%%%%%%%%%%%%
%%%%%%%%%%%%%%%%%%%%%%%%%%%%%%%%%%%%%%%%%%%%%%%%%%%%%%%%%%%%
%%%%%%%%%%%%%%%%%%%%%%%%%%%%%%%%%%%%%%%%%%%%%%%%%%%%%%%%%%%%
\subsubsection{Applying the effective Nullstellensatz}
Now we apply the Nullstellensatz for $\pjb$ with the ideal membership for $\IilfA.$ Let $T$ be a new variable; applying the
Nullstellensatz in $\C(\A)[\Xb,\Lb][T]$, and clearing
denominators, we obtain the existence of $\alpha_j$ in
$\C[\A]-\{0\}$ and $ C_{j,\ell},B_j$ in
$\C[\A][\Xb,\Lb][T]$, such that
\begin{align*}
\alpha_j = \sum_{\ell=1}^{n+p-i+1} C_{j,\ell} G_\ell + B_j (1-\pjb T),\\  G_\ell \in 
\left\{ 
F^\A,\lagFfA, \udl-1
\right\}.
\end{align*}
Let us then define 
$$\D_{i}:=\D_{i,1} \alpha_i \cdots \alpha_n D_i \cdots D_n.$$

\begin{lemma}\label{lem:6.4}
Suppose that $\mA \in \C^{n\times n}$ does not cancel $\D_{i}$. Then $F^{\mA}$ satisfies $\bm H_i(1)$ and $\bm H_i(2),$ and the extension
\[
 \C[\Xb_{\leq i-1}]\rightarrow \C[\Xb]/(\IilA \cap \C[\Xb])
\]
is integral.
\end{lemma}

\begin{proof}
By assumption, $\D_{i,1}(\mA)$ is non-zero so that $\mA$ is
invertible, the
ideal defined by $\IilA$ is radical (this follows from Corollary \ref{cor:LagIdealRadical}, with $\ub \in \C^p$ any complex point) and 
$\WiA$ is either empty or $(i-1)$-equidimensional. By Proposition \ref{prop:RadLagPolarV} and the Jacobian Criterion~\cite[Corollary 16.20]{ECA}, we have that $\WilA$ is also either empty or $(i-1)$-equidimensional. Now, if
it is empty, we are done. Otherwise, for $j=i,\dots,n-p+1$, evaluate all indeterminates in $\A$ at the
corresponding entries of $\mA$. This gives us
an equality in $\C[\Xb,\Lb,T]$ of the form
\begin{align*}
a_j = \sum_{\ell=1}^{n+p-i+1} c_{j,\ell} g_\ell + b_j (1-p_j T),\ \  g_\ell \in 
\left\{ 
F^{\mA}, \lagFA, \udl-1
\right\},
\end{align*}
for $a_j$ in $\C$, polynomials $c_{j,\ell}$ and $b_j$ in
$\C[\Xb,\Lb,T]$ and $p_j$ in
$\C[\Xb_{\leq i-1},X_j]$. Since neither $\alpha_j$ nor $D_j$
vanish at $\mA$, $a_j$ is non-zero and the leading coefficient of
$p_j$ in $X_j$ is a non-zero constant.

The conclusion is now routine. Replace $T$ by $1/p_j$ in the
previous equality; after clearing denominators, this gives a
membership equality of the form 
\[
p_j{}^k \in \IilA \cap \C[\Xb],
\]
for some integer $k \ge 1$ (we cannot have $k=0$, since we assumed that $\WiA$ is not empty, which implies that $\WilA$ is not empty). Since $\IilA$ is radical,
$p_j$ is in $\IilA$. Repeating this for all $j$ proves that 
\[
\C[\Xb_{\leq i-1}]\rightarrow\C[\Xb]/(\IilA \cap \C[\Xb])
\]
is integral.
\end{proof}

\noindent
To estimate the degree of $\D_{i}$, what remains is to give an upper
bound on the degrees of $\alpha_i,\dots,\alpha_n$. This will come as an
application of the effective Nullstellensatz given in~\cite{EN}, for
which we first need to determine degree bounds, separately in $\Xb,\Lb,T$
and $\A,$ of the polynomials in the membership relationship. We have
\begin{align*}
\deg_{\Xb,\Lb,T}
\left\{ 
F^{\A},\lagFfA, \udl-1
\right\}
\leq d, ~
\deg_{\Xb,\Lb,T}(1-T\pjb) \leq (2d)^{2n} +1, 
\end{align*}
and we have 
\begin{align*}
\deg_{\A}  
\left\{ 
F^{\A},\lagFfA, \udl-1
\right\} 
\leq d,~\textrm{and }
\deg_{\A}(1-T\pjb)& \leq (2d)^{2n}.
\end{align*}
For each $j \in \{i,\hdots,n-p+1\},$ since  the number of equations in the ideal we consider is less than or equal to the ambient dimension, it follows from \cite[Theorem 0.5]{EN} that  
\[
\deg(\alpha_j) \le (2n+2)d^{2n+1}((2d)^{2n}+1);
\]
we will use
the slightly less precise bound \[\deg(\alpha_j) \le 4n(2d)^{4n}.\] Since $\D_{i,1}$ has degree at most
$nd^{5n}$ and all $D_j$'s have degree at most $(2d)^{2n}$, this gives
the upper bound
$$\deg(\D_i) \le nd^{5n} +  4n^2(2d)^{4n} + n(2d)^{2n} \leq 6n^2(2d)^{5n}.$$


%%%%%%%%%%%%%%%%%%%%%%%%%%%%%%%%%%%%%%%%%%%%%%%%%%%%%%%%%%%%
%%%%%%%%%%%%%%%%%%%%%%%%%%%%%%%%%%%%%%%%%%%%%%%%%%%%%%%%%%%%
%%%%%%%%%%%%%%%%%%%%%%%%%%%%%%%%%%%%%%%%%%%%%%%%%%%%%%%%%%%%
\subsubsection{Proof of $\textbf{H}_i(3)$}
Now assume that $\ub \in \mathscr{O}$, where $\sO$ is from $\eqref{eq:sO}.$ It remains to show that if $\mA \in \C^{n\times n}$ does not cancel $\D_{i}$ then
\[
 \C[\Xb_{\leq i-1}]\rightarrow \C[\Xb]/\IiA
\]
is integral. 
\noindent
By Lemma \ref{lem:6.4}, the extension
\[
 \C[\Xb_{\leq i-1}]\rightarrow \C[\Xb]/(\IilA \cap \C[\Xb])
\]
is integral, and thus polynomials 
\[
Q_j \in \C[\Xb_{\leq i-1},T]
\]
exists, monic in $T$, for each $j \in \{i,\hdots,n-p+1\},$ with
\begin{align*}
Q_j(X_1,\hdots,X_{i-1},X_j) \in \IilA \cap \C[\Xb].
\end{align*}
%
Since $\ub \in \sO$, by Proposition~\ref{prop:polarVs1}, $Q_j \in \IiAr.$ Hence, there exists some $k \in \mathbb{N}-\{0\}$ with $Q_j^{k} \in\IiA$, where $Q_j^k$ is monic in $X_j,$ and therefore
\[
 \C[\Xb_{\leq i-1}]\rightarrow \C[\Xb]/\IiA
\]
is integral. This completes the proof of Theorem \ref{theo:NoetherPositionG}.

%%%%%%%%%%%%%%%%%%%%%%%%%%%%%%%%%%%%%%%%%%%%%%%%%%%%%%%%%%%%
%%%%%%%%%%%%%%%%%%%%%%%%%%%%%%%%%%%%%%%%%%%%%%%%%%%%%%%%%%%%
%%%%%%%%%%%%%%%%%%%%%%%%%%%%%%%%%%%%%%%%%%%%%%%%%%%%%%%%%%%%

\section{Proof of $\textbf{H}_i^{'}$}\label{Sec:Hip}

Consider a sequence of polynomials $F= (f_1,\hdots,f_p) \in
\ZZ[X_1,\hdots,X_n]^p,$ with degrees at most $d$, defining a radical
ideal $\langle F \rangle \subset \C[X_1,\hdots,X_n]$ and a smooth
variety $V=V(F) \subset \C^n$ with $\dim V = n-p$.  We now assume that
$F$ satisfies $\textbf{H}_i$, and we prove the following: {\em there
  exists a non-zero polynomial
  \[
  \Xi_{i,1} \in
  \C[S_1,\dots,S_{i-1}]
  \]
  of degree at most $d^{3n}$ such that if $
  \bm \sigma = (\sigma_1,\hdots,\sigma_{i-1}) \in \C^{i-1}$ does not cancel
  $\Xi_{i,1}$, then for any $(\xb,\lb) \in \sW(\pi_i,V(F))$, the Jacobian of the system of polynomials
  $$\left(X_1-\sigma_1,\dots,X_{i-1}-\sigma_{i-1},F,\lagF\right)$$ has full rank $n+p-1$.}

\smallskip

Let $\Psi: \C^{n+p} \times \C^{i-1} \rightarrow \C^{n}$ be the mapping defined by the polynomials
\[
  \left(X_1-S_1,\dots,X_{i-1}-S_{i-1},F,\lagF\right).
\]
%
\begin{lemma}
  $\bz$ is a regular value of $\Psi.$
\end{lemma}
\begin{proof}
At all zeros $(\xb,\lb,\bm \sigma)$ of $\Psi,$ the Jacobian matrix of
$\Psi$ has full rank $n+p-1$. Indeed, indexing columns 
    by 
    \[
    X_1,\dots,X_n,L_1,\hdots,L_p,S_1,\dots,S_{i-1},
    \]
    this matrix is equal to
    \[
    \left[ 
    \begin{array}{cc}
    \begin{array}{cc}
    \bI_{i-1}     & \bz_{(i-1)\times (n+p-i+1)}  
    \end{array}  &-\bI_{i-1}\\
    \jac_{(\xb,\bm l)}\left( F,\lb \cdot \jac_{\xb} (F,i)  \right) & \bz_{(n+p-i)\times (i-1)}
    \end{array}
    \right].
    \]
    Recall that by $\bm G_i(2)$, the Jacobian matrix $\jac_{(\xb,\bm l)}\left( F,\lb \cdot \jac_{\xb} (F,i) \right)$ has full rank $n+p-i$ at any zero $(\xb,\bm l)$. Hence, the entire    
    matrix must have full rank $n+p-1$. Thus,  $\bz$ is a regular value of $\Psi.$
    \end{proof}

Since all polynomials defining $\Psi$ have degree at most $d$, it
follows by Proposition~\ref{prop:weak_t} that there exists a non-zero
polynomial $\Xi_{i,1}$ in $\C[S_1,\dots,S_{i-1}]$ of degree at most
$d^{(n+p)+(n)}\leq d^{3n},$ with the property that, if $\Xi_{i,1}(\bm \sigma)\neq 0$ then at any root $(\xb,\lb)$ of
\[
\left(X_1-\sigma_1,\dots,X_{i-1}-\sigma_{i-1},F,\lagF\right),
\] 
  the Jacobian matrix of these
equations has full rank $n+p-1$. 


%%%%%%%%%%%%%%%%%%%%%%%%%%%%%%%%%%%%%%%%%%%%%%%%%%%%%%%%%%%%
%%%%%%%%%%%%%%%%%%%%%%%%%%%%%%%%%%%%%%%%%%%%%%%%%%%%%%%%%%%%
%%%%%%%%%%%%%%%%%%%%%%%%%%%%%%%%%%%%%%%%%%%%%%%%%%%%%%%%%%%%

\section{Proof of Theoreom \ref{theo:main}}

%%%%%%%%%%%%%%%%%%%%%%%%%%%%%%%%%%%%%%%%%%%%%%%%%%%%%%%%%%%%

\subsection{Pseudocode}
%
The following is our main algorithm. In step 4, we use \cite[Algorithm 2]{SH} to solve a square
system. This subroutine is randomized; in order to guarantee a higher
probability of success, we repeat the calculation $k$ times, for a
well-chosen parameter $k$. This subroutine also requires that the input system be given by a
straight-line program. We build it (at Step 3) in the straightforward
manner already suggested in the introduction: given $F=(f_1,\hdots,f_p)$ in $\C[X_1,\hdots,X_n]$, we can build
a straight-line program that evaluates each $f_i$ in $O(d^n)$ operations, by
computing all monomials of degree up to $d$, multiplying them by the
corresponding coefficients in $f_i$, and adding results. To obtain a
straight-line program for $f_i^\mA$, we add $O(n^2)$ steps corresponding
to the application of the change of variables $\mA$. The number of operations here is thus
\[
O(nd^n + n^3) = O^{\sim}(d^n).
\]
From this, we can compute and evaluate the required partial derivatives in the Jacobian of $F^\mA$ in
\[
O(n^2d^n) = O^{\sim}(d^n)
\]
operations ~\cite{BaSt83}.
Then, the matrix vector product with the vector of Lagrange multipliers adds a cost that is polynomial in $n$ and which we can therefore neglect in the soft oh notation. Finally, we add the linear equations
$X_1-\sigma_1,\hdots,X_{i-1}-\sigma_{i-1}$; this gives $\Gamma_i$, and the total cost for computing the straight line program is $O^{\sim}(d^n).$
%\newpage
%%%%%%%%%%%%%%%%%%%%%%%%%%%%%%%%%%%%%%
%%%%% ALGORITHM %
    %%%%%%%%%%%%%%%%%
    \begin{algorithm}[!h]
    \KwIn{$F=(f_1,\hdots,f_p) \in \ZZ[X_1,\hdots,X_n]^p$ with $\deg f_i \leq d$ and $\htt f_i \leq b$, and $0 < \epsilon < 1$}
    \KwOut{$n$ zero-dimensional parameterizations, the union of whose zeros
    includes at least one point in each connected component of $V(F) \cap \R^n$, with probability at least $1-\epsilon$} 

    \nl Construct \[S := \{1,2,\hdots,\lceil 4\epsilon^{-1}6n^3(2d)^{5n} \rceil\},\] \[T :=
    \{1,2,\hdots,\lceil 4\epsilon^{-1}3n^2(nd)^{3n} \rceil \},\] and 
    \[R :=
    \{1,2,\hdots,\lceil 4\epsilon^{-1}(nd)^n \rceil \},\]
    and randomly choose $\mA \in
    S^{n^2}, \bm \sigma \in T^{n-1},$ and $\ub \in R^p$\; 


    \caption{{Main Algorithm} \label{alg:1}} 

    \nl \For{$i\gets1$ \KwTo $n$}{
    \nl Build a straight-line program $\Gamma_i$ that computes the equations
    \[
    \left\{X_1-\sigma_1,\hdots,X_{i-1}-\sigma_{i-1},F^{\mA}, \lagFA,\sum_{i=1}^{p}u_iL_i-1 \right\}\;
    \]
    \nl Run \cite[Algorithm 2]{SH} $k \geq \lg(4n/\epsilon)$ times
    with input $\Gamma_i$\;

    \nl Let $\mathscr{Q}_i$ be the highest cardinality
    zero-dimensional parameterization returned in step 4\; 
    
    \nl  Compute a parameterization of the projection of $\mathscr{Q}_i$ onto the $\Xb$-space, and let $\mathscr{Q}^{'}_i$ denote this new parameterization \; 
    
    } 
    \nl  \Return $[\mathscr{Q}^{'}_1,\hdots,\mathscr{Q}^{'}_{n}]$.
    \end{algorithm}
    %%%%% ALGORITHM %
    %%%%%%%%%%%%%%%%%
    


If $F^\mA$ satisfies ${\bm H}_i$, and $F^\mA$ and $\bm \sigma $ satisfy $\bm H_i^{'}$, for all $i \in \{1,\hdots,n-p+1\}$, then Theorem~2 in~\cite{EMP}  tells us that the parameterizations returned in step 5 are zero dimensional. Then, if $\ub$ satisfies $\bm H_{i}^{''},$ for all $i \in \{1,\hdots,n-p+1\}$, Proposition \ref{prop:correctness} tells us that the polar varieties are contained in the projections of the Lagrangian systems and therefore Theorem~2 in~\cite{EMP} also establishes that the parameterizations returned in step 7 will contain one point on each connected component of $V \cap \R^n$. 




%%%%%%%%%%%%%%%%%%%%%%%%%%%%%%%%%%%%%%%%%%%%%%%%%%%%%%%%%%%%
%%%%%%%%%%%%%%%%%%%%%%%%%%%%%%%%%%%%%%%%%%%%%%%%%%%%%%%%%%%%
%%%%%%%%%%%%%%%%%%%%%%%%%%%%%%%%%%%%%%%%%%%%%%%%%%%%%%%%%%%%
\subsection{Bit operation cost} 
%
The following lists the costs for each step of Algorithm \ref{alg:1}:

\noindent
    (1) We defined $S := \{1,2,\hdots,\lceil 3\epsilon^{-1}6n^3(2d)^{5n} \rceil \}$
      and therefore the height of any $a_{i,j} \in S$ is at most
    \[
    \log 3/{\epsilon} + \log(6n^3(2d)^{5n}) \in O^{\sim}(\log 1/{\epsilon} + n\log d).
    \]
Since $|R| < |T| < |S|,$ we also have that the height of any $\sigma_{k} \in T$ and $u_{l} \in R$ is at most the same.
\newline 
\noindent  
    (3)
    After computing the partial derivatives in the Jacobian matrix, the height grows by at
    most another factor of $\log d$. Thus, all polynomials in the
    system considered at Step 3 have height 
    \[
    O^{\sim}(b + d(\log 1 /
    \epsilon + n\log d))
    =
        O^{\sim}(b + d\log 1 /
    \epsilon + dn).
    \]
    All integer coefficients appearing in $\Gamma_i$ 
    satisfy the same bound.    
    %% Furthermore, preparing the
    %% straight line program for the gradient of $f$ does not introduce
    %% large integers. The integers in $\grad (f)$ are of the same
    %% magnitude as in $\Gamma$.
     
\noindent  
    (4)   
    As a result, after applying \cite[Algorithm 2]{SH} $k$ times
      for each index $i$, with $k = O(\log n + \log 1 / \epsilon)$,
      the total boolean cost of the algorithm is
  \[
O^{\sim}(d^{3n+2p+1}(\log1/\epsilon)(b + \log1/\epsilon))
  \]
      where the polynomials in the output have degree at most $d^{n+p},$ and height at most
  \[
O^{\sim}(d^{n+p+1}(b + \log1/\epsilon)).
  \]
  (6) ? 
  \par 
This proves the runtime estimate, as well as our bounds on the height
of the output.




%%%%%%%%%%%%%%%%%%%%%%%%%%%%%%%%%%%%%%%%%%%%%%%%%%%%%%%%%%%%
%%%%%%%%%%%%%%%%%%%%%%%%%%%%%%%%%%%%%%%%%%%%%%%%%%%%%%%%%%%%
%%%%%%%%%%%%%%%%%%%%%%%%%%%%%%%%%%%%%%%%%%%%%%%%%%%%%%%%%%%%
\subsection{Probability of success} 
Let $\Delta_i \in \C[\A]$ be the polynomials from
Theorem~\ref{theo:NoetherPositionG}. Denote by $\Delta := \prod_{i=1}^n \D_i,$ and
note that
\begin{align}
    \deg \Delta \leq \sum_{i=1}^{n-p+1} \deg \Delta_i \leq 6n^3(2d)^{5n}.
\end{align}
If $\mA \in \C^{n \times n}$ does not cancel $\Delta,$ then $\mA$ is
invertible and $F^\mA$ satisfies $\bm H_i$ for all $i$ in
$\{1,\hdots,n-p+1\}.$ Now, assuming that $\mA$ is such a matrix, let
$\Xi_i\in \C[S_1,\dots,S_{p+i-1}]$ be the polynomials from Theorem \ref{theo:sigmaG} applied to
$F^{\mA}.$ Denote by $\Xi := \prod_{i=1}^n \Xi_i,$ and note that
\begin{align}
    \deg \Xi \leq \sum_{i=1}^{n-p+1} \deg \Xi_i \leq nd^{5n}.
\end{align}
If $\bm \sigma \in \C^{i-1}$ does not cancel $\Xi$, then $F^{\mA}$ and
$\bm \sigma$ satisfy $\bm H_i^{'}$ for all $i \in
\{1,\hdots,n-p+1\}.$ And finally, denote by $\Upsilon := \prod_{i=1}^{n-p+1} \Upsilon_i,$ and
note that
\begin{align}
    \deg \Upsilon \leq \sum_{i=1}^{n-p+1} \deg \Upsilon_i \leq n(nd)^n.
\end{align}
If $\ub \in \C^{p}$ does not cancel $\Upsilon$, then $\ub$ satisfies $\bm H_i^{''}$ for all $i \in
\{1,\hdots,n-p+1\}.$
\par 
As we argued above, the algorithm is guaranteed to
succeed, as long as our call to Algorithm 2 in~\cite{SH} succeeds. That latter reference establishes that by repeating the calculation $k$ times, and keeping the output of highest degree among those $k$ results, we succeed with probability at least $1-(1/2)^k$. When Algorithm 2 does not succeed, it either returns a proper subset of the solutions, or FAIL. Note that Algorithm 2 is shown to succeed in a single run with  probability at least $1-11/32,$ and we bound the probability of success with $1-1/2$ for simplicity. 
Now, by construction of
\[
S := \{1,2,\hdots,\lceil 4\epsilon^{-1}6n^3(2d)^{5n}\rceil \},
\] 
\[
T := \{1,2,\hdots,\lceil 4\epsilon^{-1}3n^2(nd)^{3n} \rceil \},
\] 
and 
\[
R := \{1,2,\hdots,\lceil 4\epsilon^{-1}n(nd)^n \rceil \},
\]
where $\mA \in S^{n^2}, \bm \sigma \in T^{n-1}$ and $\ub \in R^{p}$ are randomly chosen, we have 
%
\[
\pr[\Delta(\mA)=0] \leq  \frac{\deg\Delta}{|S|} = \epsilon/4,
\]
\[
\pr[\Xi(\bm \sigma)=0] \leq  \frac{\deg\Xi}{|T|} = \epsilon/4,
\]
and
\[
\pr[\Upsilon(\ub)=0] \leq  \frac{\deg\Upsilon}{|R|} = \epsilon/4.
\]
Let $\mathscr{E}$ be the event that the parameterizations
$[\mathscr{Q}_1^{'},\hdots,\mathscr{Q}_{n}^{'}]$ returned in step 7 of
Algorithm \ref{alg:1} are correct. Then, the probability of success is equal to
\begin{align*}
 &\pr[\Delta(\mA)\ne 0] \times \pr[\Xi(\bm \sigma)\ne 0 ~|~\Delta(\mA) \ne 0] \times \pr[\Upsilon(\ub)\ne 0 ~|~\Delta(\mA)\Xi(\bm \sigma) \ne 0]\\ \times ~& \pr[\mathscr{E}~|~
   \Delta(\mA)\Xi(\bm \sigma)\Upsilon(\ub) \ne 0].
\end{align*}
Set $k = \lg(4n/\epsilon)$ so that 
\[
(1-2^{-k})^n = (1 - \epsilon/(4n))^n \geq 1 - \epsilon/4,
\]
by Bernoulli's inequality. Therefore, 
\begin{align*}
\pr[\textrm{success}] &\geq (1- \epsilon/4)(1- \epsilon/4)(1-
\epsilon/4) \pr[\mathscr{E}~|~
   \Delta(\mA)\Xi(\bm \sigma)\Upsilon(\ub) \ne 0]\\ 
& \geq (1- \epsilon/4)(1- \epsilon/4)(1-
\epsilon/4)(1-2^{-k})^n \\ & \geq (1- \epsilon/4)(1- \epsilon/4)(1- \epsilon/4)(1-
\epsilon/4)\\ & \geq 1 - \epsilon.
\end{align*}
This finishes the proof of theorem \ref{theo:main}.
%



%======================================================================
\section{Conclusions}\label{chap:conclusions}
%======================================================================

%%%%%%%%%%%%%%%%%%%%%%%%%%%%%%%%%%%%%%%%%%%%%%%%%%%%%%%%%%%%%%
%%%%%%%%%%%%%%%%%%%%%%%%%%%%%%%%%%%%%%%%%%%%%%%%%%%%%%%%%%%%%%
%%%%%%%%%%%%%%%%%%%%%%%%%%%%%%%%%%%%%%%%%%%%%%%%%%%%%%%%%%%%%%
\subsection{Contributions}
Our main contributions were to analyze precisely what conditions on our
parameter choices guarantee success. And we accomplished this by
revisiting key ingredients in the proofs given
in~\cite{BaGiHeMb97} and~\cite{EMP}, and giving quantitative versions
of these results, bounding the degrees of the hypersurfaces we have to
avoid.  



%%%%%%%%%%%%%%%%%%%%%%%%%%%%%%%%%%%%%%%%%%%%%%%%%%%%%%%%%%%%%%
%%%%%%%%%%%%%%%%%%%%%%%%%%%%%%%%%%%%%%%%%%%%%%%%%%%%%%%%%%%%%%
%%%%%%%%%%%%%%%%%%%%%%%%%%%%%%%%%%%%%%%%%%%%%%%%%%%%%%%%%%%%%%
\subsection{Further work}
This work should be seen as a step toward the analysis of
further randomized algorithms in real algebraic geometry.  In particular, randomized algorithms for deciding {\em
  connectivity queries} on smooth and bounded algebraic sets have been
developed in a series of papers
\cite{SchostMohabBabySteps2011,SchostMohabBabySteps2014}, and could be
revisited using the techniques introduced here. Indeed, we have accomplished some of the work that is needed for these references; these algorithms require Noether position for polar varieties and transversality for algebraic sets. 






%\blinddocument
\bibliographystyle{plain}
\bibliography{refs.bib}
%\printbibliography

















\end{document}
