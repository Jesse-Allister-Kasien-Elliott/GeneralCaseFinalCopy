\documentclass[a4paper]{article}
\usepackage[margin=25mm]{geometry}
\usepackage{amsmath}
\usepackage{amsfonts}
\usepackage{amssymb}
\usepackage{graphicx}
%\pagenumbering{gobble}
\usepackage{verbatim}
\immediate\write18{texcount -tex -sum  \jobname.tex > \jobname.wordcount.tex}
%%%%%%%%%%%%%%%%%%%%%%%%%%%%%%%%%%%%%%%%%%%%%%%%%%%%%%%%%%%%%%%%%%%%%%%%%%%%%%
%%%%%%%%%%%%%%%%%%%%%%%%%%%%%%%%%%%%%%%%%%%%%%%%%%%%%%%%%%%%%%%%%%%%%%%%%%%%%%
% packages 
\usepackage[ruled,vlined]{algorithm2e}
\usepackage[titles]{tocloft}%----------------------------------------
\usepackage{mathrsfs}
\usepackage{bm}
\usepackage{enumitem}
\usepackage{amsmath}
\usepackage{amssymb}
\usepackage{amsthm}
\usepackage{chngcntr}
\usepackage{titlesec}
\usepackage{stfloats}
\usepackage{hyperref}
\hypersetup{linktocpage}
%\usepackage{xcolor}
\usepackage[dvipsnames]{xcolor} 
%\definecolor{ao(english)}{rgb}{0.0, 0.5, 0.0}
\hypersetup{
    colorlinks,
    citecolor=red,
    filecolor=black,
    linkcolor=red ,
    urlcolor=black
}
%%%%%%%%%%%%%%%%%%%%%%%%%%%%%%%%%%%%%%%%%%%%%%%%%%%%%%%%%%%%%%%%%%%%%%%%%%%%%%
%%%%%%%%%%%%%%%%%%%%%%%%%%%%%%%%%%%%%%%%%%%%%%%%%%%%%%%%%%%%%%%%%%%%%%%%%%%%%%
% new commands and shortcuts
%%%%%%%%%%%%%%%%%%%%%%%%%%%%%%%%%%%%%%%%%%%%%%%%%%%%%%%%%%%%%%%%%%%%%%%%%%%%%%





\newcommand\myeq{\mathrel{\stackrel{\makebox[0pt]{\mbox{\normalfont\tiny ?}}}{=}}}





% script variables 
\def\sO{\mathscr{O}}
\def\sL{\mathscr{L}}
\def\sLA{\mathscr{L}^{\mA}}
\def\sLfA{\mathscr{L}^{\A}}
\def\sA{\mathscr{A}}
\def\scrQ{\ensuremath{\mathscr{Q}}}
\def\rs{\mathscr{R}}
\def\sW{\mathscr{W}}
\def\sI{\mathscr{I}}
% frak variables 
%%%%%%%%%%%%%%%%%%%%%%%%%%%%%%%%%%%%%%%%%%%%%%%%%%%%%%%%%%%%%%%%%%%%%%%%%%%%%%
\def\fA{{\mathfrak A}}
\def\fZ{{\mathfrak Z}}
\def\fZp{{\mathfrak Z^{'}}}
\def\fB{{\mathfrak B}}
\def\fN{{\mathfrak N}}
\def\fD{{\mathfrak D}}
\def\fp{\mathfrak{P}}
\def\Z{\mathfrak{Z}}
\def\A{\mathfrak{A}}
%\def\L{\mathfrak{L}}
%%%%%%%%%%%%%%%%%%%%%%%%%%%%%%%%%%%%%%%%%%%%%%%%%%%%%%%%%%%%%%%%%%%%%%%%%%%%%%
% bold variables 
\def\bs{\textit{\textbf{s}}}
\def\bz{\textit{\textbf{0}}}
\def\Lb{\textit{\textbf{L}}}
\def\Xb{\textit{\textbf{X}}}
\def\Lambdab{\bm{\Lambda}}
\def\Thetab{\bm{\Theta}}
\def\thetab{\bm{\vartheta}}
\def\mA{{\bm A}}
\def\ub{{\bm u}}
\def\lb{{\bm l}}
\def\lxb{{\bm l_{\xb}}}
\def\ax{\bm \alpha_{\xb}} 
\def\xb{\textit{\textbf{x}}}
\def\fb{\textit{\textbf{f}}}
\def\gb{\textit{\textbf{g}}}
\def\ab{\textit{\textbf{a}}}
\def\pb{\textit{\textbf{p}}}
\def\ajb{\overline{a_j}}
\def\bjb{\overline{b_{j,l}}}
\def\pjb{\overline{P_{j}}}
\def\bI{\textbf{I}}
%%%%%%%%%%%%%%%%%%%%%%%%%%%%%%%%%%%%%%%%%%%%%%%%%%%%%%%%%%%%%%%%%%%%%%%%%%%%%%
% tilde variables 
\def\jt{\widetilde{J}}
\def\At{\widetilde{A}}
\def\Yt{\widetilde{Y}}
\def\dtt{\widetilde{d}}
\def\kt{\widetilde{k}}
\def\Dt{\widetilde{D}}
%% \def\dt{\widetilde{d}}
%%%%%%%%%%%%%%%%%%%%%%%%%%%%%%%%%%%%%%%%%%%%%%%%%%%%%%%%%%%%%%%%%%%%%%%%%%%%%%
% greek letters  
\def\vt{\vartheta}
\def\d{\delta}
\def\D{\Delta}
%%%%%%%%%%%%%%%%%%%%%%%%%%%%%%%%%%%%%%%%%%%%%%%%%%%%%%%%%%%%%%%%%%%%%%%%%%%%%%
% declare math operator  
\DeclareMathOperator{\GL}{GL}
\DeclareMathOperator{\LCM}{LCM}
\DeclareMathOperator{\J}{J}
\DeclareMathOperator{\htt}{ht}
\DeclareMathOperator{\sing}{sing}
\DeclareMathOperator{\codim}{codim}
\DeclareMathOperator{\jac}{jac}
%\DeclareMathOperator{\exp}{exp}
\DeclareMathOperator{\grad}{grad}
\DeclareMathOperator{\rank}{rank}
\DeclareMathOperator{\reg}{reg}
\DeclareMathOperator{\rk}{rank}
\def\minors{\textrm{Minors}(F,p)}
\def\minorsA{\textrm{Minors}(F^{\mA},p)}
\def\minorsfA{\textrm{Minors}(F^{\A},p)}

%%%%%%%%%%%%%%%%%%%%%%%%%%%%%%%%%%%%%%%%%%%%%%%%%%%%%%%%%%%%%%%%%%%%%%%%%%%%%%
% other math notation shortcuts  
\def\pa{\partial}
\newcommand{\softO}{{O^{\sim}}}
\def\dt{s}
%%%%%%%%%%%%%%%%%%%%%%%%%%%%%%%%%%%%%%%%%%%%%%%%%%%%%%%%%%%%%%%%%%%%%%%%%%%%%%
% sets of numbers
\newcommand{\ZZ}{{\mathbb{Z}}}
\def\C{\mathbb{C}}
\def\Q{\mathbb{Q}}
\def\R{\mathbb{R}}
\def\K{\mathbb{K}}
\def\pr{\mathbb{P}}
\def\P{\mathscr{P}}
%%%%%%%%%%%%%%%%%%%%%%%%%%%%%%%%%%%%%%%%%%%%%%%%%%%%%%%%%%%%%%%%%%%%%%%%%%%%%%
% polar varieties
% minors 
\def\Wi{W(\pi_i,V)}
\def\WiA{W(\pi_i,V(F^{\mA}))}
\def\WifA{W(\pi_i,V(F^{\A}))}
\def\I{\mathfrak{I}}
\def\IA{\mathfrak{I}^{\mA}}
\def\IfA{\mathfrak{I}^{\A}}
\def\Ii{\mathfrak{I}(i,F)}
\def\IiA{\mathfrak{I}(i,F^A)}
\def\Ir{\sqrt{\mathfrak{I}}}
\def\Iir{\sqrt{\mathfrak{I}(i,F)}}
\def\IiAr{\sqrt{\mathfrak{I}(i,F^{\mA})}}
\def\IAr{\sqrt{\mathfrak{I}^{\mA}}}
\def\IfAr{\sqrt{\mathfrak{I}^{\A}}}
\def\IifAr{\sqrt{\mathfrak{I}(\pi_i,F^{\A})}}
\def\IifA{\mathfrak{I}(\pi_i,F^{\A})}
\def\ji{\jac_{\xb}(F,i)}
\def\jiA{\jac_{\xb}(F^{\mA},i)}
\def\jifA{\jac_{\xb}(F^{\A},i)}
\DeclareMathOperator{\lag}{Lagrange}
\def\lagF{{\bm{Lagrange}}(F,i,(L_1,\hdots,L_c))}
\def\lagFA{{\bm{Lagrange}}(F^{\mA},i,(L_1,\hdots,L_c))}
\def\lagFAl{{\bm{Lagrange}}(F^{\mA},i,(l_1,\hdots,l_c))}
\def\lagFfA{{\bm{Lagrange}}(F^{\A},i,(L_1,\hdots,L_c))}
%\def\W{\mathfrak{I}}
% lagrange 
\def\Il{\mathscr{I}_{\ub}}
\def\Iil{\mathscr{I}_{\ub}(i,F)}
\def\IilA{\mathscr{I}_{\ub}(i,F^{\mA})}
\def\IilAnu{\mathscr{I}(i,F^{\mA})}
\def\IilfA{\mathscr{I}_{\ub}(i,F^{\A})}
\def\IilfAV{\mathscr{I}_{\bm V}(i,F^{\A})}

\def\IlfA{\mathscr{I}_{\ub}^{\A}}
\def\IlA{\mathscr{I}_{\ub}^{\mA}}
\def\jiA{\jac_{\xb}(F^{\mA},i)}
\def\Wil{\mathscr{W}_{\ub}(\pi_i,V)}
\def\WilA{\mathscr{W}_{\ub}(\pi_i,V^{\mA})}
\def\WilAnu{\mathscr{W}(\pi_i,V^{\mA})}
\def\WilfA{\mathscr{W}_{\ub}(\pi_i,V^{\fA})}
\def\Wl{\mathscr{W}_{\ub}}
\def\WlA{\mathscr{W}_{\ub}^{\mA}}
\def\udl{\sum_{i=1}^pu_iL_i}
\def\Vdl{\sum_{i=1}^pV_iL_i}
%%%%%%%%%%%%%%%%%%%%%%%%%%%%%%%%%%%%%%%%%%%%%%%%%%%%%%%%%%%%%%%%%%%%%%%%%%%%%%
% brackets
\def\la{\langle}
\def\ra{\rangle}
%%%%%%%%%%%%%%%%%%%%%%%%%%%%%%%%%%%%%%%%%%%%%%%%%%%%%%%%%%%%%%%%%%%%%%%%%%%%%%
% matrices 
\def\bbm{\begin{bmatrix}}
\def\ebm{\end{bmatrix}}
%%%%%%%%%%%%%%%%%%%%%%%%%%%%%%%%%%%%%%%%%%%%%%%%%%%%%%%%%%%%%%%%%%%%%%%%%%%%%%
% other shortcuts 
\def\td{{\bf todo}}
\def\gip{\Gamma_i^{'}}
\def\gi{\Gamma_i}
%%%%%%%%%%%%%%%%%%%%%%%%%%%%%%%%%%%%%%%%%%%%%%%%%%%%%%%%%%%%%%%%%%%%%%%%%%%%%%
%%%%%%%%%%%%%%%%%%%%%%%%%%%%%%%%%%%%%%%%%%%%%%%%%%%%%%%%%%%%%%%%%%%%%%%%%%%%%%
% Theorems 
\newtheorem{theorem}{Theorem}[section]
\newtheorem{corollary}[theorem]{Corollary}
\newtheorem{lemma}[theorem]{Lemma}
\newtheorem{ex}[theorem]{Example}
\newtheorem{observation}[theorem]{Observation}
\newtheorem{prop}[theorem]{Proposition}
\newtheorem{definition}[theorem]{Definition}
\newtheorem{claim}[theorem]{Claim}
\newtheorem{fact}[theorem]{Fact}
\newtheorem{assumption}[theorem]{Assumption}
\newtheorem{remark}[theorem]{Remark}
\newtheorem{question}{Question}
%\newtheorem*{ex}{Example}
\newtheorem{cor}[theorem]{Corollary}
%
%
%
% Keywords command
\providecommand{\keywords}[1]
{
  \small	
  \textbf{\textit{Keywords---}} #1
}
%
%
%
\title{Bit complexity for computing one point on each connected component of a smooth real algebraic set}
\author{Jesse Elliott$\dagger$, Mark Giesbrecht$\dagger$, Eric Schost$\dagger$  \\
        \small $\dagger$David R. Cheriton School of Computer Science, University of Waterloo, On, Canada \\
}
\date{} % Comment this line to show today's date
%
%
%
\begin{document}
%
\maketitle
%
\tableofcontents
\newpage
%
%
%
\begin{abstract}
We present a full analysis of the bit complexity of an efficient
  algorithm for the computation of at least one point in each
  connected component of a smooth real algebraic set. This work is a continuation of an analysis by Elliott, Giesbrecht, and Schost ({\em On the bit complexity of finding points on connecting components of a smooth real hypersurface}, ISSAC'20), where the hypersurface case was analyzed. In this paper, we extend the analysis to the more general algebraic set case; we determine the bit complexity for computing at least one point in each connected component of a smooth real algebraic set, defined by the vanishing of a finite sequence of polynomials.  
    
    The analysis is of an algorithm by Safey El Din and Schost ({\em
    Polar varieties and computation of one point in each connected
    component of a smooth real algebraic set}, ISSAC'03). This
  algorithm uses random changes of variables that are proved to
  generically ensure certain desirable geometric properties. The
  cost of the algorithm was given in an algebraic complexity
  model. We analyze the bit complexity and the error probability. And we provide a
  quantitative analysis of the genericity statements. 

\keywords{Real algebraic geometry; regular sequences; weak transversality; Noether position; complexity}
\end{abstract}
%
%
%



%%%%%%%%%%%%%%%%%%%%%%%%%%%%%%%%%%%%%%%%%%%%%%%%%%%%%%%%%%%%
%%%%%%%%%%%%%%%%%%%%%%%%%%%%%%%%%%%%%%%%%%%%%%%%%%%%%%%%%%%%
%%%%%%%%%%%%%%%%%%%%%%%%%%%%%%%%%%%%%%%%%%%%%%%%%%%%%%%%%%%%

\section{Introduction}
\label{sec:overview}

%
\subsection{Background and problem statement}
%
In this work, we focus on the problem of computing one point in each connected component of a real algebraic
set $S$. This is a basic but important subroutine in real algebraic and semi-algebraic
geometry that is used in higher level algorithms. It is also useful on its own, since it allows one to
decide if $S$ is empty or not, and it allows one to determine an upper bound on the number of connected components of $S$.  

We consider the case where $S$ is given as $S=V \cap
\R^n$, where $V=V(F) \subset \C^n$ is a complex algebraic set defined
by a sequence of polynomials $F = (f_1,\hdots,f_p)$ in $\ZZ[X_1,\dots,X_n]$.  This work is a continuation of an analysis by Elliott, Giesbrecht, and Schost ({\em On the bit complexity of finding points on connecting components of a smooth real hypersurface}, ISSAC'20), where the hypersurface case was analyzed. In this paper, we extend the analysis to the algebraic set case.

Suppose that the polynomials in $F = (f_1,\hdots,f_p)$ each have total degree at most $d$, and coefficients of bit-size $h$. Without making any assumption on $F$, the algorithm given
in~\cite[Section~13.1]{BaPoRo03} solves our problem using $p^{n+1}d^{O(n)}$
operations in $\Q$. And in addition, the output of the algorithm is
represented by polynomials of degree $d^{O(n)}$ and with coefficients of
bit-size $hd^{O(n)}$.  The key idea of the algorithm originates with~\cite{GrVo88}, where sample points are found through the computation of
critical points of well-chosen functions on $V(F)$.


The number of connected components of $V(F)$ admits the lower bound $\td$, so up to polynomial factors this result is
optimal. However, due to the generality of the algorithm, the constant
hidden in the exponent $O(n)$ in its runtime is
large, because the algorithm relies on infinitesimal deformations that affect
the runtime non-trivially.


In this paper, we assume that
$V=V(F)$ is a {\em smooth} complex algebraic set. And we place ourselves in
the continuation of the line of work initiated by~\cite{BaGiHeMb97}:
that reference deals with cases where $V$ is smooth and $V \cap \R^n$
is compact, pointing out how {\em polar varieties} (that were
introduced in the 1930's in order to define characteristic
classes~\cite{Piene78,Teissier88}) can play a role in effective real
geometry. This paper was extended in several directions: to $V$ being
a smooth complete intersection, still with $V\cap \R^n$
compact~\cite{BaGiHeMb01}, then without the compactness
assumption~\cite{EMP,BaGiHePa05}; the smoothness assumption was then partly
dropped in~\cite{BaGiHe14,BaGiHeLePa12}.



As already mentioned, this work is a continuation of \cite{ElGiSh20}, where the starting point is the algorithm in~\cite{EMP}. Its runtime is  polynomial in the
complexity of evaluation of the input system. The algorithm is randomized and we need to assume
that we are in generic coordinates; this is done by applying a random
change of coordinates prior to all computations. In addition, the
algorithm relies on procedures for solving systems of polynomial
equations that are also randomized.  Altogether, we choose
$n^{O(1)}$ random vectors, each of them in an affine space of
dimension $n^{O(1)}$; every time a choice is made, there exists a
hypersurface of the parameter space that one has to avoid in order to
guarantee success. In this paper, we extend the work started in \cite{ElGiSh20}, from te hypersurface case to algebraic sets defined by regular sequences. And we give a
complete analysis of the probability of success and the bit complexity.








%%%%%%%%%%%%%%%%%%%%%%%%%%%%%%%%%%%%%%%%%%%%%%%%%%%%%%%%%%%%%
%%%%%%%%%%%%%%%%%%%%%%%%%%%%%%%%%%%%%%%%%%%%%%%%%%%%%%%%%%%%%
%%%%%%%%%%%%%%%%%%%%%%%%%%%%%%%%%%%%%%%%%%%%%%%%%%%%%%%%%%%%%
\subsection{Bit size and data structures}
The {\em height} of a non-zero $a=u/v \in \Q$ is the maximum of
$\ln(|u|)$ and $\ln(v),$ where $u \in \mathbb{Z}$ and $v \in
\mathbb{N}$ are coprime. For a polynomial $f$ with rational
coefficients, if $v \in \mathbb N$ is the minimal common denominator
of all non-zero coefficients of $f$, then the \textit{height}
$\htt(f)$ of $f$ is defined as the maximum of the logarithms of $v$
and of the absolute values of the coefficients of $vf$. Note that when $f$ has integer coefficients, this is simply the maximum of the logarithms of the absolute values
of these coefficients.


The output of the algorithm is a finite set in $\overline{\Q}{}^n$. To
represent it, we rely on a widely used data structure based on
univariate
polynomials~\cite{Kronecker82,Macaulay16,GiMo89,GiHeMoPa95,ABRW,GiHaHeMoMoPa97,GiHeMoMoPa98,Rouillier99}.
Consider a zero-dimensional algebraic set $S \subset \C^n$ defined
over $\Q$. A {\em zero-dimensional parametrization}
$\scrQ=((q,v_1,\dots,v_n),\lambda)$ of $S$ consists in polynomials
$(q,v_1,\dots,v_n)$, such that $q\in \Q[T]$ is monic and squarefree,
all $v_i$'s are in $\Q[T]$ and satisfy $\deg(v_i) < \deg(q)$, and in a
$\Q$-linear form $\lambda$ in variables $X_1,\dots,X_n$, such that
\begin{itemize}
\item $\lambda(v_1,\dots,v_n)=T q' \bmod q$;
\item we have the equality
  $S=\left \{\left(
      \frac{v_1(\tau)}{q'(\tau)},\dots,\frac{v_n(\tau)}{q'(\tau)}\right
    ) \ \mid \ q(\tau)=0 \right \}.$
\end{itemize}
The constraint on $\lambda$ says that the roots of $q$ are the values
taken by $\lambda$ on $S$. The parametrization of the coordinates by
rational functions having $q'$ as a denominator goes back
to~\cite{Kronecker82,Macaulay16}: as pointed out in~\cite{ABRW}, it
allows one to control precisely the size of the coefficients of
$v_1,\dots,v_n$.













%%%%%%%%%%%%%%%%%%%%%%%%%%%%%%%%%%%%%%%%%%%%%%%%%%%%%%%%%%%%%
%%%%%%%%%%%%%%%%%%%%%%%%%%%%%%%%%%%%%%%%%%%%%%%%%%%%%%%%%%%%%
%%%%%%%%%%%%%%%%%%%%%%%%%%%%%%%%%%%%%%%%%%%%%%%%%%%%%%%%%%%%%
\subsection{Main result} 

%
%
\begin{theorem}\label{theo:main}
  Let $F= (f_1,\hdots,f_p)\in\ZZ[X_1,\hdots,X_n]^p$ be a sequence of polynomials with $\deg(f_i) \leq d$ and $\htt(f_i) \leq b$. Suppose that the ideal $\langle f_1,\hdots, f_p \rangle $ is radical and that $V=V(F) \subset \C^n$ is 
  smooth with $\dim V = n-p$. Also suppose that $0 < \epsilon < 1.$

  There exists a randomized algorithm that takes $F$ and $\epsilon$ as input and produces $n$ zero-dimensional parameterizations, the union
  of whose zeros includes at least one point in each connected
  component of $V(F) \cap \R^n$, with probability at least
  $1-\epsilon$. Otherwise, the algorithm either returns a proper
  subset of the points, or FAIL.  In any case, the algorithm uses
  \[
O^{\sim}(d^{3n+2p+1}(\log1/\epsilon)(b + \log1/\epsilon))
  \]
  bit operations. The polynomials in
  the output have degree at most $d^{n+p},$ and height 
  \[
O^{\sim}(d^{n+p+1}(b + \log1/\epsilon)).
  \]
\end{theorem}
%
%
\noindent 
Here we assume that $F$ is given as a sequence of dense polynomials.  Following
references such
as~\cite{GiHeMoPa95,GiHaHeMoMoPa97,GiHeMoMoPa98,BaGiHeMb97,EMP}, it
would be possible to refine the runtime estimate by assuming that $F$
is given by a {\em straight-line program} (that is, a sequence of
operations $+,-,\times$ that takes as input $X_1,\dots,X_n$ and
evaluates $F$). Any polynomial of degree $d$ in $n$ variables can be computed by a straight-line program that does $O(d^n)$ operations: evaluate all monomials of degree up to $d$ in $n$ variables, multiply them by their respective coefficients and sum the results. However, some inputs may be given by shorter straight-line program, and the algorithm would actually be able to benefit from this. 






%%%%%%%%%%%%%%%%%%%%%%%%%%%%%%%%%%%%%%%%%%%%%%%%%%%%%%%%%%%%%
%%%%%%%%%%%%%%%%%%%%%%%%%%%%%%%%%%%%%%%%%%%%%%%%%%%%%%%%%%%%%
%%%%%%%%%%%%%%%%%%%%%%%%%%%%%%%%%%%%%%%%%%%%%%%%%%%%%%%%%%%%%
\subsection{Related work}






%%%%%%%%%%%%%%%%%%%%%%%%%%%%%%%%%%%%%%%%%%%%%%%%%%%%%%%%%%%%%
%%%%%%%%%%%%%%%%%%%%%%%%%%%%%%%%%%%%%%%%%%%%%%%%%%%%%%%%%%%%%
%%%%%%%%%%%%%%%%%%%%%%%%%%%%%%%%%%%%%%%%%%%%%%%%%%%%%%%%%%%%%
\subsection{Outline}



















%%%%%%%%%%%%%%%%%%%%%%%%%%%%%%%%%%%%%%%%%%%%%%%%%%%%%%%%%%%%
%%%%%%%%%%%%%%%%%%%%%%%%%%%%%%%%%%%%%%%%%%%%%%%%%%%%%%%%%%%%
%%%%%%%%%%%%%%%%%%%%%%%%%%%%%%%%%%%%%%%%%%%%%%%%%%%%%%%%%%%%
\section{Preliminaries}
Let $\Xb = (X_1, \hdots , X_n)$ be a sequence of variables, and for $l \in \{1,\hdots,n\}$ let $\Xb_{\leq l}$ be the subsequence of variables $(X_1, \hdots , X_l)$.



%%%%%%%%%%%%%%%%%%%%%%%%%%%%%%%%%%%%%%%%%%%%%%%%%%%%%%%%%%%%
%%%%%%%%%%%%%%%%%%%%%%%%%%%%%%%%%%%%%%%%%%%%%%%%%%%%%%%%%%%%
%%%%%%%%%%%%%%%%%%%%%%%%%%%%%%%%%%%%%%%%%%%%%%%%%%%%%%%%%%%%
\subsection{Algebraic sets} 
Consider a sequence of polynomials $F=(f_1,\hdots,f_p)$ in $\C[X_1,\hdots,X_n].$
An \textit{algebraic set}: 
\begin{align*}
V = ~&V\big(F) = V(f_1,\hdots,f_p) \\
= ~&\left\{(a_1,\hdots,a_n) \in \C^n~|~f_1(a_1,\hdots,a_n)=\hdots=f_p(a_1,\hdots,a_n)=0\right\} \subset \C^n
\end{align*}
is the set of common zeros of $F$. We denote by $\langle F \rangle$ the ideal generated by $F$ in $\C[X_1,\hdots,X_n]$. Given an algebraic set $V \subset \C^n$, the set of polynomials in $\C[X_1,\hdots, X_n]$ that vanish at all points of $V$ is called the \textit{ideal of $V$,} and which we denote by
\[
I(V) = \left\{f \in \C[X_1,\hdots,X_n]~|~(a_1,\hdots,a_n) \in X \Rightarrow f(a_1,\hdots,a_n)=0\right\}. 
\]





%%%%%%%%%%%%%%%%%%%%%%%%%%%%%%%%%%%%%%%%%%%%%%%%%%%%%%%%%%%%
%%%%%%%%%%%%%%%%%%%%%%%%%%%%%%%%%%%%%%%%%%%%%%%%%%%%%%%%%%%%
%%%%%%%%%%%%%%%%%%%%%%%%%%%%%%%%%%%%%%%%%%%%%%%%%%%%%%%%%%%%
\subsubsection{Irreducibility}
An algebraic set $V \subset \C^n$ is \textit{irreducible} when $V = V_1 \cup V_2$ implies $V=V_1$  or $V=V_2$, for any $V_1,V_2 \subset V$. An algebraic set $V \subset \C^n$ can be uniquely decomposed into a finite union of irreducible algebraic sets:
\[
V = V_1 \cup V_2 \cup \hdots V_r,
\]
where $V_j, 1\leq j \leq r,$ are called the irreducible
components of $V$.
%%%%%%%%%%%%%%%%%%%%%%%%%%%%%%%%%%%%%%%%%%%%%%%%%%%%%%%%%%%%
%%%%%%%%%%%%%%%%%%%%%%%%%%%%%%%%%%%%%%%%%%%%%%%%%%%%%%%%%%%%
%%%%%%%%%%%%%%%%%%%%%%%%%%%%%%%%%%%%%%%%%%%%%%%%%%%%%%%%%%%%
\subsubsection{Dimension}
The \textit{dimension} of an algebraic set $V \subset \C^n$, denoted $\dim V,$ can be defined in the following ways:
\begin{enumerate}
    \item The number of generic hyperplanes needed to intersect with $V$ to obtain a finite set. 
    \item The \textit{Krull dimension} of $\C[X_1,\hdots,X_n]/I(V)$, which is the supremum of the lengths of all chains of prime ideals
\end{enumerate}
The \textit{codimension} of $V$ is $n - \dim V$. An algebraic set is \textit{equidimensional} if each irreducible component has the same dimension.  If each component has dimension $d$ then we say the algebraic set is $d$-equidimensional.
\par
We especially care about zero dimensional algebraic sets; these sets are finite and their degree is equal to their cardinality. 


%%%%%%%%%%%%%%%%%%%%%%%%%%%%%%%%%%%%%%%%%%%%%%%%%%%%%%%%%%%%
%%%%%%%%%%%%%%%%%%%%%%%%%%%%%%%%%%%%%%%%%%%%%%%%%%%%%%%%%%%%
%%%%%%%%%%%%%%%%%%%%%%%%%%%%%%%%%%%%%%%%%%%%%%%%%%%%%%%%%%%%
\subsubsection{Degree}
The \textit{degree} of an irreducible algebraic set is the number of intersection points between itself and $\dim (V)$ generic hyperplanes. And the degree of an arbitrary algebraic set is defined as the sum of the degrees of its irreducible components.
%
\begin{ex}
An algebraic set of dimension zero is a finite set, with degree equal to its cardinality (as mentioned above).
\end{ex}
%
\begin{ex}
An algebraic set of dimension 1 is a curve, with degree equal to the number of intersection points with a generic hyperplane.
\end{ex}
%
We will often apply \label{Bezout} B\'ezout’s bound \cite[Theorem 1]{H}, which tells us the following. Considering a sequence of polynomials $F = (f_1,\hdots,f_p) \in \K[X_1,\hdots,X_n]^p,$ with $\K$ a field, if each polynomial in the sequence has degree at most $D,$ then the algebraic set $V(F)$ has degree at most $D^p$. 
%%%%%%%%%%%%%%%%%%%%%%%%%%%%%%%%%%%%%%%%%%%%%%%%%%%%%%%%%%%%
%%%%%%%%%%%%%%%%%%%%%%%%%%%%%%%%%%%%%%%%%%%%%%%%%%%%%%%%%%%%
%%%%%%%%%%%%%%%%%%%%%%%%%%%%%%%%%%%%%%%%%%%%%%%%%%%%%%%%%%%%
\subsubsection{Noether position}
For $i \in
\{1,\hdots,n-p+1\},$  let $\pi_i$ denote the projection 
%
\begin{align*}
\C^n~~~~~ &\rightarrow~~~~~ \C^i \\
(x_1,\hdots,x_n) &\mapsto  (x_1,\hdots,x_i).    
\end{align*} 
A $d$-equidimensional algebraic set $Y \subset \C^n$ is in \textit{Noether position} for the projection
$\pi_d$ when the extension \[\C[X_1,\hdots,X_{d}] \rightarrow
\C[X_1,\hdots,X_n]/I(Y)\] is integral; here, $I(Y) \subset
\C[X_1,\hdots,X_n]$ is the defining ideal of $Y$. It is then a consequence that, for any $\xb \in \C^d,$ the fiber $Y \cap \pi_d^{-1}(\xb)$ has dimension zero and is thus finite and not empty.


%%%%%%%%%%%%%%%%%%%%%%%%%%%%%%%%%%%%%%%%%%%%%%%%%%%%%%%%%%%%%%%
%%%%%%%%%%%%%%%%%%%%%%%%%%%%%%%%%%%%%%%%%%%%%%%%%%%%%%%%%%%%%%%
%%%%%%%%%%%%%%%%%%%%%%%%%%%%%%%%%%%%%%%%%%%%%%%%%%%%%%%%%%%%%%%
\subsection{The Zariski-tangent space and regular / singular points}
Assume that $V \subset \C^n$ is a $d$-equidimensional algebraic set. Let $\grad_{\xb}(f)$ be the evaluation of the gradient vector of $f \in \C[X_1,\hdots,X_n]$ at $\xb \in \C^n$. The \textit{Zariski-tangent space} to $V$ at $\xb \in V$ is the vector space $T_{\xb}V$ defined by the equations 
\[
\grad_{\xb} (g) \cdot \bm v = 0,
\] 
for all polynomials $g$ that vanish on $V$ (i.e. for all polynomials $g \in I(V)$). 
\par 
The point $\xb \in V$ is a \textit{regular point} if $\dim (T_{\xb}V) = d.$ Otherwise $\xb$ is a \textit{singular point}. We let $\reg(V)$ and $\sing(V)$ respectively denote the regular and singular points of $V$. The image of a regular point is a \textit{regular value} and the image of a singular point is a \textit{singular value}. When $\sing(V)$ is empty then we say that $V$ is \textit{smooth}.
%
\par 
The following is a direct consequence of \cite[Corollary 16.20]{ECA}
%
\begin{prop}
If $V \subset \C^n$ is a $d$-equidimensional algebraic set with ideal $I(V)$ generated by polynomials 
\[
G=(g_1,\hdots,g_p) \in \C[X_1,\hdots,X_n]^p,
\]
then at any point $\xb$ of $\reg(V),$ $\jac_{\xb}(G)$ has full rank $n - \dim(V)$ and the kernel of $\jac_{\xb}(G)$ is equal to $T_{\xb}V.$ 
\end{prop}
%



%%%%%%%%%%%%%%%%%%%%%%%%%%%%%%%%%%%%%%%%%%%%%%%%%%%%%%%%%%%%
%%%%%%%%%%%%%%%%%%%%%%%%%%%%%%%%%%%%%%%%%%%%%%%%%%%%%%%%%%%%
%%%%%%%%%%%%%%%%%%%%%%%%%%%%%%%%%%%%%%%%%%%%%%%%%%%%%%%%%%%%
%\subsection{Critical points and polar varieties} 
%
%
%
%For $i \in
%\{1,\hdots,\td\},$ denote by $\pi_i$ the projection
%\begin{align*}
% \C^n ~~~~~&\rightarrow~~~~~ \C^i    \\
%%(x_1,\hdots,x_n) &\mapsto (x_1,\hdots,x_i).
%\end{align*}
%The $i$-th
%\textit{polar variety} 
%\[
%\Wi := \{\xb \in V~|~\dim \pi_i(T_\xb
%V) < i\}
%\]
%is the set of critical points of $\pi_i$ on $V$. Let $J_i$ denote the truncated Jacobian matrix of $\fb=(f_1,\hdots,f_p)$ with respect to $\Xb:$
%\[
%J_i = 
%\left[ 
%\begin{array}{ccc}
%\frac{\pa f_1}{\pa X_{i+1}}&\hdots& \frac{\pa f_1}{\pa X_{n}} \\
%\vdots& &\vdots\\
%\frac{\pa f_p}{\pa X_{i+1}}&\hdots& \frac{\pa f_p}{\pa X_{n}} 
%\end{array}
%\right] ,
%\]
%and let $\minors = M_{i,1},\hdots,M_{i,S_i}$ be the minors of size $p$ of %$J_i$. Notice that
%\[
%S_i = \binom{p}{p}\binom{n-i}{p}=\binom{n-i}{p}.
%\]
%Thus, $\Wi$ is defined by the vanishing of $\fb$ and $\minors.$
%For $i$ in $\{1,\dots,n-p\}~(n-p+1$?) and $\fb$ as above, we will let $\Ii$
%denote the sequence of polynomials 
%\[
%\big( \fb, \minors \right\big).
%\]
%
%%%%%%%%%%%%%%%%%%%%%%%%%%%%%%%%%%%%%%%%%%%%%%%%%%%%%%%%%%%%
%%%%%%%%%%%%%%%%%%%%%%%%%%%%%%%%%%%%%%%%%%%%%%%%%%%%%%%%%%%%
%%%%%%%%%%%%%%%%%%%%%%%%%%%%%%%%%%%%%%%%%%%%%%%%%%%%%%%%%%%%
%
\subsection{Changes of variables}
For a matrix $\mA$ in $\C^{n\times n}$ and a polynomial $g$ in
$\C[X_1,\hdots,X_n],$ we write \[g^\mA:=g(\mA \Xb) \in
\C[X_1,\dots,X_n],\] where $\Xb$ is the column vector with entries
$X_1,\dots,X_n$. And for a sequence of polynomials $G=(g_1,\hdots,g_p) \in \C[X_1,\hdots,X_n]^p$, we write 
\[
G^{\mA} = \left(g_1^{\mA},\hdots,g_p^{\mA}\right) = \left(g_1({\mA}\xb),\hdots,g_p({\mA}\xb)\right) \in \C[X_1,\hdots,X_n]^p. 
\]
For a variety $Y \subset \C^n$ and a matrix $\mA \in \GL(n),$ we  define $Y^{\mA}$ as the image of $Y$ by the map $\phi_{\mA} : \xb \mapsto \mA^{-1}\xb.$ Notice that 
\[
V(G^{\mA}) = \phi_{\mA}(V(G)) = V(G)^{\mA}. 
\]

We will also have to consider matrices with generic entries. For this,
we introduce $n^2$ new indeterminates $(\frak A_{j,k})_{1\le j,k \le
  n}$. Then, $\A$ will denote the matrix with entries $(\frak
A_{j,k})_{1\le j,k \le n}$, $\C(\A)$ will denote the rational function
field $\C((\frak A_{j,k})_{1\le j,k \le n})$ and $\C[\A]$ the
polynomial ring $\C[(\frak A_{j,k})_{1\le j,k \le n}]$.  For $g$ as
above, we will then define the polynomial $g^\A:=g(\A \Xb)$, 
which we may consider in either
$\C(\A)[X_1,\dots,X_n]$ or $\C[\A,X_1,\dots,X_n]$.




%%%%%%%%%%%%%%%%%%%%%%%%%%%%%%%%%%%%%%%%%%%%%%%%%%%%%%%%%%%%
%%%%%%%%%%%%%%%%%%%%%%%%%%%%%%%%%%%%%%%%%%%%%%%%%%%%%%%%%%%%
%%%%%%%%%%%%%%%%%%%%%%%%%%%%%%%%%%%%%%%%%%%%%%%%%%%%%%%%%%%%
\section{The algorithm}
%%%%%%%%%%%%%%%%%%%%%%%%%%%%%%%%%%%%%%%%%%%%%%%%%%%%%%%%%%%%
%%%%%%%%%%%%%%%%%%%%%%%%%%%%%%%%%%%%%%%%%%%%%%%%%%%%%%%%%%%%
%%%%%%%%%%%%%%%%%%%%%%%%%%%%%%%%%%%%%%%%%%%%%%%%%%%%%%%%%%%%
To describe the algorithms, we need to
define {\em polar varieties}. 


%%%%%%%%%%%%%%%%%%%%%%%%%%%%%%%%%%%%%%%%%%%%%%%%%%%%%%%%%%%%
%%%%%%%%%%%%%%%%%%%%%%%%%%%%%%%%%%%%%%%%%%%%%%%%%%%%%%%%%%%%
%%%%%%%%%%%%%%%%%%%%%%%%%%%%%%%%%%%%%%%%%%%%%%%%%%%%%%%%%%%%
\subsection{Critical points and polar varieties} 
Let $V=V(F)\subset \C^n$ be an equidimensional algebraic set, with $F = (f_1,\hdots,f_p)$ a sequence of polynomials in  $\C[X_1,\hdots,X_n]$. Suppose that the ideal $\langle F \rangle \subset \C[X_1,\hdots,X_n]$ is radical and that $V$ is smooth with $\dim V = n-p$.

Recall that, for $i \in
\{1,\hdots,n-p+1\},$ we denote by $\pi_i$ the projection 
%
\begin{align*}
\C^n~~~~~ &\rightarrow~~~~~ \C^i \\
(x_1,\hdots,x_n) &\mapsto  (x_1,\hdots,x_i).    
\end{align*} 
%
A \textit{critical point} is a singular point on the projection $\pi_i.$ In other words, the point $\xb \in V$ is a critical point if the dimension of $\pi_i( T_{\xb}V)$ is less than $i$.
The $i$-th
\textit{polar variety} 
\[
\Wi := \left\{\xb \in V~|~\dim \pi_i(T_\xb
V) < i\right\}
\]
is the set of critical points of $\pi_i$ on $V$. 
%


%%%%%%%%%%%%%%%%%%%%%%%%%%%%%%%%%%%%%%%%%%%%%%%%%%%%%%%%%%%%
%%%%%%%%%%%%%%%%%%%%%%%%%%%%%%%%%%%%%%%%%%%%%%%%%%%%%%%%%%%%
%%%%%%%%%%%%%%%%%%%%%%%%%%%%%%%%%%%%%%%%%%%%%%%%%%%%%%%%%%%%
\subsection{Determinantal modeling of polar varieties} 
Let $\jac(F,i)$ denote the truncated Jacobian matrix of $F=(f_1,\hdots,f_p)$ with respect to $\Xb = (X_1,\hdots,X_n):$
\[
\left[ 
\begin{array}{ccc}
\frac{\pa f_1}{\pa X_{i+1}}&\hdots& \frac{\pa f_1}{\pa X_{n}} \\
\vdots& &\vdots\\
\frac{\pa f_p}{\pa X_{i+1}}&\hdots& \frac{\pa f_p}{\pa X_{n}} 
\end{array}
\right]. 
\]
And let \[\minors = \left(M_{i,1},\hdots,M_{i,S_i}\right)\] be the minors of size $p$ of $\jac(F,i)$. Notice that
\[
S_i = \binom{p}{p}\binom{n-i}{p}=\binom{n-i}{p}.
\]
%
For each $i$ in $\{1,\dots,n-p+1\}$ and $F$ as above, we will let $\Ii$
denote the sequence of polynomials 
\[
\big( F, \minors \big) = \big(f_1,\hdots,f_p,M_{i,1},\hdots,M_{i,S_{i}}\big).
\]
%
\begin{prop}\label{prop:3.1}
Consider $F=(f_1,\hdots,f_p)$ in $\C[X_1,\hdots,X_n].$ If the ideal defined by $F$ is radical and $V = V(F)$ is smooth with $\dim V= d,$ then the polar variety \[W(\pi_i,V) = \{\xb \in V~|~\dim \pi_i(T_{\xb}V) < i\}\] is defined by $F$ and $\minors$.  
\end{prop}
%
\noindent 
To prove Proposition~\ref{prop:3.1}, we will use the following two lemmas.
%
\begin{lemma}\label{lem:3.2} 
Let $\widetilde{A} = \bbm A \\ B \ebm \in \mathbb{R}^{m\times n}$ be a matrix. Then, 
\[ 
\rank(\widetilde{A}) = 
\rank(A) + \rank(B|\ker(A)),
\]
where $B|\ker(A)$ is the restriction of the linear map defined by $B$ to the kernel of $A$. 
\end{lemma}
%
\begin{proof}
By the rank nullity theorem, we have that \begin{align*} 
\dim \ker(A) &= 
\rank(B|\ker(A)) +
\dim \ker(B| \ker(A)) \\
&= \rank(B|\ker(A)) +
\dim (\ker(B \cap \ker(A)) \\
&= \rank(B|\ker(A)) +
\dim (\ker(\widetilde{A}))
\end{align*}
Therefore, 
\[
m - \rank(A) = \rank(B|\ker(A)) + m - \rank(\At)
\]
so that 
\[
\rank(\At) = \rank(A) + \rank(B| \ker (A)).
\]
\end{proof}
%
%
\begin{lemma}\label{lem:3.3}
Consider $F=(f_1,\hdots,f_p)$ in $\C[X_1,\hdots,X_n]$ and $\xb \in V(f_1,\hdots,f_p).$ Let $J$ denote the Jacobian matrix $\jac_{\xb}( F)$ and let $T$ denote the kernal of $J.$ Assume the dimension of $T$ is equal to $d$ so that the rank of $J$ is equal to $n-d.$ Then, for $i \in \{1,\hdots,n-p+1\},$  the dimension of $\pi_i(T)$ is less than $i$ if and only if the rank of $\jac(F,i)$ is less than $n-d.$
\end{lemma} 
%
%
\begin{proof}
Let $\jt$ denote the matrix 
\[ 
\left[
\begin{array}{c|c}
\bz_{p \times i} & \jac(F,i)  \\
\hline \\
\bm I_i & \bz_{i \times n-i}  
\end{array}
\right].
\]
Now we can calculate the rank of $\jt$ in two different ways. We have 
\[
\rank(\jt) = i + \rank(\jac(F,i)).\] And, by Lemma ~\ref{lem:3.2}, 
\begin{align*}
    \rank(\jt) &= \rank(J) + \dim(\pi_i(\ker(J)))\\ 
    &= \rank(J) + \dim(\pi_iT) \\
    &= n-d + \dim \pi_i (T).
\end{align*}
Equating both we have \[
i + \rank(\jac(F,i)) = n-d + \dim\pi_i(T),
\]
which implies that 
\[
\rank(\jac(F,i)) + (i -\dim\pi_i (T) )= n-d.
\]
Therefore, the rank of $\jac(F,i)$ is less than $n-d$ if and only if the dimension of $\pi_i (T)$ is less than $i.$
\end{proof}
%
Now Proposition \ref{prop:3.1} follows from Lemma \ref{lem:3.3} and therefore the polar variety $\Wi$ is defined by the vanishing of $F$ and $\minors.$ Furthermore, this means that when $V=V(F)=V(f)$ is a hypersurface, then the polar variety $W(\pi_i,V(f))$ is defined by  
\[
V\left(f,\frac{\pa f}{\pa
  X_{i+1}},\hdots,\frac{\pa f}{\pa X_n}\right).
\] 







\begin{ex} (\cite[Example 3.1]{TWT})
Let $f = X_1^2 + X_2^2 + X_3^2-1\in \C[X_1,X_2,X_3]$ and consider the hypersurface
\[
         V=V(X_1^2 + X_2^2 + X_3^2-1) \subset \mathbb{C}^{3}.
\]
The critical points of the projection $\pi_2: (x_1,x_2,x_3) \mapsto (x_1,x_2)$ on $V(f)$ are defined by 
\[
X_1^2 + X_2^2 + X_3^2-1=X_3=0
\]
and thus the second polar variety $W(\pi_2,V(f))$ is
$V\left(f,\pa f/\pa X_3\right).$ 
\par 
Now, the critical points of the projection $\pi_1: (x_1,x_2,x_3) \mapsto (x_1)$ on $V(f)$ are defined by 
\[
X_1^2 + X_2^2 + X_3^2-1=X_2=X_3=0.
\]
and thus the first polar variety $W(\pi_1,V(f))$ is $V\left(f,\pa f/\pa X_2,\pa f/\pa X_3\right).$ 

Note the dimensions of the polar varieties: $i-1.$
\begin{figure}[h]
%\includegraphics[width=0.25\linewidth]{polarvariety.png}
\includegraphics[width=0.75\linewidth]{pi1Andpi2.png}
\centering
\caption{The polar varieties $W(0, 1, V(f))$ and $W(0, 2, V(f)),$ where \[f = X_1^2 + X_2^2 + X_3^2-1\in \C[X_1,X_2,X_3]\] \cite[Example 3.1]{TWT}.}
\end{figure}
%
\end{ex}
%



















%
%
\begin{prop}\label{prop:degreePolarV}
Let $F=(F_1,\hdots,f_p) \in \C[X_1,\hdots,X_n]^p$ be a sequence of polynomials defining a radical ideal and a smooth variety. The degree of $\Wi$ is at most $(nd)^n$.
\end{prop}
\begin{proof}
The conclusion follows from \cite[Proposition 2.3]{Heintz1980}.
\end{proof}



%%%%%%%%%%%%%%%%%%%%%%%%%%%%%%%%%%%%%%%%%%%%%%%%%%%%%%%%%%%%
%%%%%%%%%%%%%%%%%%%%%%%%%%%%%%%%%%%%%%%%%%%%%%%%%%%%%%%%%%%%
%%%%%%%%%%%%%%%%%%%%%%%%%%%%%%%%%%%%%%%%%%%%%%%%%%%%%%%%%%%%
\subsection{Lagrangian modeling of polar varieties} 
%
When the algebraic set $V=V(F)=V(f_1,\hdots,f_p)$ is defined by a sequence of polynomials with $n-i \sim 2p,$ the length of the sequence of minors $\minors$ is \[S_i = \binom{n-i}{p}\sim 2^{n-i+1}.\] 
The number of equations therefore grows exponentially and the determinantal modeling becomes impractical algorithmically. To avoid this problem, we use an alternative modeling of the polar varieties which involves introducing Lagrange variables. 
%
\begin{question}
Is this right? Or do we use the Lagrangian reformulation so that we can solve a square system? 
\end{question}
%
\begin{definition}
Let $\Lb = (L_1,\hdots,L_p)$ be new indeterminants. Then $\lagF$ denotes the entries of the vector
\[
[L_1\cdots L_p]\cdot \jac (F,i),
\] 
where the Jacobian is in the variables $(X_{i+1},\hdots,X_n).$
\end{definition}
%
\noindent 
Let $\sW(\pi_i,V)$ denote the variety
\[
V\left(F, \lagF\right) \subset \C^{n+p} 
\]
in indeterminants $\Xb=(X_1,\hdots,X_n)$ and $\Lb=(L_1,\hdots,L_p)$. We want to enforce that $\lb = (0,\hdots,0)$ is not included in any solutions; we add a linear form $\sum_{i=1}^pu_i\cdot L_i-1$, for a randomly chosen point $\ub = (u_1,\hdots,u_p) \subset \C^p.$ We let $\Wil$ denote the variety
\[
V\left(F, \lagF,\sum_{i=1}^pu_i\cdot L_i-1\right) \subset \C^{n+p},
\]
again in indeterminants $\Xb=(X_1,\hdots,X_n)$ and $\Lb=(L_1,\hdots,L_p)$. We further let $\sI(i,F)$
denote the polynomials 
\[
\left(F, \lagF \right) \in \C[\Xb,\Lb]^{p+n-i},
\]
and for $\ub = (u_1,\hdots,u_p) \in \C^p,$ we let $\Iil$ denote the polynomials
\[
\left(F, \lagF,\sum_{i=1}^pu_iL_i-1 \right)\in \C[\Xb,\Lb]^{p+n-i+1}.
\]
%
%
%
\begin{ex}
Consider again $f=X_1^2 + X_2^2 + X_3^2-1 \in \C[X_1,X_2,X_3]$ and 
\[V(X_1^2 + X_2^2 + X_3^2-1)\subset \C^3.\]
Since $\jac(X_1^2 + X_2^2 + X_3^2-1,2)=2X_3,$ the Lagrangian modeling gives us 
\[
 V(X_1^2 + X_2^2 + X_3^2-1, LX_3, L-1) = V(X_1^2 + X_2^2-1,X_3,L-1),
\]
where the equations on the right hand side are a lexicographically ordered Gr\"obner basis of the ideal $\la X_1^2 + X_2^2 + X_3^2-1, LX_3, L-1 \ra.$ 
Notice that we have only 1 Lagrange multiplier, $L$,  and we have $\ub=(u)=(1)$, but any $u \not = 0$ would suffice here. We therefore have that \[ \pi_{\Xb}(V(X_1^2 + X_2^2-1,X_3,L-1))\] describes the set where 
\[ X_1^2 + X_2^2 + X_3^2-1 = X_3 = 0\] and therefore describes the polar variety $W(\pi_2,V(f)).$
%
\end{ex}
%
%
%
\vspace{2mm}
\begin{prop}
The degree of $\Wil$ is at most $d^{\hspace{.5mm}p+n}.$
\end{prop}
\begin{proof}
Let $g_1,\hdots,g_{n-i}$ in $\C[\Xb,\Lb]$ denote the \textit{Lagrange polynomials}:
\[
\lagF = [L_1 \cdots L_p]\cdot \jac (F,i).
\]
Since each polynomial in the sequence $F=(f_1,\hdots,f_p)$ has degree at most $d$, the degrees of each $g_i$ are also bound by $d.$  Thus, after adding the linear form $\sum_{i=1}^pu_iL_i-1,$ by B\'ezout's bound, we obtain
\[
\deg \Wil \leq d^{\hspace{.5mm}p+n-i+1} \leq d^{\hspace{.5mm}p+n}. 
\]
\end{proof}
%







%%%%%%%%%%%%%%%%%%%%%%%%%%%%%%%%%%%%%%%%%%%%%%%%%%%%%%%%%%%%
%%%%%%%%%%%%%%%%%%%%%%%%%%%%%%%%%%%%%%%%%%%%%%%%%%%%%%%%%%%%
%%%%%%%%%%%%%%%%%%%%%%%%%%%%%%%%%%%%%%%%%%%%%%%%%%%%%%%%%%%%
\subsection{Genericity Statements}\label{sec:gen}
Consider a sequence of polynomials $F= (f_1,\hdots,f_p) \in \ZZ[X_1,\hdots,X_n]^p,$ with degrees at most $d$, defining a radical ideal $\langle F \rangle \subset \C[X_1,\hdots,X_n]$ and a smooth variety $V=V(F) \subset \C^n$ with $\dim V = n-p$. %
%
%

\paragraph*{Establishing correctness for Lagrange systems.}
%
We need to prove that for a carefully chosen $\ub \in \C^p,$ we have the inclusion 
\[\Iil \cap \C[X_1,\hdots,X_n] \subset \Iir.\] For $i \in \{1,\hdots,n-p+1\},$ let $Z$ be an irreducible component of the polar variety $\Wi$.
\begin{lemma}\label{lem:ratF}
There exists a non-zero vector of rational functions
\[\lb = (l_1,\hdots,l_p)=\left(\frac{N_1}{D_1},\hdots,\frac{N_p}{D_p}\right)\in \C(Z)^p-\{\bz\},\] 
and an open and dense subset $Z^{'} \subset Z$ defined by
\begin{enumerate}
    \item $D_1(\xb)\cdot D_2(\xb) \cdots D_p(\xb) \not = 0$,
    \item $N_j(\xb) \not = \bz,$ for some $j \in \{1,\hdots,p\}$, 
\end{enumerate}
with $\lb(\xb) \cdot \ji = \bz,$  when $\xb \in Z^{'}$.
%
%
\end{lemma}
%
\begin{proof}
Recall that for $\xb \in Z,$ the rank of $\ji$ is at most $p-1$, because $\Wi$ is defined by the vanishing of $F$ and $\minors$. Now,  at $\xb \in Z,$ consider the Jacobian matrix $\jac_{\xb}(F,i)$  with entries taken modulo the ideal $I(Z)$. Consider these entries as elements in $\C(Z)$ and note that since $Z$ is irreducible, the function field $\C(Z)$ is well defined; let $\jt$ denote this matrix.  Since the $p$-minors of $\ji$ are zero over $\xb \in Z,$ the $p$-minors of $\jt$ are also zero at $\xb \in Z.$ And therefore the rank of $\jt$ is also at most $p-1$ at $\xb \in Z.$ Thus, some well defined 
\[\lb = (l_1,\hdots,l_p)=\left(\frac{N_1}{D_1},\hdots,\frac{N_p}{D_p}\right)\in \C(Z)^p-\{\bz\},\]
\noindent 
exists with rational entries in $\xb$, with $\lb({\xb}) \cdot \jt = \bz$, and therefore with $\lb({\xb}) \cdot \ji = \bz$ as well. Since $\lb \not = \bz$ and since $\lb$ is well defined, $D_1(\xb)\cdot D_2(\xb) \cdots D_p(\xb) \not = 0$ and $N_j(\xb) \not = 0,$ for some $j \in \{1,\hdots,p\}$. 
\end{proof}
%

\noindent
Using the same notation from the lemma above, let $K$ denote the polynomial 
\begin{equation}\label{eq:K}
K := D_1 \cdots D_p N_j \in \C[X_1,\hdots,X_n] - \{\bz\}.
\end{equation}
%
\begin{prop}\label{prop:Kdegree}
The degree of $K$ is at most $2n^2d.$
\end{prop}
\begin{proof}
The degrees of each denominator $D_l$ and numerator $N_k,  1 \leq l,k \leq p,$ are bound by the degrees of the polynomials in the sequence $\minors$, which are at most $nd.$ Therefore the degree of $K$ is at most \[(p+1)nd \leq (n+1)nd \leq 2n^2d.\] 
\end{proof}
%
\noindent 
Let $\lb_Z$ be the vector of rational functions corresponding to the irreducible component $Z$. Consider the open set
\[
\mathscr{O}_{Z} := 
\left\{\ub \in \C^p~|~\ub^T\cdot \lb_{Z} \not = \textit{\textbf{0}}\right\}.
\]
Clearly $\mathscr{O}_Z$ is not empty because $\lb_{Z} \not = \bz$.  Let $\sO$ denote the intersection
\[
\mathscr{O} := 
\bigcap_Z \mathscr{O}_{Z},
\]
which we also know to be open and non-empty, because finite intersection of non-empty open sets are non-empty.

\begin{prop}\label{prop:polarVs1}
For $\ub \in \mathscr{O},$ we have the inclusion 
\[
\Iil \cap \C[X_1,\hdots,X_n] \subset \Iir.
\]
\end{prop}
%
%
\noindent 
In proving Proposition \ref{prop:polarVs1}, we will need the following two lemmas.
%\begin{lemma}\label{lem:3.8}
%For all irreducible components $Z$ of $\Wi$ and $\ub \in \mathscr{O},$ there exists $Z^o \subset Z,$ Zariski open and dense in $Z$, such that for all $\xb \in Z^o, \ub^T\cdot \lb_{\xb} \not = \bz.$ (Now this istrivial?)
%\end{lemma}
\begin{lemma}\label{lem:3.8}
Take any $\ub \in \mathscr{O}.$ For all irreducible components $Z$ of $\Wi$, and for $Z^{'} \subset Z$ as defined in Lemma \ref{lem:ratF}, there exists a Zariski open and dense subset $Z^o \subset Z^{'},$ where if  $\xb \in Z^o$ then $\ub^T\cdot \lb_{Z}(\xb) \not = \bz.$ 
\end{lemma}
%
\begin{proof}
For $\xb \in Z$, let $\psi$ denote the mapping 
\begin{align*}
Z  &\rightarrow \C(Z)\\    
    \xb &\mapsto \ub^T \cdot \lb_{Z}(\xb).    
\end{align*}
 By Lemma \ref{lem:ratF}, we know $\psi$ is a rational function in $\xb.$ Let us write 
\[
\psi(\xb) = \sum_{i=1}^p\frac{u_iN_{i}(\xb)}{D_{i}(\xb)}
= \frac{u_1N_1(\xb)D_2(\xb)\cdots D_p(\xb) + \hdots + u_pN_p(\xb)D_1(\xb)\cdots D_{p-1}(\xb)}{D_1(\xb) \cdots D_p(\xb)}
. 
\]
Denote by $D(\xb)$ the non-zero product $D_1(\xb) \cdots D_p(\xb)$ and put  
\begin{align*}
Z^o := \{\xb \in Z~&|~D(\xb)\cdot \psi(\xb) \not = 0\} \\
= \{\xb \in Z~&|~u_1N_1(\xb)D_2(\xb)\cdots D_p(\xb) + \hdots + u_pN_p(\xb)D_1(\xb)\cdots D_{p-1}(\xb) \not = 0\}.
\end{align*}
Notice that $Z^{o}$ is not empty becuae $\ub \in \sO.$ Thus, $Z^o \not = \emptyset$, $Z$ is irreducible, and $Z^o \subset Z^{'} \subset Z$, which means that $Z^o$ is both open and dense in $Z$.  By construction we also have that if $\xb \in Z^o$ then $D({\xb})\psi(\xb) = D({\xb})(\ub^T \cdot \lb_Z(\xb))\not = 0$ so that $\ub^T \cdot \lb_{Z}(\xb)\not = 0$.
\end{proof}
%
\begin{lemma}\label{lem:3.9}
Let $\ub \in \mathscr{O}$ and consider the algebraic set $\Wil$. For all irreducible components $Z$ of $\Wi,$ we have the inclusion \[Z^o \subset \pi_{\Xb}\left(\Wil\right).\]
\end{lemma}
%
\begin{proof}
By Lemma \ref{lem:3.8}, $\xb \in Z^o$ implies that $\ub^T \cdot \lb_{Z}(\xb) \not = \bz.$ Set
\[
\bm \alpha_{\xb} = \left(\xb,\left(\frac{1}{\ub^T\cdot \lb_{Z}(\xb)}\right)\lb_{Z}(\xb)\right). 
\]
Then $F(\xb)=\bz,$
\begin{align*}
    \left(\frac{1}{\ub^T\cdot \lb_{Z}(\xb)}\right)\lb_{Z}(\xb)\cdot \ji = \left(\frac{1}{\ub^T\cdot \lb_{Z}(\xb)}\right)\cdot \bz = \bz,
\end{align*}
and
\[
\ub^T\cdot\left(\frac{1}{\ub^T\cdot \lb_{Z}(\xb)}\right)\lb_{Z}(\xb)-1 =
    \cdot\left(\frac{\ub^T\cdot\lb_{Z}(\xb)}{\ub^T\cdot \lb_{Z}(\xb)}\right)-1=1-1=0,
\]
so that $\ax \in \Wil.$ Therefore $Z^o \subset \pi_{\Xb}\left(\Wil\right).$
%
\end{proof}
%
%
%
\noindent 
Now take $P \in \Iil\cap \C[X_1,\hdots,X_n],$ and $Z$ an irreducible component of $\Wi.$ Then, for $\xb \in Z^o$ it follows from Lemma \ref{lem:3.9} that there exists $\bm \alpha \in \C^p$ with $(\xb, \bm \alpha) \in \Wil.$ Then 
\[
P(\xb, \bm \alpha) = P(\xb) = \bz.
\]
Hence, $P$ is zero on $Z^o,$ and thus $P$ is also zero on $Z$ by the definition of Zariski closure. Therefore, $P$ is zero on $\Wi,$ and thus $P$ is in $\Iir.$ This finishes the proof of Proposition \ref{prop:polarVs1}. 

\noindent
\paragraph*{Genericity statements.}
For $i \in \{1,\hdots,n-p+1\},$ we say that $F$ satisfies $\bm H_i$ if
%
%
\begin{enumerate}
\item  $W(\pi_i,V(F))$ is either empty or $(i-1)$-equidimensional.
\item For any $(\xb,\lb)$ in $\mathscr{W}(\pi_i,V(F))$, the Jacobian matrix of $\mathscr{I}(i,F)$ has full rank $p+n-i$ at $(\xb,\lb)$.  

{\em By the Jacobian Criterion~\cite[Corollary 16.20]{ECA}, this
  implies that $\mathscr{I}(i,F)$ defines a radical ideal.}
\item $W(\pi_i,V(F))$ is either empty or in Noether position for
  $\pi_{i-1}$.
\smallskip
\end{enumerate}
%
\noindent 
For $i \in \{1,\hdots,n-p+1\}.$ Let $Z_{i,1},\hdots,Z_{i,r_i}$ be the irreducible components of $W(\pi_i,V(F))$ so that
\[
W(\pi_i,V(F)) = \bigcup_{k=1}^{r_i} Z_{i,k}.
\] Assuming that $F$ satisfies $\bm H_i$, given $\bm \sigma = (\sigma_1,\hdots,\sigma_{i-1})$ 
in $\C^{i-1}$, we further say that $F$ and $\bm \sigma$ satisfy
$\textbf{H}'_i$ if
\begin{enumerate}
\item For any $(\xb,\lb)$ in $\mathscr{W}(\pi_i,V(F))$, the Jacobian of the system of polynomials  
\[
 X_1-\sigma_1,\dots,X_{i-1}-\sigma_{i-1},F, \lagF
\]
has full rank $p+n-1$. 

{\em By the Jacobian Criterion~\cite[Corollary 16.20]{ECA}, this implies that there are finitely many solutions to these equations.}

\item The point $\bm \sigma = (\sigma_1,\hdots,\sigma_{i-1})$ is not in 
\[\pi_{i-1} \left(\bigcup_{k=1}^{r_i} V(K_{Z_{i,k}}) \cap Z_{i,k}\right),\] 
where $K_{Z_{i,k}}$ are the polynomials from \ref{eq:K}.

\end{enumerate}

\noindent 
We in addition say that $\bm u$ satisfies $\textbf{H}_i^{''}$ when, given an irreducible components $Z$ of the polar variety $W(\pi_i,V),$ if $\xb \in Z$ and $l_Z(\xb) \not = (0,\hdots,0)$ then 
\[
\ub \cdot l_Z(\xb) \not = 0.
\]
\noindent 
We prove the following.  
%
\begin{theorem}\label{theo:NoetherPositionG}
  For $i=1,\dots,n-p+1$, there exists a non-zero polynomial $\D_i\in\C[\A]$ of degree at most $6n^2(2d)^{5n}$ such that if $\mA \in
  \C^{n\times n}$ does not cancel $\D_i$, then
  $F^\mA$ satisfies $\textbf{H}_i$.
\end{theorem}
%
\begin{theorem}\label{theo:sigmaG}
  For $i=1,\dots,n-p+1$, suppose that $F$ satisfies $\textbf{H}_i$, then there exists a non-zero
  polynomial $\Xi_{i} \in \C[S_1,\dots,S_{i-1}]$ of degree at most
  $3n(nd)^{3n}$ such that if $\bm \sigma \in \C^{i-1}$ does not
  cancel $\Xi_{i}$, then $F$ and $\bm \sigma$ satisfy $\textbf{H}_i'$.
\end{theorem}
%
\begin{theorem}\label{the:u}
  For $i=1,\dots,n-p+1$, suppose that $F$ satisfies $\textbf{H}_i$ and suppose that $F$ and $\bm \sigma$ satisfy $\textbf{H}_i^{'},$ there exists a non-zero
  polynomial $\Upsilon_{i} \in \C[T_1,\dots,T_{p}]$ of degree at most $(nd)^{n}$ such that if $\bm u \in \C^{p}$ does not
  cancel $\Upsilon_{i}$, then $\bm u$ satisfies $\textbf{H}_i{''}$.
\end{theorem}
%

%%%%%%%%%%%%%%%%%%%%%%%%%%%%%%%%%%%%%%%%%%%%%%%%%%%%%%%%%%%%
%%%%%%%%%%%%%%%%%%%%%%%%%%%%%%%%%%%%%%%%%%%%%%%%%%%%%%%%%%%%
%%%%%%%%%%%%%%%%%%%%%%%%%%%%%%%%%%%%%%%%%%%%%%%%%%%%%%%%%%%%
\subsection{Proof of Theorem \ref{the:u}}
Let $Z$ be an irreducible component of the polar variety $W(\pi_i,V),$ and assume that$\xb \in Z$ and $l_Z(\xb) \not = \bz.$ Then put
\[
\Upsilon_i(\ub) := \prod_{Z} \ub \cdot \lb_Z(\xb).
\]
Or should we define it as:
\[
\Upsilon_i(\ub) := \prod_{\xb} \ub \cdot \lb_Z(\xb).
\]
for all $\xb$ that is a solution of 
\begin{equation}\label{sys:minors}
X_1 - \sigma_1,\hdots,X_{i-1} - \sigma_{i-1}, F, \minors  
\end{equation}


Clearly if $\Upsilon_i(\ub) \not = 0$ then $\bm H_i^{''}$ is satisfied. Since the number of irreducible components of the polar variety is bound by the degree of the polar variety,
\[
\deg \Upsilon_i \leq \deg W(\pi_i,V) \leq (nd)^{n},
\]
by \ref{prop:degreePolarV}.



%%%%%%%%%%%%%%%%%%%%%%%%%%%%%%%%%%%%%%%%%%%%%%%%%%%%%%%%%%%%
%%%%%%%%%%%%%%%%%%%%%%%%%%%%%%%%%%%%%%%%%%%%%%%%%%%%%%%%%%%%
%%%%%%%%%%%%%%%%%%%%%%%%%%%%%%%%%%%%%%%%%%%%%%%%%%%%%%%%%%%%
\subsection{Related results in the literature}
Some related results appear in the literature. For instance, Lemma 5
in~\cite{JeSa10} or Proposition 4.5
in~\cite{SharpEstimatesForTheEffectiveN} are quantitative Noether
position statements. However, Theorems~\ref{theo:H} and \ref{theo:NoetherPositionG} do not follow
from these previous results. Indeed, those references would allow us
to quantify when $W(\pi_i,V(F))^\mA$ is in Noether position, whereas
we need to understand when $W(\pi_i,V(F^\mA))$ is. And these two sets are in general different; for instance, their dimensions may vary. 



%
%
%%%%%%%%%%%%%%%%%%%%%%%%%%%%%%%%%%%%%%%%%%%%%%%%%%%%%%%%%%%%
%%%%%%%%%%%%%%%%%%%%%%%%%%%%%%%%%%%%%%%%%%%%%%%%%%%%%%%%%%%%
%%%%%%%%%%%%%%%%%%%%%%%%%%%%%%%%%%%%%%%%%%%%%%%%%%%%%%%%%%%%




















%%%%%%%%%%%%%%%%%%%%%%%%%%%%%%%%%%%%%%%%%%%%%%%%%%%%%%%%%%%%
%%%%%%%%%%%%%%%%%%%%%%%%%%%%%%%%%%%%%%%%%%%%%%%%%%%%%%%%%%%%
%%%%%%%%%%%%%%%%%%%%%%%%%%%%%%%%%%%%%%%%%%%%%%%%%%%%%%%%%%%%
\subsection{Computing at least one point in each connected component of $V(F) \cap \R^n$} 
%
Assuming that $F$ satisfies $\bm H_i$, and assuming that $F$ and $\bm \sigma = (\sigma_1,\hdots,\sigma_{i-1}) \in \C^{i-1}$ satisfy $\bm H_i^{'}$, it suffices to solve the systems defined by
%
\begin{equation}\label{eq:syst1}
X_1-\sigma_1,\dots,X_{i-1}-\sigma_{i-1}, F^{\mA},  \minorsA  
\end{equation}
%
for $i=1,\dots,n-p+1$.  They all admit finitely many solutions, and
Theorem~2 in~\cite{EMP} proves that the union of their solution sets
contains one point on each connected component of $V \cap \R^n$. However, since the number of minors can grow exponentially in $n$ (recall that the number of minors equals $S_i \sim  \binom{n-i}{p}= 2^{n-i+1}$ when $n-i \sim 2p$)
we avoid explicitly solving the equations in \eqref{eq:syst1} and instead use the Lagrangian modeling of polar varieties. If in addition we assume that $\ub \in \C^p$ satisfies $\bm H_i^{''},$ then we can instead solve the equations
%
\begin{equation}\label{eq:syst2}
X_1-\sigma_1,\dots,X_{i-1}-\sigma_{i-1}, F^{\mA}, \lagFA ,\udl-1,
\end{equation}
%
and then compute the projections of each solution set on the $\Xb$-space. 
%
\begin{prop}\label{prop:correctness}
For $i\in\{1,\hdots,n-p+1\}$, assuming that $F$ satisfies $\bm H_i,$ $F$ and $\bm \sigma = (\sigma_1,\hdots,\sigma_{i-1}) \in \C^{i-1}$ satisfy $\bm H_i^{'},$ and assuming that $\ub = (u_1,\hdots,u_p) \in \C^p$ satisfies $\bm H_i^{''}$, if $\xb \in \C^n$ is a solution of the system 
\begin{equation}\label{sys:minors}
X_1 - \sigma_1,\hdots,X_{i-1} - \sigma_{i-1}, F, \minors  
\end{equation}
then $\xb$ is also a solution of the system 
\begin{equation}\label{sys:lag}
X_1 - \sigma_1,\hdots,X_{i-1} - \sigma_{i-1}, F,\lagF,\sum_{i=1}^p u_i L_i -1
\end{equation}
after projecting to the $\Xb$-space. 
\end{prop}
\begin{proof}
Since $\xb$ is a solution to $\ref{sys:minors}$, there exists an irreducible component $Z$ of $W(\pi_i,V)$ with $\xb \in Z$. Then, by $\bm H_i^{'}(2),$ the polynomial $K_Z$ from \ref{eq:K} will not be canceled by $\xb,$ and thus $\lb_Z(\xb) \not = 0.$ Now, by the same argument given in the proof of Lemma \ref{lem:3.9}, we have that 
\[
\xb \in \pi_{\Xb} ( \Wil).
\]
\end{proof}
\noindent 
Therefore, by computing the solutions of the equations \ref{eq:syst2} and then computing the projection on the $\Xb$-space, we will not miss any points on the polar varieties $W(\pi_i,V).$ Furthermore, in Section \ref{sec:additionalStatementsLagSysms}, we prove Proposition \ref{prop:polarVs2}, which tells us that for any $\ub \in \C^p$, assuming that $F$ satisfies $\bm H_i(1)$ and $\bm H_i(2)$, we have the inclusion 
    \[
    \sqrt{\frak{I}(i,F)} \subset \Iil.
    \]
And it therefore follows that, 
\[
\pi_{\Xb}\left(\Wil\right) \subset V\left(\sqrt{\Ii}\right) = W(\pi_i,V).
\]
This means that, if we take any point that satisfies the Lagrangian system \ref{eq:syst2}, then its image, by the projection mapping to the $\Xb$-space is in the polar variety. 
\par 
And again, to solve the equations~\eqref{eq:syst2}, we use the algorithm
in~\cite{SH}, for which a complete bit complexity analysis is
available. 



%%%%%%%%%%%%%%%%%%%%%%%%%%%%%%%%%%%%%%%%%%%%%%%%%%%%%%%%%%%%%%
%%%%%%%%%%%%%%%%%%%%%%%%%%%%%%%%%%%%%%%%%%%%%%%%%%%%%%%%%%%%%%
%%%%%%%%%%%%%%%%%%%%%%%%%%%%%%%%%%%%%%%%%%%%%%%%%%%%%%%%%%%%%%
\subsection{Further Work}
This work should be seen as a generalization of \cite{ElGiSh20}, where the analysis was done for the hypersurface case. In addition, this work should also be seen as a step toward the analysis of
further randomized algorithms in real algebraic geometry.  In particular, randomized algorithms for deciding {\em
  connectivity queries} on smooth, compact algebraic sets have been
developed in a series of papers
\cite{SchostMohabBabySteps2011,SchostMohabBabySteps2014}, and could be
revisited using the techniques introduced here. The techniques would apply to algorithms in real algebraic geometry where transversality or Noether position are required geometric properties established by a random change of coordinates.



%%%%%%%%%%%%%%%%%%%%%%%%%%%%%%%%%%%%%%%%%%%%%%%%%%%%%%%%%%%%
%%%%%%%%%%%%%%%%%%%%%%%%%%%%%%%%%%%%%%%%%%%%%%%%%%%%%%%%%%%%
%%%%%%%%%%%%%%%%%%%%%%%%%%%%%%%%%%%%%%%%%%%%%%%%%%%%%%%%%%%%
\section{Transversality}
Many of our genericity properties are consequences of weak transversality. We develop a quantitative extension of Thom’s weak transversality theorem, specialized to the particular case of transversality to a point. And we apply this result in bounding the degree of particular hypersurfices we have to avoid; by choosing changes of variables that do not cancel these hypersurfaces, we show that particular cases of transversality hold for conveniently defined polynomial mappings, thus establishing some of our generosity properties.  



%%%%%%%%%%%%%%%%%%%%%%%%%%%%%%%%%%%%%%%%%%%%%%%%%%%%%%%%%%%%
%%%%%%%%%%%%%%%%%%%%%%%%%%%%%%%%%%%%%%%%%%%%%%%%%%%%%%%%%%%%
%%%%%%%%%%%%%%%%%%%%%%%%%%%%%%%%%%%%%%%%%%%%%%%%%%%%%%%%%%%%
\subsection{Critical points of polynomials mappings}
We will need the following definitions. Consider a mapping $\bm \Psi : Y \rightarrow \C^t$ 
from a smooth algebraic set $Y$ to $\C^t$, with $t\le \dim(Y)$. A {\em
  critical point} of $\bm \Psi$ is a point $\bm y \in Y$ with the
image of the tangent space $T_{\bm y} Y$ by the differential $d_{\bm
  Y} \bm \Psi$ having dimension less than $t$. For instance, if
$Y=\C^v$, then we have $T_{\bm y} Y=\C^v$ and this condition is equivalent
to the Jacobian of $\bm \Psi$ having rank less than $t$ at $\bm y$.
{\em Critical values} are the images by $\Psi$ of critical points. And the
complement of this set are the {\em regular values}. (Notice then, a regular
value is not necessarily in the image of~$\bm\Psi$).


%%%%%%%%%%%%%%%%%%%%%%%%%%%%%%%%%%%%%%%%%%%%%%%%%%%%%%%%%%%%
%%%%%%%%%%%%%%%%%%%%%%%%%%%%%%%%%%%%%%%%%%%%%%%%%%%%%%%%%%%%
%%%%%%%%%%%%%%%%%%%%%%%%%%%%%%%%%%%%%%%%%%%%%%%%%%%%%%%%%%%%
\subsection{Sard's lemma}
Sard's lemma states that the set of critical values of a smooth
function $\R^n \to \R^m$ has measure zero. One can give ``algebraic''
versions of it, for semi-algebraic mappings $\R^n \to \R^m$ as
in~\cite[Chapter~9]{bochnak1998real}, or polynomial mappings $\C^n \to
\C^m$ as in~\cite[Chapter~3]{Mumford76}, for which the sets of
critical values are contained in strict semi-algebraic,
resp.\ algebraic sets in the codomain. Thom's weak transversality
lemma, as given for instance in~\cite{demazure2000bifurcations},
generalizes Sard's lemma. In this section, we consider a particular
case of this result (transversality to a point), and establish a
quantitative version of it; this will allow us to establish the first
item in property ${\bf H}_i$, as well as property ${\bf H}'_i$. 


%%%%%%%%%%%%%%%%%%%%%%%%%%%%%%%%%%%%%%%%%%%%%%%%%%%%%%%%%%%%
%%%%%%%%%%%%%%%%%%%%%%%%%%%%%%%%%%%%%%%%%%%%%%%%%%%%%%%%%%%%
%%%%%%%%%%%%%%%%%%%%%%%%%%%%%%%%%%%%%%%%%%%%%%%%%%%%%%%%%%%%
\subsection{Weak transversality}
Let $n,\dt,$ and $m$ be positive integers, with $m \le n$, and denote
by $\bm\Phi :\C^n \times \C^{\dt} ~ \rightarrow \C^{m}$ a mapping
defined by polynomials in $\C[\Xb,\Thetab]$, where $\Xb$,
resp.\ $\Thetab$, is a set of $n$, resp.\ $\dt$, indeterminates.  For
$\thetab$ in $\C^{\dt}$, let $\bm\Phi_{\thetab} : \C^n \rightarrow
\C^{m}$ be the induced mapping $\xb\mapsto \bm\Phi(\xb,\thetab)$.  

Thom’s weak transversality theorem tells us that, when $\bz$ is a regular value of the big mapping, $\bm \Phi$, then $\bz$ is a regular value of the induced mapping $\bm \Phi_{\bm \vt}$, for a generic $\bm \vt$. Here, we are dealing with the particular case of transversality to a point, which we have rephrased entirely in terms of regular and critical values. We have developed a quantitative version of Thom’s weak transversality, which is the following.

\begin{prop} [Weak transversality]\label{prop:weak_t}
Let $\mathscr{O} \subset \C^n$ be a Zariski open set and suppose that $\bm 0$ is a regular value of $\bm\Phi$ on $\mathscr{O} \times \C^{s}$. Then there
    exists a non-zero polynomial $\Gamma \in \C[\Thetab]$ of degree at
    most $d^{m+n}$ such that for $\thetab$ in $\C^\dt$, if
    $\Gamma(\thetab)\ne 0$, then $\bm 0$ is a regular value
    of~$\bm\Phi_{\thetab}$ on $\mathscr{O}$.
\end{prop}
\noindent 
The following simple example shows this result at work. 
%
\begin{ex}
Consider a
squarefree polynomial $f$ in $\C[X_1,X_2]$, with degree at most $d$, defining a smooth curve  $V(f)$ in
$\C^2$, and let the mapping $\bm\Phi:\C^2\times \C \to \C^2$ be
defined by $\bm\Phi(X_1,X_2,\Theta) = (f(X_1,X_2), X_1-\Theta)$. One
checks that the Jacobian of $\bm\Phi$ with respect to
$(X_1,X_2,\Theta)$ has full rank two at any point in $\bm\Phi^{-1}(\bm 0)$, which is to say that $\bz$ is a regular value of $\bm \Phi$ and therefore the assumptions of the proposition apply. We then deduce that a non-zero polynomial $\Gamma \in \C[\Theta]$ exists, with degree at most $d^{4}$ with the property that, if $\vartheta$ in $\C$ does not cancel $\Gamma$ then $\bz$ is a regular value of the induced mapping $\bm \Phi_{\bm \vt}$. That is, for all $\vartheta$ in $\C$
except a finite number, the ideal $(f(X_1,X_2), X_1-\vartheta)$ is
radical in $\C[X_1,X_2]$; equivalently, $f(\vartheta, X_2)$ is
squarefree. Now, if we take a finite subset $S \subset \C$, then for $\bm \vt \in S$ chosen randomly, independently and uniformly, it then follows by DeMillo-Lipton-Schwartz–Zippel that 
\[
\mathbb{P}[\Gamma(\bm \vt) = \bz] \leq \frac{d^4}{|S|}.
\]
We will revisit this example in Section~\ref{proof:hip} (when proving $\bm H_i^{'}$).
\end{ex}
%


%%%%%%%%%%%%%%%%%%%%%%%%%%%%%%%%%%%%%%%%%%%%%%%%%%%%%%%
%%%%%%%%%%%%%%%%%%%%%%%%%%%%%%%%%%%%%%%%%%%%%%%%%%%%%%%
%%%%%%%%%%%%%%%%%%%%%%%%%%%%%%%%%%%%%%%%%%%%%%%%%%%%%%%
\subsection{Proof of weak transversality: Proposition 4.1}
The rest of the subsection is devoted to the proof of the proposition.
The proof of \cite[Theorem B.3]{TWT} already shows the existence of
$\Gamma$; it is essentially the classical proof for smooth
mappings~\cite[Section~3.7]{demazure2000bifurcations}, written in an
algebraic context. In what follows, we revisit this proof,
establishing a bound on the degree of $\Gamma$.

As mentioned above, Thom’s weak transversality theorem is a generalization of Sard’s Lemma, which states that the set of critical values of a smooth function has measure zero. In Thom’s weak transversality, the bad parameters (for us, the changes of variables that do not give us our desired genericity properties) show up as the critical values of a smooth function. The proof begins by characterizing the critical points of this function.

%%%%%%%%%%%%%%%%%%%%%%%%%%%%%%%%%%%%%%%%%%%%%%%%%%%%%%%
%%%%%%%%%%%%%%%%%%%%%%%%%%%%%%%%%%%%%%%%%%%%%%%%%%%%%%%
%%%%%%%%%%%%%%%%%%%%%%%%%%%%%%%%%%%%%%%%%%%%%%%%%%%%%%%
\subsubsection{Characterizing the critical points}
    Put $V^{'} = \bm\Phi^{-1}(\bm 0) \cap (\sO \times \C^s)$, and let $V$ be the Zariski closure of $V^{'}.$ If $V$ is empty, there is nothing to do, since all values $\thetab$ in $\C^\dt$ satisfy the conclusion of the proposition. We therefore assume that $V$ is not empty. Take $(\xb, \vt)$ in $V^{'}$; then by assumption, $\jac_{(\xb,\vt)}({\bm \Phi})$ has full rank $m$. Since in a neighborhood of $(\xb,\vt), V$ coincides with ${\bm \Phi}^{-1}(0) \cap \sO,$ the Jacobian criterion ~\cite[Corollary 16.20]{ECA} implies that there is a unique irreducible component $V_{(\xb,\vt)}$ of $V$ that contains $(\xb,\vt),$ that $(\xb,\vt)$ is regular on this component, that $\dim V_{(\xb,\vt)}=n+s-m$ and that $T_{(\xb,\vt)}$ is the nullspace of $\jac_{(\xb,\vt)}({\bm \Phi}).$
    
    Since every irreducible component of $V$ intersects $V^{'}$, this implies that $V$ itself is equidimensional of dimension $n+s-m$, and thus that $V^{'}$ is contained in $\reg (V)$. Furthermore, it also follows that for $(\xb,\vt) \in V^{'}, T_{(\xb,\vt)}V$ is the nullspace of $\jac_{(\xb,\vt)}(\Phi)$ in $\C^n \times \C^s.$  
    
    We will reuse the following fact, proved in~\cite{TWT}.  Consider the
    projection 
    %
    \begin{align*}
    \pi: \C^{n+\dt} &\rightarrow\C^{\dt} \\
    (\xb, \thetab)&\mapsto\thetab, 
    \end{align*}
    %
    and let $Z$ be the set of critical points of
    $\pi_{|V^{'}}$. That is,  \[Z := \{(\xb,\bm \vt) \in V^{'}~|~\dim (\pi(T_{\xb,\bm \vt}V^{'}))<s\}.\] Consider its projection $\pi(Z)$ in $\C^s$. This is
    the set of critical values of $\pi_{|V^{'}}$. Also let $Z^{'}$ be the critical points of $\pi_{|\reg(V)}$ so that, by the algebraic form
    of Sard's lemma (see \cite[Theorem~3.7]{Mumford76} for irreducible $V^{'}$
    and~\cite[Proposition~B.2]{TWT} for general $V^{'}$), its Zariski closure
    $\overline{\pi(Z^{'})}$ is a strict closed subset of $\C^s$. Then, since $V^{'} \subset \reg(V)$ and since $Z \subset Z^{'},$ the Zariski closure $\overline{\pi(Z)}$ is a strict closed subset of $\C^s$.  As we will
    see below, if $\thetab \in \C^{\dt}$ is not in $\overline{\pi(Z)}$,
    then $\bm 0$ is a regular value of $\bm\Phi_{\thetab}$ on $\sO.$
    
    To describe the set $Z$ of critical points of $\pi_{|V^{'}}$, let $\bm M$
    denote the $(s+m) \times (s+n)$ Jacobian matrix with entries in
    $\C[\Xb,\Thetab]$ given by $\bm M = \jac_{\Xb,\Thetab}(\pi,\bm\Phi)$,
    that is,
    \begin{align*}
    \bm M &= 
    \bbm 
    \jac_{\Xb,\Thetab}(\pi)\\
    \jac_{\Xb,\Thetab}(\bm\Phi) 
    \ebm 
    =
    \bbm 
    \textbf{0}_{\dt \times n}\hspace{5mm}\textbf{I}_{\dt} \\
    \jac_{\Xb,\Thetab}(\bm\Phi)
    \ebm.
    \end{align*}

    \begin{lemma} 
      For $(\xb,\thetab)$ in $V^{'}$, $(\xb,\thetab)$ is in $Z$ if and only if
      the matrix $\bm M$ has rank less than $\dt+m$ at $(\xb,\thetab)$.
    \end{lemma}
    \begin{proof}
      Take $(\xb,\thetab)$ on $V^{'}$, and let $\bm K(\xb,\thetab)$ be the
      Jacobian matrix $\jac_{\Xb,\Thetab}(\bm\Phi)$ taken at $(\xb,\thetab)$.
      Then, the rank of $\bm M(\xb,\thetab)$ can be written as
      $\textup{rank}(\bm K(\xb,\thetab)) + \textup{rank}([\textbf{0}_{\dt
          \times n}~\textbf{I}_{\dt}] \mid \ker \bm K(\xb,\thetab))$, where
      the latter is the rank of the restriction of $[\textbf{0}_{\dt
          \times n}~\textbf{I}_{\dt}]$ to the nullspace of
      $\bm K(\xb,\thetab)$.
    
      Since $(\xb,\thetab) \in V^{'}$ and since $\bz$ is a regular value of $\bm \Phi$, $\bm K(\xb,\thetab)$ has full rank ${\rm codim}
      (V) = m$. On the other hand, the nullspace of $\bm K(\xb,\thetab)$
      is the tangent space $T_{\xb,\thetab} V$, and
      $\textup{rank}([\textbf{0}_{\dt \times n}~\textbf{I}_{\dt}] \mid
      \ker \bm K(\xb,\thetab))$ is the dimension of $\pi(T_{\xb,\thetab}
      V)$.  In other words, the rank of $\bm M(\xb,\thetab)$ is equal to
      $m+\dim(\pi(T_{\xb,\thetab} V))$; this implies the claim in the
      lemma.
    \end{proof}



    Therefore, we can characterize the set $Z$ of critical points of
    $\pi_{|V}$ as those points satisfying $\bm\Phi(\xb,\thetab)=\bm 0$ and
    where all minors of $\bm M$ of order $\dt+m$ vanish. We can actually
    describe this set using a smaller matrix, by discarding certain minors
    that are identically zero. Let indeed $\bm J$ denote the $m \times n$
    submatrix of the Jacobian of $\bm\Phi$ consisting of the first $n$
    columns. This is the Jacobian matrix of $\bm \Phi$ with respect to $\Xb$.
    \begin{lemma} 
      For $(\xb,\thetab)$ in $V^{'}$, $(\xb,\thetab)$ is in $Z$ if and only if
      $\bm J(\xb,\thetab)$ has rank less than $m$.
    \end{lemma}
    \begin{proof}
    Notice 
    %
    \begin{align*}
    M(\xb,\thetab)&= 
    \bbm 
    \textbf{0}_{\dt \times n} &\textbf{I}_{\dt} \\
    \bm J(\xb,\thetab)     &\bm J'(\xb,\thetab)
    \ebm,
    \end{align*}
    where $\bm J'$ consists of the remaining columns of the Jacobian matrix of
    $\bm\Phi$.  Then, the rank of the former matrix is equal to the rank
    of
    \begin{align*}
    M(\xb,\thetab)&= 
    \bbm 
    \textbf{0}_{\dt \times n} &\textbf{I}_{\dt} \\
    \bm J(\xb,\thetab)     & \bm 0_{m \times s}
    \ebm,
    \end{align*}
    and the conclusion follows.
    \end{proof}
\noindent
    In particular, take $\thetab$ in $\C^\dt - \overline{\pi(Z)}$.
    Then for all $\xb$ in $\bm \Phi_{\thetab}^{-1}(\bm 0)$ on $\sO$, $(\xb,\thetab)$
    is in $V^{'}$, so it is not in $Z$. The previous lemma then implies that
    the Jacobian matrix $\bm J$ of $\bm\Phi_{\thetab}$ has full rank $m$ at
    $(\xb,\thetab)$. In other words, $\bm 0$ is a regular value of $\bm
    \Phi_{\thetab}$, as claimed.
    
    
    
    %%%%%%%%%%%%%%%%%%%%%%%%%%%%%%%%%%%%%%%%%%%%%%%%%%%%%%%
    %%%%%%%%%%%%%%%%%%%%%%%%%%%%%%%%%%%%%%%%%%%%%%%%%%%%%%%
    %%%%%%%%%%%%%%%%%%%%%%%%%%%%%%%%%%%%%%%%%%%%%%%%%%%%%%%
    \subsubsection{Bounding the degree of the set of critical values}
    %
    Our next step is to bound the degree of the critical points $Z$, from which we can bound the degree of $\overline{\pi(Z)},$ the Zariski closure of the critical values, and prove Proposition 4.1. To obtain an estimate on the degree of $Z$, rather than
    considering minors of $\bm J$, we will rewrite the condition that $\bm
    J(\xb,\thetab)$ has rank less than $m$ as the existence of a
    non-trivial left kernel element. For this, we once again introduce Lagrangian systems. We let $\textit{\textbf{L}}=(L_1,\hdots,L_m)$ be new
    variables, thought of as Lagrange multipliers, and consider the
    ``Lagrange polynomials'' $\mathscr{L}_1,\dots,\mathscr{L}_{n}$, with
    \[
    [\mathscr{L}_1 ,\dots, \mathscr{L}_{n}]=\textit{\textbf{L}}\cdot \bm J(\xb,\thetab).
    \] 
    Denote by $\frak{Z} \subset \C^{n+\dt+m}$ the algebraic set defined by
    the vanishing of $\mathscr{L}_1,\hdots,\mathscr{L}_{n},$ and
    $\bm\Phi$, and by $\frak{Z}'$ the algebraic set
    \[
    \frak{Z}' := \overline{\frak{Z} \cap \sO \times \C^s \times \C^n - \{(\xb,\thetab,0,\hdots,0) \in \C^{n+\dt+m}~|~(\xb,\thetab,0\hdots,0) \in \frak{Z}\}},
    \]
    where the bar denotes Zariski closure (we have to remove such points,
    since $L_1=\cdots=L_m=0$ is always a trivial solution to the Lagrange
    equations). Finally, consider the projection
    \begin{align*} 
    \mu :~ \C^{n+\dt+m} &\rightarrow \C^{n+\dt}\\
    (\xb,\thetab,\bm \ell)~ &\mapsto (\xb,\thetab).
    \end{align*}
    \noindent 
    %
    \begin{lemma}
    The critical points $Z$ are contained in $\mu(\frak{Z}').$
    \end{lemma}
    %
    \begin{proof}
    Take $(\xb,\bm \vt) \in Z.$ Then, there exists $\lb \not = (0,\hdots,0) \in \C^n$ and $\xb \in \sO$, with 
    \[
    \lb \cdot \bm J(\xb,\thetab) = \bz.
    \]
    Thus, $(\xb,\bm \vt) \in \mu(\frak{Z}')$ so that $Z \subset \mu(\frak{Z}').$
    \end{proof}
    %
    \noindent
    Let $Y$ be an irreducible component of $\frak{Z}'.$ There exists an open and dense subset $Y^o \subset Y$ such that for all $(\xb,\bm \vt,\lb) \in Y^o,$ $\xb \in \sO$ and $\lb \not = \bz.$ It then follows that 
    $\mu(Y^o) \subset Z$ and $\mu(\cup_Y Y^o) \subset Z.$ Thence, 
    \begin{align}
    \overline{\mu(\cup_Y Y^o)} \subset \overline{Z}.
    \end{align}
    Now put $W := \overline{\mu(\fZp)}$.
    %
    \begin{lemma}
    The algebraic set $\overline{\pi(Z)}$ is equal to $\overline{\pi(W)}.$
    \end{lemma}
    \begin{proof}
    Since 
    \[
    \overline{\cup_Y Y^o} = \fZp,
    \]
    we get 
    \[
    W = \overline{\mu(\fZp)} = \overline{\mu(\overline{\cup_Y Y^o})}
    = \overline{\mu(\cup_Y Y^o)} \subset \overline{Z},
    \]
    where the last inclusion follows by (3) above.
    Hence, by lemma 4.5,
    \[
    Z\subset\mu(\frak{Z}')\subset \overline{\mu(\frak{Z}')} = W \subset \overline{Z}.
    \]
    Then 
    \[
    \pi(Z) \subset \pi(W) \subset \pi(\overline{Z}).
    \]
    And thus
    \[
    \overline{\pi(Z)} \subset \overline{\pi(W)} \subset \overline{\pi(\overline{Z})} =\overline{\pi(Z)}.
    \]
    Therefore,
    \[
    \overline{\pi(Z)} = \overline{\pi(W)}.
    \]
    \end{proof}
    %
    %
    %
    %
    \begin{corollary} 
      The degree of $\overline{\pi(Z)}$ is at most $d^{m+n}.$
    \end{corollary}
    \begin{proof}
      The algebraic set $\frak Z$ is defined by $m+n$ equations, all of
      them having degree at most $d$. It follows from B\'ezout's
      Theorem~\cite{H} that $\deg(\frak{Z}) \leq d^{m+n}$, and the same
      upper bound holds for $\deg(\frak{Z}')$, since it consists of
      certain irreducible components of $\frak Z$. Since degree will not increase after projection or closure, 
      \[
      \deg (W) = \deg( \overline{\mu(\fZp)}) \leq \deg(\fZp).
      \]
      And by Lemma 4.6,
      \[
      \deg(\overline{\pi(Z)})
      =
      \deg(\overline{\pi(W)})
      \leq
      \deg(W).
      \]
    \end{proof}
    \noindent
    It then suffices to take for $\Gamma$ any
    non-zero polynomial of degree at most $d^{m+n}$ that vanishes on
    $\overline{\pi(Z)}$; this proves Proposition~\ref{prop:weak_t}.


%%%%%%%%%%%%%%%%%%%%%%%%%%%%%%%%%%%%%%%%%%%%%%%%%%%%%%%%%%%%
%%%%%%%%%%%%%%%%%%%%%%%%%%%%%%%%%%%%%%%%%%%%%%%%%%%%%%%%%%%%
%%%%%%%%%%%%%%%%%%%%%%%%%%%%%%%%%%%%%%%%%%%%%%%%%%%%%%%%%%%%
\section{Applications of weak transversality: $\textbf{H}_i(1), \textbf{H}_i(2),$ and $\textbf{H}_i^{'}$}\label{sec:applications}

Let $F=(f_1,\hdots,f_p) \in \ZZ[X_1,\hdots,X_n]^p$ be a sequence of polynomials defining a radical ideal, and where the degree of each polynomial is at most $d$. Also, assume that the variety $V(F)
\subset \C^n$ is smooth. Recall that $\A$ denotes the matrix of indeterminates with entries $(\frak
A_{j,k})_{1\le j,k \le n}$ and let $\J_i(\Xb,\A)$ denote the matrix
    \[
    \bbm 
    \jac_{\xb}(F)\\
    \A_{1,1}~~ \hdots ~~\A_{1,n}\\
    \vdots\hspace{10mm}\vdots\\
     \A_{i,1}~~ \hdots ~~\A_{i,n}
     \ebm.
    \]
Consider elements $\bm a \in \C^{in}$ as vectors of length $i$ of the form $\bm a = (\bm a_1,\hdots,\bm a_i)$ with $\bm a_i \in \C^n$. Then for such an $\bm a, \J_i(\Xb,\bm a)$ is naturally defined with the indeterminates evaluated at $\bm a$. We say that $\bm a$ has rank $i$ when $\bm a$ is a sequence of linearly independent vectors. Let $\bm \Phi$ define the polynomial mapping
    %
    \begin{align*}
    \C^{n+p+i}\times \C^{i n} &\rightarrow \C^{p+n}\\
    (\xb,\bm \lambda,\bm \vt, \bm a)&
    \mapsto
    \left(F(\xb),[\lambda_1, ~\cdots~, \lambda_c, \vt_1, ~\cdots~, \vt_i ] \cdot 
    \J_i(\xb,\bm a)
    \right)
    \end{align*}
    %
    and $\bm \Phi_{\bm a}$ the induced mapping   
    %
    \begin{align*}
    \C^{n+p+i} &\rightarrow \C^{p+n}\\
    (\xb,\bm \lambda,\bm \vt)&\mapsto \bm \Phi(\xb,\bm \lambda,\bm \vt,\bm a).
    \end{align*}
    %
    Let $\sA$ be defined by the rank conditions: $\rk (\jac_{\xb}(F))=p$ and $\bm \lambda=(\lambda_1,\hdots,\lambda_p) \not = \bz.$ In \cite[Section 3.2]{BaGiHeSaSh10}, it is shown that, for any $(\xb, \bm \lambda,\bm \vt, \bm a)$ in $\sA$, the Jacobian matrix $\jac_{(\xb, \bm \lambda, \bm \vt, \bm a)}\bm \Phi$ has full rank $p + n$, which in particular holds for $(\xb, \bm \lambda, \bm \vt, \bm a)$ in $\bm \Phi^{-1}(\bz)$ and therefore $\bz$ is a regular value of $\bm \Phi.$ It therefore follows by Proposition \ref{prop:weak_t}, there exists a non-zero polynomial $\Gamma_i \in \C[\A_{1,1},\hdots,\A_{i,n}]$ of degree at most 
    %
    \[
    d^{(n+p+i)+(p+n)} \leq d^{3n+2n} = d^{5n},
    \]
    %
    such that if $\ab \in \C^{i \times n}$ does not cancel $\Gamma_i$, then $\bz$ is a regular value of $\bm \Phi_{\bm a}.$ Thus, for $(\xb, \bm \lambda,\bm \vt) \in \sA \cap \bm \Phi^{-1}_{\bm a}(\bz)$, the Jacobian matrix $\jac_{(\xb, \bm \lambda,\bm \vt)}(\bm \Phi_{\bm a})$ has full rank $p+n$.  
    
   
%%%%%%%%%%%%%%%%%%%%%%%%%%%%%%%%%%%%%%%%%%%%%%%%%%%%%%%%%%%%
%%%%%%%%%%%%%%%%%%%%%%%%%%%%%%%%%%%%%%%%%%%%%%%%%%%%%%%%%%%%
%%%%%%%%%%%%%%%%%%%%%%%%%%%%%%%%%%%%%%%%%%%%%%%%%%%%%%%%%%%%
    \subsection{Application: proof of $\textbf{H}_i(1)$}\label{Gi1}
In what follows, we fix $i$ in $\{1,\dots,n-p+1\}$, and
    we prove the following: {\em There exists a non-zero polynomial
    $\D_{i,1} \in \C[\A]$ of degree at most $nd^{5n}$ such that if $\mA
    \in \C^{n\times n}$ does not cancel $\D_{i,1}$, then $\mA$ is
    invertible and the polar variety $\WiA$ is either empty or $(i-1)$-equidimensional.}
    
    Consider $\bm A \in \C^{n \times n}$ such that the first $i$ rows of $\bm A^{-1}$ do not cancel $\gi$, and let $\bm b$ denote the first $i$ rows of $\bm A^{-1}$. Note that $\bm b$ has full rank $i$. Let $\fB \in \C(\A)-\C$ denote $\A^{-1}$ and let $\fB_1=[\fB_{1,1},\hdots,\fB_{1,n}],\hdots,\fB_n=[\fB_{n,1},\hdots,\fB_{n,n}]$ denote the rows of $\fB.$ Set  
    \[
    \Delta_{i,1} := \Gamma_i(\fB_1,\hdots,\fB_i)\cdot (\det \A)^{\deg \gi}. 
    \]
    By multiplying through by $(\det \A)^{\deg \gi},$ we cancel all denominators and thus make $\D_{i,1}$ a polynomial. 
    %
    \begin{lemma}
    The degree of $\Delta_{i,1}$ is at most $nd^{5n}.$
    \end{lemma}
    %
    \begin{proof}
        Assume that 
        \[
        \fB_{i,j}=\fN_{i,j}/\fD_{i,j},~ \fN_{i,j},\fD_{i,j} \in \C[\A]-\{\C\}.
        \]
        Then, $\deg \fN_{i,j} \leq n-1 \leq n,$ and since we have cleared all denominators $\fD_{i,j}$ by multiplying through with $(\det \A)^{\deg \gi},$ we therefore obtain 
        \[
        \deg \Delta_{i,1} \leq n\deg \gi \leq nd^{5n}.
        \]
    \end{proof}
    %
    \noindent
    Now put 
    \[
    Y(\bm a) := \left\{\xb \in V(F)~|~\rk \J_i(\Xb,\bm a) < p+i\right\}.
    \]
    %
    \begin{lemma}
    For $\mA \in \C^{n \times n}$ with $\Delta_{i,1}(\mA) \not = 0,$ 
    \[
    Y^{\mA}(\bm b) = W\left(\pi_i,V\left(F^{\mA}\right)\right).
    \]
    \end{lemma}
    % 
    \begin{proof}
    %Let $L_1, \hdots , L_p$ and $T_1, \hdots , T_i$ be new indeterminants.
    %First note that if $\bm A \in \C^{n \times n}$ satisfies
    %\[
    %\Delta_{i,1}(\bm A) = \Gamma_i(b_{1,1},\hdots,b_{i,n})\cdot (\det \bm A)^{d^{5n}}\not = 0, 
    %\]
    %so that
    %\[        
    %\gi(b_{1,1},\hdots,b_{i,n}) \not = 0,
    %\]
    %then we have that, for $(\xb, \bm \lambda, \bm\vt) \in \sA \cap \bm \Phi^{-1}_{\bm b}(\bz)$, the Jacobian of the polynomials 
    %\[
    %\left(F,[L_1\cdots L_p~ T_1\cdots T_i ] \cdot 
    %\J_i(\Xb,\bm b)
    %\right)
    %\]
    %at $(\xb, \bm \lambda, \bm\vt)$ has full rank $p + n$. 
    %We have
    %\[
    %Y^{\mA}(\bm b) = \left\{\xb \in V\left(F^{\mA}\right)~
    %|~\rk     
%\J_i(\mA \xb,\bm b)
    % < p+i\right\}.
    %\]
    Consider the identity $\jac(F^{\mA}) = \jac(F)^{\mA}\mA$ and notice that
    \[
    \J_i(\mA\xb,\bm b)
    =
    \bbm 
    \jac_{\xb}(F)^\mA\\
    \bm b
     \ebm
     =
    \bbm 
    \jac_{\xb}(F^\mA)\mA^{-1}\\
    \bbm
    \bm 1_i ~~~~~ \bm 0
    \ebm\mA^{-1} 
     \ebm
     =
     \bbm 
    \jac_{\xb}(F^\mA)\\
    \jac_{\xb}(\pi_i)
     \ebm
    \mA^{-1}
    \]
    and therefore 
    \[
    \rk \J_i(\mA\xb,\bm b)
    =
    \rk
     \bbm 
    \jac_{\xb}(F^\mA)\\
    \jac_{\xb}(\pi_i)
     \ebm.
    \]
    Furthermore, since $V$ is smooth, it follows from \cite[Corollary 16.20]{ECA} that for all $\xb$ in $V, \jac_{\xb}(F)$ has full rank $n - \dim V = n -(n-p) = p ,$ which is a property also established in virtue of the rank conditions on $\sA.$
    Therefore 
    %
    \begin{align*}
     Y^{\mA}(\bm b) &= \left\{\xb \in V(F^{\mA})~
    \big|~\rk \jac_{\xb}(F^{\mA})=p
    \textrm{ and }
    \rk     
    \bbm 
    \jac_{\xb}(^{\mA})\\
    \jac_{\xb}(\pi_i)
     \ebm
     < p+i\right\} \\
     &= W\left(\pi_i,V\left(F^A\right)\right). 
    \end{align*}
    %
    \end{proof}
    %
    %
    \begin{lemma}
    The set $\dim Y(\bm b)$ is either empty or ($i-1)$-equidimensional.
    \end{lemma}
    \begin{proof}
    The conclusion follows from \cite[Lemma B.5, Lemma B.11]{TWT}, where our choice of $\bm b$ allows us to apply B.11. 
    \end{proof}
    %
    \noindent
    Therefore, the polar variety $W(\pi_i,V(F^{\mA}))=\dim Y(\bm b)$ is either empty or ($i-1)$-equidimensional; hence, $\bm G_i(1)$ is established. 
    %
     
%
%In Section~\ref{sec:Hi2}, we will need the following by-product of
%this result: if we consider $f^\A \in \C(\frak
%A_{j,k})[X_1,\dots,X_n]$ as defined Section~\ref{sec:gen}, this
%polynomial satisfies the rank property $\textbf{H}_i(1)$.






%%%%%%%%%%%%%%%%%%%%%%%%%%%%%%%%%%%%%%%%%%%%%%%%%%%%%%%%%%%%
%%%%%%%%%%%%%%%%%%%%%%%%%%%%%%%%%%%%%%%%%%%%%%%%%%%%%%%%%%%%
%%%%%%%%%%%%%%%%%%%%%%%%%%%%%%%%%%%%%%%%%%%%%%%%%%%%%%%%%%%%
\subsection{Application: proof of $\textbf{H}_i(2)$}\label{Hi2}
Here we in addition prove: \textit{if $\mA
    \in \C^{n\times n}$ does not cancel $\D_{i,1}$, then for any $(\xb,\lb)$ in $\mathscr{W}(\pi_i,V(F))$, the Jacobian matrix of $\mathscr{I}(i,F)$ has full rank $p+n-i$ at $(\xb,\lb)$.}    
    \par 
    Take $\mA \in \C^{n \times n}$ so that $\Delta_{i,1}(\mA) \not = 0$, and again let $\bm b$ denote the first $i$ rows of $\mA^{-1}.$
    \begin{prop}\label{prop:LagIdeal}
    At any $(\xb,\lb) \in \WilAnu$, the Jacobian of the polynomials 
    \[
    \left(F^{\mA},\lagFA\right)
    \]
    has full rank $p+n-i$. 
    \end{prop}
    %
    \begin{proof}
    Again, let $L_1, \hdots , L_p$ and $T_1, \hdots , T_i$ be new indeterminants.
    Recall from the beginning of Section \ref{sec:generalCaseApplications}, that for $(\xb, \bm \lambda, \bm \vt) \in \sA \cap \bm \Phi^{-1}_{\bm b}(\bz)$, the Jacobian matrix of the polynomials 
    \[
    \left(F,[L_1 ~\cdots~ L_p~ T_1 ~\cdots~ T_i ] \cdot 
   \J_i(\Xb,\bm b)
    \right)
    \]
    has full rank $p + n$. Recall that 
    \[
    \J_i(\mA\xb,\bm b)
    =
    \bbm 
    \jac_{\xb}(F)^\mA\\
    \bm b
     \ebm
     =
    \bbm 
    \jac_{\xb}(F^\mA)\mA^{-1}\\
    \bbm
    \bm 1_i ~~~~~ \bm 0
    \ebm\mA^{-1} 
     \ebm
     =
     \bbm 
    \jac_{\xb}(F^\mA)\\
    \jac_{\xb}(\pi_i)
     \ebm
    \mA^{-1}
    \]
    and therefore 
    \[
    \rk \J_i(\mA\xb,\bm b)
    =
    \rk
     \bbm 
    \jac_{\xb}(F^\mA)\\
    \jac_{\xb}(\pi_i)
     \ebm.
    \] 
    Consider the Jacobian of the polynomials
    \[
    \left(F,[L_1 ~\cdots~ L_p~ T_1 ~\cdots~ T_i ] \cdot 
   \J_i(\Xb,\bm b)
    \right)
    \]    
    taken with respect to the variables
     \[
    X_1,\dots,X_n,L_1,\hdots,L_p,T_1,\hdots,T_i,
    \]
    at a point $(\xb,\bm \lambda,\bm \vt)$ that cancels equations. This Jacobian has full rank $p + n.$ Indeed,
    %
    it is equal to 
    \begin{align*}
    \left[ 
    \begin{array}{c|c|c}
    \jac_{\xb}(F^{\mA}) & \bz_{p\times p} & \bz_{p\times i}\\
    \ast \ast \ast & \ast \ast \ast & \bm I_{i}\\
    \ast \ast \ast & \ast \ast \ast & \bz_{p\times i}
    \end{array}
    \right]
    &=
    \left[ 
    \begin{array}{c|c}
    \jac_{(\xb,\lb)}(F^{\mA}) & \bz_{p\times i}\\
    \jac_{(\xb,\lb)}\left([\bm \lambda, \bm \vt ] \cdot 
    \bbm 
    \jac_{\xb}(F^{\mA})\\
    \jac_{\xb}(\pi_i)\\
    \ebm\right) & \bbm \bm I_{i}\\ \bz_{p\times i} \ebm\\
    \end{array}
    \right]
    \\
    &=
    \left[ 
    \begin{array}{c|c}
    \jac_{(\xb,\lb)}(F^{\mA}) & \bz_{p\times i} \\
    \ast \ast \ast & I_{i}\\
    \jac_{(\xb,\lb)}\left(\bm l \cdot \jac_{\xb}(F^{\mA},i) \right)& \bz_{p\times i}
    \end{array}
    \right].
    \end{align*}
    %
    Therefore, after rearranging blocks and after removing $i$ columns, we can see that 
    %
    \begin{align}
    \jac_{(\xb, \lb)}\left( F^{\mA},\bm l \cdot \jac_{\xb}(F^{\mA},i)\right) =
    \left[ 
    \begin{array}{c}
    \jac_{(\xb,\lb)}(F^{\mA}) \\
    \jac_{(\xb,\lb)}\left(\bm l \cdot \jac_{\xb}(F^{\mA},i) \right) 
    \end{array}
    \right]
    \end{align}
    has full rank $p + n-i$. Now, recall that
    \[
    \bm l \cdot \jac_{\xb}(F^{\mA},i) = \lagFA,
    \]
    and therefore it becomes clear that, at any 
    \[(\xb,\lb) \in V(\IilAnu)=\WilAnu,\]
    the Jacobian of the polynomials 
    \[
    \left(F^{\mA},\lagFA\right)
    \]
    has full rank $p+n-i$. 
    \end{proof}
    \noindent 
    This establishes $\bm G_i(2).$ 
    
    %%%%%%%%%%%%%%%%%%%%%%%%%%%%%%%%%%%%%%%%%%%%%%%%%
    %%%%%%%%%%%%%%%%%%%%%%%%%%%%%%%%%%%%%%%%%%%%%%%%%
    \subsection{Additional statements for Lagrangian systems}\label{sec:additionalStatementsLagSysms}

    Again assume that $\mA \in \C^{n \times n}$ has the property that $\Delta_{i,1}(\mA) \not = 0.$    
    %
    \begin{corollary} (Do we use this?)
     The ideal defined by $\sI(i,F^{\mA}) = \left(F^{\mA},\lagFA\right)$ is radical.
    \end{corollary}
    %
    \begin{proof}
    Given Proposition \ref{prop:LagIdeal}, the claim now follows from the Jacobian Criterion~\cite[Corollary 16.20]{ECA}.
    \end{proof}
    %
    \begin{prop}\label{prop:RadLagPolarV}
     Let $\ub = (u_1,\hdots,u_p) \in \C^p$ be any complex point. Then, for any \[(\xb,\lb)\in \mathscr{W}_{\ub}\left(\pi_i,V(F^{\mA})\right) \subset \C^{n+p},\] the Jacobian matrix of the polynomials
     \[
    \left( F^{\mA},\lagFA,\udl-1\right) 
    \]
 has full rank $p+n-i+1$ at $(\xb,\lb).$ 
    \end{prop}
    \begin{proof}
    Note that $(\xb,\lb)\in\mathscr{W}_{\ub}(\pi_i,V(F^{\mA}))$ implies that $(\xb,\lb)\in\mathscr{W}(\pi_i,V(F^{\mA}))$; thus, by Proposition \ref{prop:LagIdeal}, the Jacobian of the polynomials \[\sI(i,F^{\mA}) = (F^{\mA},\lagFA)\] has full rank $p+n-i$ at $(\xb,\lb)$. The conclusion therefore holds if $[\bz~|~\ub]$ is not in the row space of the Jacobian of the polynomials
    \[
    \left(F^{\mA},\lagFA,\udl -1\right),
    \]
    for any $(\xb,\bm l)$ that cancels equations. This matrix is equal to
    \[
    \left[ 
    \begin{array}{c|c|c}
    \frac{\pa f_1^{\mA}}{\pa X_1}(\xb) \hdots \frac{\pa f_1^{\mA}}{\pa X_n}(\xb) & \bz_{1\times p} & \bz_{1\times p}\\
    \ddots & \ddots & \ddots\\
    \frac{\pa f_p^{\mA}}{\pa X_1}(\xb) \hdots \frac{\pa f_p^{\mA}}{\pa X_n}(\xb) & \bz_{1\times p} & \bz_{1\times p}\\
    \ddots &\jac_{\xb}(F^{\mA},i)^T & \bz_{n-i+1\times p}\\
    \bz_{1 \times p} & u_1 \hdots u_p & l_1 \hdots l_p 
    \end{array}
    \right];
    \]
    consider the upper left block
    \[
    \left[ 
    \begin{array}{c|c}
    A & \bz_{p\times p} \\
    B & C 
    \end{array}
    \right]
    :=
    \left[ 
    \begin{array}{c|c}
    \frac{\pa f_1^{\mA}}{\pa X_1}(\xb) \hdots \frac{\pa f_1^{\mA}}{\pa X_n}(\xb) & \bz_{1\times p} \\
    \ddots & \ddots \\
    \frac{\pa f_p^{\mA}}{\pa X_1}(\xb) \hdots \frac{\pa f_p^{\mA}}{\pa X_n}(\xb) & \bz_{1\times p} \\
    \ddots &\jac_{\xb}(F^{\mA},i)^T
    \end{array}
    \right],
    \]
    and suppose for contradiction that $[\bz~|~\ub]$ is in the row-space. Then 
    \[
    [\bz~|~\ub]
    =
    \lambda [A~|\bz] + \mu[B~|~C]
    \]
    and 
    \[
    \ub = \mu \cdot C = \mu \cdot \jac_{\xb} (F^{\mA},i)^T
    \]
    so that 
    \[
    \ub^T = \jac_{\xb} (F^{\mA},i) \cdot \mu^T.
    \]
    Now we have a contradiction because, $(\xb,\lb)$ is such that 
    \begin{align*}
        &\lb \cdot \jac_{\xb} (F^{\mA},i)=0\\
        \Rightarrow
        ~&\lb \cdot \jac_{\xb} (F^{\mA},i) \mu^T= 0\\
        \Rightarrow
        ~&~\lb \cdot \ub^T = 0,
    \end{align*}
    when by assumption $\lb \cdot \ub^T = 1.$ 
    \end{proof}
    %
    %
    \begin{corollary}\label{cor:LagIdealRadical}
Let $\ub = (u_1,\hdots,u_p) \in \C^p$ be any complex point. Then, for any \[(\xb,\lb)\in \mathscr{W}_{\ub}(\pi_i,V(F^{\mA})) \subset \C^{n+p}\] the ideal defined by $\sI_{\ub}(i,F^{\mA}) :$\[ \left(F^{\mA},\lagFA,\sum_{i=1}^p u_iL_i-1\right)\] is radical.
    \end{corollary}
    \begin{proof}
    Given Proposition \ref{prop:RadLagPolarV}, the conclusion now follows from the Jacobian Criterion~\cite[Corollary 16.20]{ECA}.
    \end{proof}
    %
    %
    \begin{prop}\label{prop:polarVs2}
Let $\ub = (u_1,\hdots,u_p) \in \C^p$ be any complex point. Then, for any \[(\xb,\lb)\in \mathscr{W}_{\ub}(\pi_i,V(F^{\mA})) \subset \C^{n+p}\]we have the inclusion
    \[
    \sqrt{\frak{I}(i,F^{\mA})} \subset \IilA.
    \]
    \end{prop}
    \begin{proof}
    First note that $\IiA \subset \sqrt{\IilA}.$ Indeed, let $f \in \IiA$ and \[\bm \alpha = (\xb,\lb) \in V(\IilA)=\WilA.\] Then $\sum_{i=1}^p u_i l_i = 1$, so that $\lb \not = (0,\hdots,0)$ is in the left null space and therefore the rank of $\jac_{\xb} (F,i)$ is less than $p.$ Therefore all minors are zero at $\xb$ and $f(\bm \alpha)= f(\xb,\lb)=f(\xb)=0,$ so that $f \in \sqrt{\IilA}$ and $\IiA \subset \sqrt{\IilA}.$
    \par 
    Now, since by Corollary \ref{cor:LagIdealRadical}, $\sqrt{\IilA} = \IilA$, we have the inclusion
    \[
    \sqrt{\frak{I}(i,F^{\mA})} \subset \sqrt{\IilA} = \IilA.
    \]
    \end{proof}
    %
    %


%%%%%%%%%%%%%%%%%%%%%%%%%%%%%%%%%%%%%%%%%%%%%%%%%%%%%%%%%%%%
%%%%%%%%%%%%%%%%%%%%%%%%%%%%%%%%%%%%%%%%%%%%%%%%%%%%%%%%%%%%
%%%%%%%%%%%%%%%%%%%%%%%%%%%%%%%%%%%%%%%%%%%%%%%%%%%%%%%%%%%%
\subsection{Application: proof of $\textbf{H}_i^{'}(1)$}\label{proof:hip}
Again let $F = (f_1,\hdots,f_p) \in \ZZ[X_1,\hdots,X_n]^p$  define a radical ideal and a smooth variety, and let $i\in \{1,\hdots,n-p+1\}$. We now assume
that $F$ satisfies $\textbf{H}_i$, and we prove the following: {\em
  there exists a non-zero polynomial 
  \[
  \Xi_{i,1} \in
  \C[S_1,\dots,S_{i-1}]
  \]
  of degree at most $d^{3n}$ such that if $
  \bm \sigma = (\sigma_1,\hdots,\sigma_{i-1}) \in \C^{i-1}$ does not cancel
  $\Xi_{i,1}$, then for any $(\xb,\lb) \in \sW(\pi_i,V(F))$, the Jacobian of the system of polynomials
  $$\left(X_1-\sigma_1,\dots,X_{i-1}-\sigma_{i-1},F,\lagF\right)$$ has full rank $p+n-1$.}

\smallskip

Let $\bm\Psi: \C^{n+p} \times \C^{i-1} \rightarrow \C^{n}$ be the mapping defined by the polynomials
\[
  \left(X_1-S_1,\dots,X_{i-1}-S_{i-1},F,\lagF\right).
\]
%
\begin{lemma}
  $\bz$ is a regular value of $\bm\Psi.$
\end{lemma}
\begin{proof}
At all zeros $(\xb,\lb,\bm \sigma)$ of $\bm\Psi,$ the Jacobian matrix of
$\bm\Psi$ has full rank $n+p-1$. Indeed, indexing columns 
    by 
    \[
    X_1,\dots,X_n,L_1,\hdots,L_p,S_1,\dots,S_{i-1},
    \]
    this matrix is equal to
    \[
    \left[ 
    \begin{array}{cc}
    \begin{array}{cc}
    \bI_{i-1}     & \bz_{(i-1)\times (n+p-i+1)}  
    \end{array}  &-\bI_{i-1}\\
    \jac_{(\xb,\bm l)}\left( F,\lb \cdot \jac_{\xb} (F,i)  \right) & \bz_{(p+n-i)\times (i-1)}
    \end{array}
    \right].
    \]
    Recall that by $\bm G_i(2)$, the Jacobian matrix $\jac_{(\xb,\bm l)}\left( F,\lb \cdot \jac_{\xb} (F,i) \right)$ has full rank $p+n-i$ at any zero $(\xb,\bm l)$. Hence, the entire    
    matrix must have full rank $p+n-1$. Thus,  $\bz$ is a regular value of $\bm\Psi.$
    \end{proof}

Since all polynomials defining $\bm\Psi$ have degree at most $d$, it
follows by Proposition~\ref{prop:weak_t} that there exists a non-zero
polynomial $\Xi_{i,1}$ in $\C[S_1,\dots,S_{i-1}]$ of degree at most
$d^{(n+p)+(n)}\leq d^{3n},$ with the property that, if $\Xi_{i,1}(\bm \sigma)\neq 0$ then at any root $(\xb,\lb)$ of
\[
\left(X_1-\sigma_1,\dots,X_{i-1}-\sigma_{i-1},F,\lagF\right),
\] 
  the Jacobian matrix of these
equations has full rank $n+p-1$. 

































\section{Proof of $\textbf{H}_i^{'}$}\label{Sec:Hip}
Let $F = (f_1,\hdots,f_p) \in \ZZ[X_1,\hdots,X_n]^p$  define a radical ideal and a smooth variety, and let $i\in \{1,\hdots,n-p+1\}$. 
%======================================================================
%%%%%%%%%%%%%%%%%%%%%%%%%%%%%%%%%%%%%%%%%%%%%%%%%%%%%%%%%%%%
%%%%%%%%%%%%%%%%%%%%%%%%%%%%%%%%%%%%%%%%%%%%%%%%%%%%%%%%%%%%
%%%%%%%%%%%%%%%%%%%%%%%%%%%%%%%%%%%%%%%%%%%%%%%%%%%%%%%%%%%%
\subsection{Proof of $\textbf{H}_i^{'}(2)$}\label{proof:hip2}
%
We assume that $F$ satisfies $\textbf{H}_i$, and we prove the following: {\em
  there exists a non-zero polynomial 
  \[\Xi_{i,2} \in
  \C[S_1,\dots,S_{i-1}]\] 
  of degree at most $2n(nd)^{n+1}$ such that if $
  \bm \sigma = (\sigma_1,\hdots,\sigma_{i-1}) \in \C^{i-1}$ does not cancel
  $\Xi_{i,2}$, then 
  \[\bm \sigma = (\sigma_1,\hdots,\sigma_{i-1}) \not \in \pi_{i-1} \left(\bigcup_{i=1}^r \left(V(K_{Z_i}) \cap Z_i\right)\right),\]
  where the irreducible decomposition of the polar variety is $W(\pi_i,V) = Z_1 \cup Z_2 \cup \hdots \cup Z_r$ and $K_{Z_i}$ are the polynomials from \ref{eq:K}.}

\smallskip


First note that 
\[
\sum_{i=1}^r \deg Z_i = \deg W(\pi_i,V) \leq (nd)^n,
\]
where the inequality follows from Propositions \ref{prop:degreePolarV}. By Proposition \ref{prop:Kdegree}, the degree of $K_{Z_i}$ is at most $2n^2d$, Therefore, by B\'ezout's bound,
\[
\sum_{i=1}^r \deg (V(K_{Z_i}) \cap Z_i) = (2n^2d)\sum_{i=1}^r \deg Z_i \leq (2n^2d)((nd)^n) = 2n(nd)^{n+1}.
\]

\noindent 
Now, since degree will not increase after projection or closure, the same degree bound holds for
\begin{equation}\label{hypersurface:Hip2} 
\overline{\pi_{i-1} \left(\bigcup_{i=1}^r \left(V(K_{Z_i}) \cap Z_i\right)\right)}.
\end{equation}
Now take $\Xi_{i,2} \in \C[S_1,\hdots,S_{i-1}]$ as any non-zero polynomial of degree at most $2n(nd)^{n+1}$ that vanishes on \ref{hypersurface:Hip2}. Then, since 
\[
\pi_{i-1} \left(\bigcup_{i=1}^r \left(V(K_{Z_i}) \cap Z_i\right)\right) \subset\overline{\pi_{i-1} \left(\bigcup_{i=1}^r \left(V(K_{Z_i}) \cap Z_i\right)\right)},
\]
we have that if $\Xi_{i,2}(\sigma_1,\hdots,\sigma_{i-1}) \not = 0$ then
\[
(\sigma_1,\hdots,\sigma_{i-1}) \not \in \pi_{i-1} (V(K) \cap W(\pi_i,V)),
\]
and $\bm H_i^{'}(2)$ is satisfied, provided that 
\[
\overline{\pi_{i-1} \left(\bigcup_{i=1}^r \left(V(K_{Z_i}) \cap Z_i\right)\right)} \subsetneqq \C^{i-1},
\]
which indeed is the case. By $H_i(1)$ we have know that the dimension of $Z_i$ is at most $i-1.$ Since $K_{Z_i}$ is not zero on $Z_i$, it follows that 
\[
\dim \left(V(K_{Z_i}) \cap Z_i\right) \leq i-2, 
\]
and therefore 
\[
\dim \overline{\pi_{i-1} \left(\bigcup_{i=1}^r \left(V(K_{Z_i}) \cap Z_i\right)\right)} \leq i-2. 
\]
%%%%%%%%%%%%%%%%%%%%%%%%%%%%%%%%%%%%%%%%%%%%%%%%%%%%%%%%%%%%
%%%%%%%%%%%%%%%%%%%%%%%%%%%%%%%%%%%%%%%%%%%%%%%%%%%%%%%%%%%%
%%%%%%%%%%%%%%%%%%%%%%%%%%%%%%%%%%%%%%%%%%%%%%%%%%%%%%%%%%%%
\subsection{Proof of $\textbf{H}_i^{'}$}\label{proof:hipf}
Take $\Xi_{i,1}$ from Section \ref{proof:hip} and $\Xi_{i,2}$ from Section \ref{proof:hip2} and put  
\[
\Xi_i := \Xi_{i,1} \Xi_{i,2} \in \C[S_1,\hdots,S_{i-1}].
\]
The degree of $\Xi_i$ is at most 
\begin{align*}
\deg \Xi_{i,1} + \deg \Xi_{i,2} & \leq d^{3n} + 2n(nd)^{n+1}\\
& \leq (nd)^{3n} + 2n(nd)^{3n}\\
& \leq 3n(nd)^{3n}.
\end{align*}
And if we choose $\bm \sigma = (\sigma_1,\hdots, \sigma_{i-1}) \in \C^{i-1}$ with 
$\Xi_i(\bm \sigma) \not = 0$ then $\Xi_{i,1}(\bm \sigma) \not = 0$ and $\Xi_{i,2}(\bm \sigma) \not = 0$, so that, by Sections \ref{proof:hip} and \ref{proof:hip2}, respectively, $F$ and $\bm \sigma$ satisfy both $\bm G_i^{'}(1)$ and $\bm G_i^{'}(2).$  
Theorem~\ref{theo:sigmaH} is now proven.



%%%%%%%%%%%%%%%%%%%%%%%%%%%%%%%%%%%%%%%%%%%%%










































%%%%%%%%%%%%%%%%%%%%%%%%%%%%%%%%%%%%%%%%%%%%%%%%%%%%%%%%%%%%
%%%%%%%%%%%%%%%%%%%%%%%%%%%%%%%%%%%%%%%%%%%%%%%%%%%%%%%%%%%%
%%%%%%%%%%%%%%%%%%%%%%%%%%%%%%%%%%%%%%%%%%%%%%%%%%%%%%%%%%%%
\section{Proof of the Noether position property: $\textbf{H}_i(3)$}\label{ssec:Hi2}
Now let $F = (f_1,\hdots,f_p)\in\C[X_1,\hdots,X_n]^p$ define a radical ideal and a smooth variety, and fix $i \in \{1,\hdots,n-p+1\}$. We prove that there exists a non-zero polynomial $\D_{i}$ in $\C[\A]$ of degree at most $6n^2(2d)^{5n}$ such that if $\mA$ does not cancel $\D_{i}$, then $F^{\mA}$ satisfies  $\textbf{G}_i$.

Recall that we let $\Xb = (X_1, \hdots , X_n)$ be a sequence of variables, and for $l \in \{1,\hdots,n\}$ we let $\Xb_{\leq l}$ be the subsequence of variables $(X_1, \hdots , X_l)$. Consider again the $n\times n$ matrix of indeterminates
\[\A=(\A_{j,k})_{1 \le j,k \le n}\] and the field $\C(\A)$,
and define $F^{\A}=(f_1^{\A},\hdots,f_p^{\A})$  as \[(f_1(\A\Xb),\hdots,f_p(\A\Xb)) \in \C(\A)[\Xb]^p.\]  
%Since $i$ is fixed, to
%simplify notation, let $\I^\A$ denote the polynomials $\I(i,F^\A)= \big (F^\A, \minorsfA \big )$ in
%$\C(\A)[\Xb]$, and let $\Il^\A$ denote the polynomials $\IilfA= \big (F^\A, \bm L \cdot \jac_{\xb}(F^\A,i),\ub^T\cdot\Lb-1 \big )$ in
%$\C(\A)[\Xb,\Lb]$. Let $W^\A$ denote  $W(\pi_i,V(F^\A))$ and let $\Wl^\A$ denote  $\WilfA$. 




%%%%%%%%%%%%%%%%%%%%%%%%%%%%%%%%%%%%%%%%%%%%%%%%%%%%%%%%%%%%
%%%%%%%%%%%%%%%%%%%%%%%%%%%%%%%%%%%%%%%%%%%%%%%%%%%%%%%%%%%%
%%%%%%%%%%%%%%%%%%%%%%%%%%%%%%%%%%%%%%%%%%%%%%%%%%%%%%%%%%%%
\subsection{Degree bounds for the integral dependence relationship} 
%
In Section~\ref{sec:applications}, we saw that
$F^\A$ satisfies $\textbf{H}_i(1)$, so that $\WifA$ is equidimensional of dimension $i-1$. We
now point out that $F^\A$ also satisfies $\textbf{H}_i(3)$.
\begin{lemma}\label{lem:6.1}
 The extension $$\C(\A)[\Xb_{\leq i-1}]\rightarrow
 \C(\A)[\Xb]/\IifAr$$ is integral.
\end{lemma}
%
\begin{proof}
  Let $(\fp_\ell)_{1 \le \ell \leq L}$ be the prime components of $\IifAr$. By \cite[Proposition 1]{EMP}, for all
  $\ell$,
  \[
    \C(\A)[\textbf{X}_{\leq i-1}]\rightarrow\C(\A)[\Xb]/\fp_\ell
  \] 
  is integral. Therefore polynomials
  $q_{\ell,j}\in\C(\A)[\textbf{X}_{\leq i-1},X_j]$ exist, all monic in
  $X_j$, with $q_{\ell,j}(X_j)\in \fp_\ell$ for each $j$ in
  $\{i,\hdots,n-p+1\}.$ Thence, \[Q_{j} := \prod_{1 \le \ell\le L}
  q_{\ell,j}\] is monic in $X_j$ and satisfies $ Q_{j} \in \sqrt{\IifA}$, for
  each $j \in \{i,\hdots,n-p+1\}.$ This proves our claim.
\end{proof}
%
\noindent 
Now let $\ub \in \C^p$ be any complex number. 
\begin{corollary}\label{lem:6.2}
 The extension $$\C(\A)[\Xb_{\leq i-1}]\rightarrow
 \C(\A)[\Xb]/(\IilfA \cap \C(\A)[\Xb])$$ is integral.
\end{corollary}
%
\begin{proof}
By Lemma \ref{lem:6.1}, polynomials
  $P_{j}\in\C(\A)[\Xb_{\leq i-1},X_j]$ exist, all monic in
  $X_j$, with $P_{j}(X_j)\in \IifAr$ for each $j$ in
  $\{i,\hdots,n-p+1\}.$ By Proposition \ref{prop:polarVs2}, 
  \[
  \sqrt{\IifA} \subset \IilfA,
  \]
  and therefore $P_{j}(X_j)\in \IilfA$ for each $j$ in 
  $\{i,\hdots,n-p+1\}$ and 
  \[
  \C(\A)[\Xb_{\leq i-1}]\rightarrow
 \C(\A)[\Xb]/(\IilfA \cap \C(\A)[\Xb])
  \]
 is integral.
\end{proof}
%
\noindent
If $P$ is any polynomial in $\C(\frak A)[\Xb]$, we will let
$D \in \C[\frak A]$ be the minimal common denominator of all its
coefficients, and we will write $\overline P := D P$, so that
$\overline P$ is in $\C[\A,\Xb]$.


    \begin{lemma} 
    For each $j \in \{i,\dots,n-p+1\}$, there exists $P_j$ in $
    \C(\A)[\Xb_{\leq i-1},X_j]$, monic in $X_j$, with $\pjb$ in
    $\IilfA$, and such that $\deg(\pjb)\leq (2d)^{2n}.$
    \end{lemma} 

\begin{proof}
     We let $\frak L^\A$ denote the extension of $\IilfA \cap \C(\A)[\Xb]$ given by 
  \[
  \frak
  L^\A =(\IilfA \cap \C(\A)[\Xb])\cdot \C(\frak A, \Xb_{\leq i-1})[X_i,\dots,X_n].
  \]
  By Corollary \ref{lem:6.2}, $Q_j\in\C(\A)[\textbf{X}_{\leq i-1},X_j],$ exists, monic in $X_j$, with 
    \[
    Q_j(X_j) \in \IilfA \cap \C(\A)[\Xb].
    \]
  Thence,
  \begin{equation}\label{eq:3}
    \C(\frak A, \Xb_{\leq i-1}) \to \C(\frak A,
    \Xb_{\leq i-1})[X_i,\dots,X_n]/\frak L^\A
  \end{equation}
    is an algebraic extension. Let $P_j \in \C(\A)(\textbf{X}_{\leq i-1})[X_j]$ 
be the minimal polynomial of $X_j$ in \eqref{eq:3}, and note that $P_j$ is monic in $X_j.$ 
    Hence, $Q_j$ is also in the extension $\frak{L}^\A$, and thus $P_j$ divides $Q_j$ in $\C(\A)(\textbf{X}_{\leq i-1})[X_j].$ We can therefore write 
    \begin{align*}
    &Q_j = P_jR_j,~~~~ P_j,R_j \in \C(\A)(\textbf{X}_{\leq i-1})[X_j]-\C(\A)(\textbf{X}_{\leq i-1}).
    \end{align*}
    It then follows by Gauss's lemma that 
    \begin{align*}
    Q_j = p_jr_j, ~~~~p_j,r_j \in \C(\A)[\textbf{X}_{\leq i-1}][X_j]-\C(\A),
    \end{align*}
    and such that $\mu_j \in \C(\A)(\Xb_{\leq i-1})$ exists with 
    \[
    P_j = \mu_j p_j,~~~~ R_j = \mu_j^{-1}r_j.
    \]
    Since $Q_j$ is monic in $X_j$, $p_j$ and $r_j$ must also be monic in $X_j$, and $\mu_j$ must be the coefficient of the highest degree term of $P_j$ in $X_j.$ Since $P_j$ is monic in $X_j$, $\mu_j =1$ and hence \[P_j=1\cdot p_j=p_j \in \C(\A)[\textbf{X}_{\leq i-1}][X_j].\]

  Now, consider the polynomials $\IilfA:$
\[
 \left(F^{\A}, \lagFfA,\udl-1 \right)
\]
in $\C[\A,\Xb,\Lb]$, let $\frak W$ be their zero-set, and let $\deg(\frak W)$ be its
  degree, in the sense of~\cite{H}. Proposition~1 in~\cite{CGR} ($\td$ site the assumptions of this theorem)
  implies that $\pjb$ has degree at most $\deg(\frak W)$. Since all
  polynomials defining $\frak W$, seen in $\C[\A,\Xb,\Lb]$, have
  degree at most $2d$, the B\'ezout inequality of~\cite{H} gives
  \[\deg(\pjb) \le (2d)^{p+n-i+1} \le (2d)^{2n}.\]
\end{proof}





%%%%%%%%%%%%%%%%%%%%%%%%%%%%%%%%%%%%%%%%%%%%%%%%%%%%%%%%%%%%
%%%%%%%%%%%%%%%%%%%%%%%%%%%%%%%%%%%%%%%%%%%%%%%%%%%%%%%%%%%%
%%%%%%%%%%%%%%%%%%%%%%%%%%%%%%%%%%%%%%%%%%%%%%%%%%%%%%%%%%%%
\subsection{Applying the effective Nullstellensatz}
Now we apply the Nullstellensatz for $\pjb$ with the ideal membership for $\IilfA.$ Let $T$ be a new variable; applying the
Nullstellensatz in $\C(\A)[\Xb,\Lb][T]$, and clearing
denominators, we obtain the existence of $\alpha_j$ in
$\C[\A]-\{0\}$ and $ C_{j,\ell},B_j$ in
$\C[\A][\Xb,\Lb][T]$, such that
\begin{align*}
\alpha_j = \sum_{\ell=1}^{p+n-i+1} C_{j,\ell} G_\ell + B_j (1-\pjb T),\\  G_\ell \in 
\left\{ 
F^\A,\lagFfA, \udl-1
\right\}.
\end{align*}
Let us then define 
$$\D_{i}:=\D_{i,1} \alpha_i \cdots \alpha_n D_i \cdots D_n.$$

\begin{lemma}\label{lem:6.4}
Suppose that $\mA \in \C^{n\times n}$ does not cancel $\D_{i}$. Then $F^{\mA}$ satisfies $\bm G_i(1)$ and $\bm G_i(2),$ and the extension
\[
 \C[\Xb_{\leq i-1}]\rightarrow \C[\Xb]/(\IilA \cap \C[\Xb])
\]
is integral.
\end{lemma}

\begin{proof}
By assumption, $\D_{i,1}(\mA)$ is non-zero so that $\mA$ is
invertible, the
ideal defined by $\IilA$ is radical (this follows from Corollary \ref{cor:LagIdealRadical}, with $\ub \in \C^p$ any complex point) and 
$\WiA$ is either empty or $(i-1)$-equidimensional. By Proposition \ref{prop:RadLagPolarV} and the Jacobian Criterion~\cite[Corollary 16.20]{ECA}, we have that $\WilA$ is also either empty or $(i-1)$-equidimensional. Now, if
it is empty, we are done. Otherwise, for $j=i,\dots,n-p+1$, evaluate all indeterminates in $\A$ at the
corresponding entries of $\mA$. This gives us
an equality in $\C[\Xb,\Lb,T]$ of the form
\begin{align*}
a_j = \sum_{\ell=1}^{p+n-i+1} c_{j,\ell} g_\ell + b_j (1-p_j T),\ \  g_\ell \in 
\left\{ 
F^{\mA}, \lagFA, \udl-1
\right\},
\end{align*}
for $a_j$ in $\C$, polynomials $c_{j,\ell}$ and $b_j$ in
$\C[\Xb,\Lb,T]$ and $p_j$ in
$\C[\Xb_{\leq i-1},X_j]$. Since neither $\alpha_j$ nor $D_j$
vanish at $\mA$, $a_j$ is non-zero and the leading coefficient of
$p_j$ in $X_j$ is a non-zero constant.

The conclusion is now routine. Replace $T$ by $1/p_j$ in the
previous equality; after clearing denominators, this gives a
membership equality of the form 
\[
p_j{}^k \in \IilA \cap \C[\Xb],
\]
for some integer $k \ge 1$ (we cannot have $k=0$, since we assumed that $\WiA$ is not empty, which implies that $\WilA$ is not empty). Since $\IilA$ is radical,
$p_j$ is in $\IilA$. Repeating this for all $j$ proves that 
\[
\C[\Xb_{\leq i-1}]\rightarrow\C[\Xb]/(\IilA \cap \C[\Xb])
\]
is integral.
\end{proof}

\noindent
To estimate the degree of $\D_{i}$, what remains is to give an upper
bound on the degrees of $\alpha_i,\dots,\alpha_n$. This will come as an
application of the effective Nullstellensatz given in~\cite{EN}, for
which we first need to determine degree bounds, separately in $\Xb,\Lb,T$
and $\A,$ of the polynomials in the membership relationship. We have
\begin{align*}
\deg_{\Xb,\Lb,T}
\left\{ 
F^{\A},\lagFfA, \udl-1
\right\}
\leq d, ~
\deg_{\Xb,\Lb,T}(1-T\pjb) \leq (2d)^{2n} +1, 
\end{align*}
and we have 
\begin{align*}
\deg_{\A}  
\left\{ 
F^{\A},\lagFfA, \udl-1
\right\} 
\leq d,~\textrm{and }
\deg_{\A}(1-T\pjb)& \leq (2d)^{2n}.
\end{align*}
For each $j \in \{i,\hdots,n-p+1\},$ a direct application of ($\td$ state assumptions: no solution; the number of equations in the ideal we consider is less than or equal to the ambient dimension) \cite[Theorem
0.5]{EN}, gives 
\[
\deg(\alpha_j) \le (2n+2)d^{2n+1}((2d)^{2n}+1);
\]
we will use
the slightly less precise bound \[\deg(\alpha_j) \le 4n(2d)^{4n}.\] Since $\D_{i,1}$ has degree at most
$nd^{5n}$ and all $D_j$'s have degree at most $(2d)^{2n}$, this gives
the upper bound
$$\deg(\D_i) \le nd^{5n} +  4n^2(2d)^{4n} + n(2d)^{2n} \leq 6n^2(2d)^{5n}.$$


%%%%%%%%%%%%%%%%%%%%%%%%%%%%%%%%%%%%%%%%%%%%%%%%%%%%%%%%%%%%
%%%%%%%%%%%%%%%%%%%%%%%%%%%%%%%%%%%%%%%%%%%%%%%%%%%%%%%%%%%%
%%%%%%%%%%%%%%%%%%%%%%%%%%%%%%%%%%%%%%%%%%%%%%%%%%%%%%%%%%%%
\subsection{Proof of $\textbf{G}_i(3)$}
Now assume that $\ub \in \C^p$ satisfies $\bm G_i^{''}$. It remains to show that if $\mA \in \C^{n\times n}$ does not cancel $\D_{i}$ then
\[
 \C[\Xb_{\leq i-1}]\rightarrow \C[\Xb]/\IiA
\]
is integral. 
\noindent
By Lemma \ref{lem:6.4}, the extension
\[
 \C[\Xb_{\leq i-1}]\rightarrow \C[\Xb]/(\IilA \cap \C[\Xb])
\]
is integral, and thus polynomials 
\[
Q_j \in \C[\Xb_{\leq i-1},T]
\]
exists, monic in $T$, for each $j \in \{i,\hdots,n-p+1\},$ with
\begin{align*}
Q_j(X_1,\hdots,X_{i-1},X_j) \in \IilA \cap \C[\Xb].
\end{align*}
%
Since we are assuming that $\ub \in \C^p$ satisfies $\bm G_i^{''}$, by Proposition~\ref{prop:polarVs1}, $Q_j \in \IiAr.$ Hence, there exists some $k \in \mathbb{N}-\{0\}$ with $Q_j^{k} \in\IiA$, where $Q_j^k$ is monic in $X_j,$ and therefore
\[
 \C[\Xb_{\leq i-1}]\rightarrow \C[\Xb]/\IiA
\]
is integral.
\par 
This completes the proof of Theorem \ref{theo:NoetherPositionG}.




%%%%%%%%%%%%%%%%%%%%%%%%%%%%%%%%%%%%%%%%%%%%%%%%%%%%%%%%%%%%
%%%%%%%%%%%%%%%%%%%%%%%%%%%%%%%%%%%%%%%%%%%%%%%%%%%%%%%%%%%%
%%%%%%%%%%%%%%%%%%%%%%%%%%%%%%%%%%%%%%%%%%%%%%%%%%%%%%%%%%%%
\section{Proof of Theoreom \ref{theo:main}}


%%%%%%%%%%%%%%%%%%%%%%%%%%%%%%%%%%%%%%%%%%%%%%%%%%%%%%%%%%%%
%%%%%%%%%%%%%%%%%%%%%%%%%%%%%%%%%%%%%%%%%%%%%%%%%%%%%%%%%%%%
%%%%%%%%%%%%%%%%%%%%%%%%%%%%%%%%%%%%%%%%%%%%%%%%%%%%%%%%%%%%

\subsection{Pseudocode}
%
The following is our main algorithm; it expands on the sketch given in
the introduction, by quantifying the various random choices.

In step 4, we use \cite[Algorithm 2]{SH} to solve a square
system. This subroutine is randomized; in order to guarantee a higher
probability of success, we repeat the calculation $k$ times, for a
well-chosen parameter $k$.

This subroutine also requires that the input system be given by a
straight-line program. We build it (at Step 3) in the straightforward
manner already suggested in the introduction: given $F=(f_1,\hdots,f_p)$ in $\C[X_1,\hdots,X_n]$, we can build
a straight-line program that evaluates each $f_i$ in $O(d^n)$ operations, by
computing all monomials of degree up to $d$, multiplying them by the
corresponding coefficients in $f_i$, and adding results. To obtain a
straight-line program for $f_i^\mA$, we add $O(n^2)$ steps corresponding
to the application of the change of variables $\mA$. The number of operations here is thus
\[
O(nd^n + n^3) = O^{\sim}(d^n).
\]
From this, we can compute and evaluate the required partial derivatives in the Jacobian of $F^\mA$ in
\[
O(n^2d^n) = O^{\sim}(d^n)
\]
operations ~\cite{BaSt83}.
Then, the matrix vector product with the vector of Lagrange multipliers adds a cost that is polynomial in $n$, and which we can therefore neglect in the soft oh notation. Finally, we add the linear equations
$X_1-\sigma_1,\hdots,X_{i-1}-\sigma_{i-1}$; this gives $\Gamma_i$, and the total cost for computing the straight line program is $O^{\sim}(d^n).$
%\newpage
%%%%%%%%%%%%%%%%%%%%%%%%%%%%%%%%%%%%%%
%%%%% ALGORITHM %
    %%%%%%%%%%%%%%%%%
    \begin{algorithm}[!h]
    \KwIn{$F=(f_1,\hdots,f_p) \in \ZZ[X_1,\hdots,X_n]^p$ with $\deg f_i \leq d$ and $\htt f_i \leq b$, and $0 < \epsilon < 1$}
    \KwOut{$n$ zero-dimensional parameterizations, the union of whose zeros
    includes at least one point in each connected component of $V(F) \cap \R^n$, with probability at least $1-\epsilon$} 

    \nl Construct $S := \{1,2,\hdots,\lceil 3\epsilon^{-1}6n^3(2d)^{5n} \rceil\}$ and $T :=
    \{1,2,\hdots,\lceil 3\epsilon^{-1}nd^{5n} \rceil \},$ and randomly choose $\mA \in
    S^{n^2}$ and $(\bm \sigma,\ub) \in T^{p+n-1}$\; 

    \caption{{Main Algorithm} \label{alg:1}} 

    \nl \For{$i\gets1$ \KwTo $n$}{
    \nl Build a straight-line program $\Gamma_i$ that computes the equations
    \[
    \left\{X_1-\sigma_1,\hdots,X_{i-1}-\sigma_{i-1},f_1^{\mA},\hdots,f_p^{\mA}, \sLA_1,\hdots,\sLA_{n-i},\sum_{i=1}^{c}u_iL_i-1 \right\}\;
    \]
    \nl Run \cite[Algorithm 2]{SH} $k \geq \lg(3n/\epsilon)$ times
    with input $\Gamma_i$\;

    \nl Let $\mathscr{Q}_i$ be the highest cardinality
    zero-dimensional parameterization returned in step 4\; 
    
    \nl  Compute a parameterization of the projection of $\mathscr{Q}_i$ onto the $\Xb$-space, and let $\mathscr{Q}^{'}_i$ denote this new parameterization \; 
    
    } 
    \nl  \Return $[\mathscr{Q}^{'}_1,\hdots,\mathscr{Q}^{'}_{n}]$.
    \end{algorithm}
    %%%%% ALGORITHM %
    %%%%%%%%%%%%%%%%%
    












\noindent 
If $F^\mA$ satisfies ${\bf H}_i$, and $F^\mA$ and
$(\bm \sigma,\ub)$ satisfy ${\bf H}'_i$ for all $i$, then
Theorem~2 in~\cite{EMP} establishes correctness.




%%%%%%%%%%%%%%%%%%%%%%%%%%%%%%%%%%%%%%%%%%%%%%%%%%%%%%%%%%%%
%%%%%%%%%%%%%%%%%%%%%%%%%%%%%%%%%%%%%%%%%%%%%%%%%%%%%%%%%%%%
%%%%%%%%%%%%%%%%%%%%%%%%%%%%%%%%%%%%%%%%%%%%%%%%%%%%%%%%%%%%
\subsection{Bit operation cost} 
%
The following lists the costs for each step of Algorithm \ref{alg:1}:

\noindent
    (1) We defined $S := \{1,2,\hdots,\lceil 3\epsilon^{-1}6n^3(2d)^{5n} \rceil \}$
      and therefore the height of any $a_{i,j} \in S$ is at most
    \[
    \log 3/{\epsilon} + \log(6n^3(2d)^{5n}) \in O^{\sim}(\log 1/{\epsilon} + n\log d).
    \]
Since $|T| < |S|,$ we also have that the height of any $b_{i,j} \in T$ is at most the same.
\newline 
\noindent  
    (3)
    After computing the partial derivatives in the Jacobian matrix, the height grows by at
    most another factor of $\log d$. Thus, all polynomials in the
    system considered at Step 3 have height 
    \[
    O^{\sim}(b + d(\log 1 /
    \epsilon + n\log d))
    =
        O^{\sim}(b + d\log 1 /
    \epsilon + dn).
    \]
    All integer coefficients appearing in $\Gamma_i$ 
    satisfy the same bound.    
    %% Furthermore, preparing the
    %% straight line program for the gradient of $f$ does not introduce
    %% large integers. The integers in $\grad (f)$ are of the same
    %% magnitude as in $\Gamma$.
     
\noindent  
    (4)   
    As a result, after applying \cite[Algorithm 2]{SH} $k$ times
      for each index $i$, with $k = O(\log n + \log 1 / \epsilon)$,
      the total boolean cost of the algorithm is
  \[
O^{\sim}(d^{3n+2p+1}(\log1/\epsilon)(b + \log1/\epsilon))
  \]
      where the polynomials in the output have degree at most $d^{n+p},$ and height at most
  \[
O^{\sim}(d^{n+p+1}(b + \log1/\epsilon)).
  \]
\noindent
This proves the runtime estimate, as well as our bounds on the height
of the output.




%%%%%%%%%%%%%%%%%%%%%%%%%%%%%%%%%%%%%%%%%%%%%%%%%%%%%%%%%%%%
%%%%%%%%%%%%%%%%%%%%%%%%%%%%%%%%%%%%%%%%%%%%%%%%%%%%%%%%%%%%
%%%%%%%%%%%%%%%%%%%%%%%%%%%%%%%%%%%%%%%%%%%%%%%%%%%%%%%%%%%%
\subsection{Probability of success} 
Let $\Delta_i \in \C[\A]$ be the polynomials from
Theorem~\ref{theo:gen}. Denote by $\Delta := \prod_{i=1}^n \D_i,$ and
note that
\begin{align}
    \deg \Delta \leq \sum_{i=1}^{n-p+1} \deg \Delta_i \leq 6n^3(2d)^{5n}.
\end{align}
If $\mA \in \C^{n \times n}$ does not cancel $\Delta,$ then $\mA$ is
invertible and $F^\mA$ satisfies $\textbf{H}_i$ for all $i$ in
$\{1,\hdots,n-p+1\}.$ Now, assuming that $\mA$ is such a matrix, let
$\Xi_i\in \C[S_1,\dots,S_{p+i-1}]$ be the polynomials from Theorem \ref{theo:Hp} applied to
$F^{\mA}.$ Denote by $\Xi := \prod_{i=1}^n \Xi_i,$ and note that
\begin{align}
    \deg \Xi \leq \sum_{i=1}^{n-p+1} \deg \Xi_i \leq nd^{5n}.
\end{align}
If $(\bm \sigma,\ub) \in \C^{p+i-1}$ does not cancel $\Xi$, then $F^{\mA}$ and
$(\bm \sigma,\ub)$ satisfy $\textbf{H}_i^{'}$ for all $i \in
\{1,\hdots,n-p+1\};$ as we argued above, the algorithm is guaranteed to
succeed, as long as our call to Algorithm 2 in~\cite{SH} succeeds. That latter reference establishes that by repeating the calculation $k$ times, and keeping the output of highest degree among those $k$ results, we succeed with probability at least $1-(1/2)^k$. When Algorithm 2 does not succeed, it either returns a proper subset of the solutions, or FAIL. Note that Algorithm 2 is shown to succeed in a single run with  probability at least $1-11/32,$ and we bound the probability of success with $1-1/2$ for simplicity. 
Now, by construction of
\[
S := \{1,2,\hdots,\lceil 3\epsilon^{-1}6n^3(2d)^{5n}\rceil \}
\] 
and 
\[
T := \{1,2,\hdots,\lceil 3\epsilon^{-1}nd^{5n} \rceil \},
\] 
where $\mA \in S^{n^2}$ and  $(\bm \sigma,\ub) \in T^{p+n-1}$ are randomly chosen, we have 
%
\[
\pr[\Delta(\mA)=0] \leq  \frac{\deg\Delta}{|S|} = \epsilon/3
\]
and
\[
\pr[\Xi(\bm \sigma,\ub)=0] \leq  \frac{\deg\Xi}{|T|} = \epsilon/3.
\]
Let $\mathscr{E}$ be the event that the parameterizations
$[\mathscr{Q}_1^{'},\hdots,\mathscr{Q}_n^{'}]$ returned in step 7 of
Algorithm \ref{alg:1} are correct. Then, the probability of success is equal to
\begin{align*}
 \pr[\Delta(\mA)\ne 0] \times \pr[\Xi(\bm \sigma,\ub)\ne 0 ~|~\Delta(\mA) \ne 0] \times \pr[\mathscr{E}~|~
   \Delta(\mA)\Xi(\bm \sigma,\ub) \ne 0].
\end{align*}
Set $k = \lg(3n/\epsilon)$ so that 
\[
(1-2^{-k})^n = (1 - \epsilon/(3n))^n \geq 1 - \epsilon/3,
\]
by Bernoulli's inequality. Therefore, 
\begin{align*}
\pr[\textrm{success}] &\geq (1- \epsilon/3)(1-
\epsilon/3)\pr[\mathscr{E}~|~   \Delta(\mA)\Xi(\bm \sigma,\ub) \ne 0]\\ 
& \geq (1- \epsilon/3)(1-
\epsilon/3)(1-2^{-k})^n \\ & \geq (1- \epsilon/3)(1- \epsilon/3)(1-
\epsilon/3)\\ & \geq 1 - \epsilon.
\end{align*}
This finishes the proof of our main theorem.






%\blinddocument
\bibliographystyle{plain}
\bibliography{refs.bib}
%\printbibliography

















\end{document}
